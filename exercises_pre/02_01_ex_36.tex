{Legend has it that a bull Sasquatch in rut will howl approximately 9 times per hour when it is $40^{\circ}F$ outside and only 5 times per hour if it's $70^{\circ}F$.  Assuming that the number of howls per hour, $N$, can be represented by a linear function of temperature Fahrenheit, find the number of howls per hour he'll make when it's only $20^{\circ}F$ outside.  What is the applied domain of this function?  Why?}
{$N(T) = -\frac{2}{15}T + \frac{43}{3}$  and $N(20) = \frac{35}{3} \approx 12$ howls per hour.

Having a negative number of howls makes no sense and since $N(107.5) = 0$ we can put an upper bound of $107.5^{\circ}F$ on the domain.  The lower bound is trickier because there's nothing other than common sense to go on.  As it gets colder, he howls more often.  At some point it will either be so cold that he freezes to death or he's howling non-stop.  So we're going to say that he can withstand temperatures no lower than $-60^{\circ}F$ so that the applied domain is $[-60, 107.5]$.
}