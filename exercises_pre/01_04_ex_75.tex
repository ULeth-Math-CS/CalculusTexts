{\label{greatestinteger} In Section \ref{SetsofNumbers} we defined the set of \index{integer ! greatest integer function}\sword{integers} as  $\mathbb{Z} = \{ \ldots, -3, -2, -1, 0, 1, 2, 3, \ldots\}$. The \index{greatest integer function}\sword{greatest integer of \boldmath{$x$}}, denoted by $\lfloor x \rfloor$, is defined to be the largest integer $k$ with $k \leq x$.

\textbf{Note:} The use of the letter $\mathbb{Z}$ for the integers is ostensibly because the German word \textit{zahlen} means `to count.'

\begin{enumerate}

\item  Find $\lfloor 0.785 \rfloor$, $\lfloor 117 \rfloor$, $\lfloor -2.001 \rfloor$, and $\lfloor \pi + 6 \rfloor$

\item  Discuss with your classmates how $\lfloor x \rfloor$ may be described as a piecewise defined function.

\smallskip

\textbf{HINT:}  There are infinitely many pieces!

\item  Is $\lfloor a + b \rfloor = \lfloor a \rfloor + \lfloor b \rfloor$ always true?  What if $a$ or $b$ is an integer?  Test some values, make a conjecture, and explain your result.

\end{enumerate}
}
{\begin{enumerate}

\item  $\lfloor 0.785 \rfloor = 0$, $\lfloor 117 \rfloor = 117$, $\lfloor -2.001 \rfloor = -3$, and $\lfloor \pi + 6 \rfloor = 9$

\end{enumerate}
}