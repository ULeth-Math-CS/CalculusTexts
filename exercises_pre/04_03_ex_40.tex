{According to the Centers for Disease Control and Prevention \href{http://www.cdc.gov}{\underline{www.cdc.gov}}, a person's Body Mass Index $B$ is directly proportional to his weight $W$ in pounds and inversely proportional to the square of his height $h$ in inches. \index{BMI, body mass index}

\begin{enumerate}

\item Express this relationship as a mathematical equation. \label{BMIfirst} 
\item If a person who was $5$ feet, $10$ inches tall weighed 235 pounds had a Body Mass Index of 33.7, what is the value of the constant of proportionality? \label{BMIsecond}
\item Rewrite the mathematical equation found in part \ref{BMIfirst} to include the value of the constant found in part \ref{BMIsecond} and then find your Body Mass Index.

\end{enumerate}}
{\begin{enumerate}
\item $B = \dfrac{kW}{h^{2}}$
\item $k = 702.68$ (The CDC uses 703.)
\item $B = \dfrac{702.68W}{h^{2}}$
\end{enumerate}
}