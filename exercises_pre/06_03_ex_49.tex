{In Example \ref{expfracinverse}, we found that the inverse of $f(x) = \dfrac{5e^{x}}{e^{x}+1}$ was $f^{-1}(x) = \ln\left(\dfrac{x}{5-x}\right)$ but we left a few loose ends for you to tie up.  

\begin{enumerate}

\item Show that $\left(f^{-1} \circ f\right)(x) = x$ for all $x$ in the domain of $f$ and that $\left(f \circ f^{-1}\right)(x) = x$ for all $x$ in the domain of $f^{-1}$.

\item Find the range of $f$ by finding the domain of $f^{-1}$.

\item Let $g(x) = \dfrac{5x}{x+1}$ and $h(x) = e^{x}$.  Show that $f = g \circ h$ and that $(g \circ h)^{-1} = h^{-1} \circ g^{-1}$. 
(We know this is true in general by Exercise \ref{fcircginverse} in Section \ref{InverseFunctions}, but it's nice to see a specific example of the property.)

\end{enumerate}}
{}