{Yeast is often used in biological experiments.  A research technician estimates that a sample of yeast suspension contains 2.5 million organisms per cubic centimetre (cc).  Two hours later, she estimates the population density to be 6 million organisms per cc.  Let $t$ be the time elapsed since the first observation, measured in hours.  Assume that the yeast growth follows the Law of Uninhibited Growth $N(t) = N_{\text{\tiny $0$}} e^{kt}$.

\begin{enumerate}

\item  Find the growth constant $k$. Round your answer to four decimal places.

\item  Find a function which gives the number of yeast (in millions) per cc $N(t)$ after $t$ hours.

\item  What is the doubling time for this strain of yeast?

\end{enumerate}}
{\begin{enumerate} \item  $k = \frac{1}{2}\frac{\ln(6)}{2.5} \approx 0.4377$

\item  $N(t) = 2.5e^{0.4377 t}$

\item  $t = \frac{\ln(2)}{0.4377} \approx 1.58$ hours

\end{enumerate}}