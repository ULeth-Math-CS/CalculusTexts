{The volume $V$ enclosed by a cube, in cubic centimetres, is a function of the length of one of its sides $x$, when measured in centimetres.  This relation is expressed by the formula $V(x) = x^3$ for $x > 0$.  Find $V(5)$ and solve $V(x) = 27$.  Interpret your answers to each.  Why is $x$ restricted to $x > 0$?}
{$V(5) = 125$, so the volume enclosed by a cube with a side of length $5$ centimetres is $125$ cubic centimetres.  The solution to $V(x) = 27$ is $x = 3$.  This means for the volume enclosed by the cube to be $27$ cubic centimetres, the length of the side needs to $3$ centimetres.  Since $x$ represents a length, $x > 0$.}