{We shall now prove that $y = m_{\mbox{\tiny$1$}}x + b_{\mbox{\tiny$1$}}$ is perpendicular to $y = m_{\mbox{\tiny$2$}}x + b_{\mbox{\tiny$2$}}$ if and only if $m_{\mbox{\tiny$1$}} \cdot m_{\mbox{\tiny$2$}} = -1$.  To make our lives easier we shall assume that $m_{\mbox{\tiny$1$}} > 0$ and $m_{\mbox{\tiny$2$}} < 0$.  We can also ``move'' the lines so that their point of intersection is the origin without messing things up, so we'll assume $b_{\mbox{\tiny$1$}} = b_{\mbox{\tiny$2$}} = 0.$  (Take a moment with your classmates to discuss why this is okay.)  Graphing the lines and plotting the points $O(0, 0)\;$, $P(1, m_{\mbox{\tiny$1$}})\;$ and $Q(1, m_{\mbox{\tiny$2$}})$ gives us the following set up. \label{perpendicularlineproof}

\begin{center}
\myincludegraphics{figures/LinearQuadraticGraphics/LinearFunctions-16}
\end{center}

The line $y = m_{\mbox{\tiny$1$}}x$ will be perpendicular to the line $y = m_{\mbox{\tiny$2$}}x$ if and only if $\bigtriangleup OPQ$ is a right triangle.  Let $d_{\mbox{\tiny$1$}}$ be the distance from $O$ to $P$, let $d_{\mbox{\tiny$2$}}$ be the distance from $O$ to $Q$ and let $d_{\mbox{\tiny$3$}}$ be the distance from $P$ to $Q$.  Use the Pythagorean Theorem to show that $\bigtriangleup OPQ$ is a right triangle if and only if $m_{\mbox{\tiny$1$}} \cdot m_{\mbox{\tiny$2$}} = -1$ by showing $d_{\mbox{\tiny$1$}}^{2} + d_{\mbox{\tiny$2$}}^{2} = d_{\mbox{\tiny$3$}}^2$ if and only if $m_{\mbox{\tiny$1$}} \cdot m_{\mbox{\tiny$2$}} = -1$.  }
{}