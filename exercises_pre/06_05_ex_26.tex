{\label{PainesvillePopulationTwoPoint} During the early years of a community, it is not uncommon for the population to grow according to the Law of Uninhibited Growth.  According to the Painesville Wikipedia entry, in 1860, the Village of Painesville had a population of 2649.  In 1920, the population was 7272.  Use these two data points to fit a model of the form $N(t) = N_{\text{\tiny $0$}} e^{kt}$ were $N(t)$ is the number of Painesville Residents $t$ years after 1860.  (Use $t = 0$ to represent the year 1860.  Also, round the value of $k$ to four decimal places.)  According to this model, what was the population of Painesville in 2010?  (The 2010 census gave the population as 19,563) What could be some causes for such a vast discrepancy?  }
{$N_{\text{\tiny $0$}} = 2649$,  $k = \frac{1}{60} \ln\left( \frac{7272}{2649}\right) \approx 0.0168$, $N(t) = 2649e^{0.0168t}$.  $N(150) \approx 32923$, so the population of Painesville in 2010 based on this model would have been 32,923.}