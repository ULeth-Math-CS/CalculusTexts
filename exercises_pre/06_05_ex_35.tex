{\label{catenary} In Exercise \ref{parabolicbridgecable} in Section \ref{QuadraticFunctions}, we stated that the cable of a suspension bridge formed a parabola but that a free hanging cable did not.  A free hanging cable forms a \underline{catenary} and its basic shape is given by $y = \frac{1}{2}\left(e^{x} + e^{-x}\right)$.  Use your calculator to graph this function.  What are its domain and range?  What is its end behaviour?  Is it invertible?  How do you think it is related to the function given in Exercise \ref{hyperbolicsine} in Section \ref{ExpEquations} and the one given in the answer to Exercise \ref{inversehyptangent} in Section \ref{LogEquations}?  When flipped upside down, the catenary makes an arch.  The Gateway Arch in St. Louis, Missouri has the shape \[y = 757.7 - \frac{127.7}{2}\left(e^{\frac{x}{127.7}} + e^{-\frac{x}{127.7}}\right)\] where $x$ and $y$ are measured in feet and $-315 \leq x \leq 315$.  Find the highest point on the arch.}
{}