{\item We have through our examples tried to convince you that, in general, $f(a + b) \neq f(a) + f(b)$.  It has been our experience that students refuse to believe us so we'll try again with a different approach.  With the help of your classmates, find a function $f$ for which the following properties are always true.

\begin{enumerate}

\item $f(0) = f(-1 + 1) = f(-1) + f(1)$
\item $f(5) = f(2 + 3) = f(2) + f(3)$
\item $f(-6) = f(0 - 6) = f(0) - f(6)$
\item $f(a + b) = f(a) + f(b)\;$ regardless of what two numbers we give you for $a$ and  $b$.

\end{enumerate}

How many functions did you find that failed to satisfy the conditions above?  Did $f(x) = x^{2}$ work?  What about $f(x) = \sqrt{x}$ or $f(x) = 3x + 7$ or $f(x) = \dfrac{1}{x}$?  Did you find an attribute common to those functions that did succeed?  You should have, because there is only one extremely special family of functions that actually works here.  Thus we return to our previous statement, {\bf in general}, $f(a + b) \neq f(a) + f(b)$.

}
{}