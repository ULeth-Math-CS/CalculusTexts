{According to \href{http://en.wikipedia.org/wiki/Vibrating_string}{\underline{this webpage}}, the frequency $f$ of a vibrating string is given by $f = \dfrac{1}{2L} \sqrt{\dfrac{T}{\mu}}$ where $T$ is the tension, $\mu$ is the linear mass\footnote{Also known as the linear density.  It is simply a measure of mass per unit length.} of the string and $L$ is the length of the vibrating part of the string.  Express this relationship using the language of variation.}
{ Rewriting $f = \dfrac{1}{2L} \sqrt{\dfrac{T}{\mu}}$ as $f = \dfrac{\frac{1}{2} \sqrt{T}}{L \sqrt{\mu}}$ we see that the frequency $f$ varies directly with the square root of the tension and varies inversely with the length and the square root of the linear mass.}