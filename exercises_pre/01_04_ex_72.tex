{\label{piecewiseordering} For $n$ copies of the book \textit{Me and my Sasquatch}, a print on-demand company charges $C(n)$ dollars, where $C(n)$ is determined by the formula \[{\displaystyle C(n) = \left\{ \begin{array}{rcl}  15n & \mbox{ if } & 1 \leq n \leq 25  \\
                                                            13.50n  & \mbox{ if } & 25 < n \leq 50 \\
                                                            12n & \mbox{ if } & n > 50 \\
                                     \end{array} \right. }\]
                                     
                                     
\begin{enumerate}

\item  Find and interpret $C(20)$.  % Ans:  $C(20) = 300$.  It costs $\$300$ for 20 copies of the book.

\item  \label{50vs51} How much does it cost to order 50 copies of the book?  What about 51 copies? %  Ans:  $C(50) = 675$, $\$ 675$.  $C(51) = 612$, $\$ 612$.

\item  Your answer to \ref{50vs51} should get you thinking. Suppose a bookstore estimates it will sell 50 copies of the book.  How many books can, in fact, be ordered for the same price as those 50 copies? (Round your answer to a  whole number of books.)  % Ans:  56 books.

\end{enumerate}
 }
{\begin{enumerate}

\item $C(20) = 300$.  It costs $\$300$ for 20 copies of the book.

\item $C(50) = 675$, so it costs $\$ 675$ for 50 copies of the book.  $C(51) = 612$, so it costs $\$ 612$ for 51 copies of the book.

\item $56$ books.

\end{enumerate}}