{The Law of Uninhibited Growth also applies to situations where an animal is re-introduced into a suitable environment.  Such a case is the reintroduction of wolves to Yellowstone National Park.   According to the \href{http://www.nps.gov/yell/naturescience/wolves.htm}{\underline{National Park Service}}, the wolf population in Yellowstone National Park was 52 in 1996 and 118 in 1999.  Using these data, find a function of the form $N(t) = N_{\text{\tiny $0$}} e^{kt}$  which models the number of wolves $t$ years after 1996.  (Use $t = 0$ to represent the year 1996.  Also, round your value of $k$ to four decimal places.)  According to the model, how many wolves were in Yellowstone in 2002?  (The recorded number is 272.)}
{$N_{\text{\tiny $0$}} = 52$,  $k = \frac{1}{3} \ln\left( \frac{118}{52}\right) \approx 0.2731$, $N(t) = 52e^{0.2731t}$.  $N(6) \approx 268$. }