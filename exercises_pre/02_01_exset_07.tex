{\phantomsection
\label{perpendicularlines}

\noindent (Perpendicular Lines) \index{line ! perpendicular} Recall from high school that two non-vertical lines are perpendicular if and only if they have negative reciprocal slopes.  That is to say, if one line has slope $m_{\mbox{\tiny$1$}}$ and the other has slope $m_{\mbox{\tiny$2$}}$ then $m_{\mbox{\tiny$1$}} \cdot m_{\mbox{\tiny$2$}} = -1$.  (You will be guided through a proof of this result in Exercise \ref{perpendicularlineproof}.)  Please note that a horizontal line is perpendicular to a vertical line and vice versa, so we assume $m_{\mbox{\tiny$1$}} \neq 0$ and $m_{\mbox{\tiny$2$}} \neq 0$. In Exercises}
{, you are given a line and a point which is not on that line.  Find the line perpendicular to the given line which passes through the given point.
}
\exinput{exercises_pre/02_01_ex_65}
\exinput{exercises_pre/02_01_ex_66}
\exinput{exercises_pre/02_01_ex_67}
\exinput{exercises_pre/02_01_ex_68}
\exinput{exercises_pre/02_01_ex_69}
\exinput{exercises_pre/02_01_ex_70}