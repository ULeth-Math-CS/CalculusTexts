{\label{Richterexercise} \index{Richter Scale} \index{earthquake ! Richter Scale} Earthquakes are complicated events and it is not our intent to provide a complete discussion of the science involved in them.  Instead, we refer the interested reader to a solid course in Geology\footnote{Rock-solid, perhaps?} or the U.S. Geological Survey's Earthquake Hazards Program found \href{http://earthquake.usgs.gov/}{\underline{here}} and present only a simplified version of the \href{http://en.wikipedia.org/wiki/Richter_scale}{\underline{Richter scale}}.  The Richter scale measures the magnitude of an earthquake by comparing the amplitude of the seismic waves of the given earthquake to those of a ``magnitude 0 event'', which was chosen to be a seismograph reading of $0.001$ millimetres recorded on a seismometer 100 kilometres from the earthquake's epicentre.  Specifically, the magnitude of an earthquake is given by \[M(x) = \log \left(\dfrac{x}{0.001}\right)\] where $x$ is the seismograph reading in millimetres of the earthquake recorded 100 kilometres from the epicentre.  

\begin{enumerate}

\item Show that $M(0.001) = 0$.
\item Compute $M(80,000)$.
\item Show that an earthquake which registered 6.7 on the Richter scale had a seismograph reading ten times larger than one which measured 5.7.
\item Find two news stories about recent earthquakes which give their magnitudes on the Richter scale.  How many times larger was the seismograph reading of the earthquake with larger magnitude?

\end{enumerate}}
{ \begin{enumerate}

\item $M(0.001) = \log \left(\frac{0.001}{0.001} \right) = \log(1) = 0$.
\item $M(80,000) = \log \left(\frac{80,000}{0.001} \right) = \log(80,000,000) \approx 7.9$.

\end{enumerate}}