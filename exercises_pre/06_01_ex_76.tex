{\label{decibelexercise} \index{decibel} \index{sound intensity level ! decibel} While the decibel scale can be used in many disciplines,\footnote{See this  \href{http://en.wikipedia.org/wiki/Decibel}{\underline{webpage}} for more information.} we shall restrict our attention to its use in acoustics, specifically its use in measuring the intensity level of sound.\footnote{As of the writing of this exercise, the Wikipedia page given \href{http://en.wikipedia.org/wiki/Sound_intensity_level}{\underline{here}} states that it may not meet the ``general notability guideline'' nor does it cite any references or sources.  I find this odd because it is this very usage of the decibel scale which shows up in every College Algebra book I have read.  Perhaps those other books have been wrong all along and we're just blindly following tradition.}  The Sound Intensity Level $L$ (measured in decibels) of a sound intensity $I$ (measured in watts per square meter) is given by \[L(I) = 10\log\left( \dfrac{I}{10^{-12}} \right).\] Like the Richter scale, this scale compares $I$ to baseline: $10^{-12} \frac{W}{m^{2}}$ is the threshold of human hearing. 

\begin{enumerate}

\item Compute $L(10^{-6})$.
\item Damage to your hearing can start with short term exposure to sound levels around 115 decibels.  What intensity $I$ is needed to produce this level? 
\item Compute $L(1)$.  How does this compare with the threshold of pain which is around 140 decibels?

\end{enumerate}}
{ \begin{enumerate}

\item $L(10^{-6}) = 60$ decibels.
\item $I = 10^{-.5} \approx 0.316$ watts per square meter.
\item Since $L(1) = 120$ decibels and $L(100) = 140$ decibels, a sound with intensity level 140 decibels has an intensity 100 times greater than a sound with intensity level 120 decibels.

\end{enumerate}}