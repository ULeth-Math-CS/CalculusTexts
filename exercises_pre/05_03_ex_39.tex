{As a follow-up to Exercise \ref{WindChillTemperature}, suppose the air temperature is $28^{\circ}$F.  

\begin{enumerate}

\item Use the formula from Exercise \ref{WindChillTemperature} to find an expression for the wind chill temperature as a function of the wind speed, $W(V)$.  

\item  \label{WindChill0} Solve $W(V) = 0$, round your answer to two decimal places,  and interpret.  

\item  Graph the function $W$ using your calculator and check your answer to part \ref{WindChill0}. 


\end{enumerate}}
{\begin{enumerate}

\item $W(V) = 53.142 - 23.78 V^{0.16}$.  Since we are told in Exercise \ref{WindChillTemperature} that wind chill is only effect for wind speeds of more than 3 miles per hour, we restrict the domain to $V > 3$.

\item $W(V)=0$ when $V \approx 152.29$.  This means, according to the model, for the wind chill temperature to be $0^{\circ}$F, the wind speed needs to be $152.29$ miles per hour.

\item The graph is below.  \\
\centerline{\myincludegraphics[width=0.5\columnwidth]{figures/FurtherGraphics/WINDCHILL}}


\end{enumerate}}