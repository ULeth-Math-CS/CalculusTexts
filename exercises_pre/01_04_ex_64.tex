{The area $A$ enclosed by a circle, in square meters, is a function of its radius $r$, when measured in meters.  This relation is expressed by the formula $A(r) = \pi r^2$ for $r > 0$.  Find $A(2)$ and solve $A(r) = 16\pi$.  Interpret your answers to each.  Why is $r$ restricted to $r > 0$?}
{$A(2) = 4\pi$, so the area enclosed by a circle with radius $2$ meters is $4\pi$ square meters.  The solutions to $A(r) = 16\pi$ are $r = \pm 4$.  Since $r$ is restricted to $r > 0$, we only keep $r = 4$.  This means for the area enclosed by the circle to be $16\pi$ square meters, the radius needs to be $4$ meters.  Since $r$ represents a radius (length), $r > 0$.}
