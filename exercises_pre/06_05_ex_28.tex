{The half-life of the radioactive isotope Carbon-14 is about 5730 years.  

\begin{enumerate}

\item Use Equation \ref{radioactivedecay} to express the amount of Carbon-14 left from an initial $N$ milligrams as a function of time $t$ in years.

\item What percentage of the original amount of Carbon-14 is left after 20,000 years?

\item If an old wooden tool is found in a cave and the amount of Carbon-14 present in it is estimated to be only 42\% of the original amount, approximately how old is the tool?

\item Radiocarbon dating is not as easy as these exercises might lead you to believe.  With the help of your classmates, research radiocarbon dating and discuss why our model is somewhat over-simplified.  

\end{enumerate}}
{\begin{enumerate}

\item $A(t) = Ne^{-\left(\frac{\ln(2)}{5730}\right)t} \approx Ne^{-0.00012097t}$
\item $A(20000) \approx 0.088978 \cdot N$ so about 8.9\% remains
\item $t \approx \dfrac{\ln(.42)}{-0.00012097} \approx 7171$ years old

\end{enumerate}}