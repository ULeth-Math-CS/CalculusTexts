{On page \pageref{angleofelevation} we defined the angle of inclination (also known as the angle of elevation) and in this exercise we introduce a related angle - \index{angle ! of depression} the angle of depression (also known as \index{angle ! of declination} the angle of declination).  The angle of depression of an object refers to the angle whose initial side is a horizontal line above the object and whose terminal side is the line-of-sight to the object below the horizontal.  This is represented schematically below.
\label{angleofdepression}

\begin{center}
\myincludegraphics{figures/IntroTrigGraphics/CircularFunctions-15}

The angle of depression from the horizontal to the object is $\theta$

\end{center}

\begin{enumerate}

\item Show that if the horizontal is above and parallel to level ground then the angle of depression (from observer to object) and the angle of inclination (from object to observer) will be congruent because they are alternate interior angles.

\item \label{sasquatchfire} From a firetower 200 feet above level ground in the Sasquatch National Forest, a ranger spots a fire off in the distance.  The angle of depression to the fire is $2.5^{\circ}$.  How far away from the base of the tower is the fire?

\item  The ranger in part \ref{sasquatchfire} sees a Sasquatch running directly from the fire towards the firetower.  The ranger takes two sightings.  At the first sighting, the angle of depression from the tower to the Sasquatch is $6^{\circ}$.  The second sighting, taken just 10 seconds later, gives the the angle of depression as $6.5^{\circ}$.  How far did the Saquatch travel in those 10 seconds?  Round your answer to the nearest foot.  How fast is it running in miles per hour? Round your answer to the nearest mile per hour.  If the Sasquatch keeps up this pace, how long will it take for the Sasquatch to reach the firetower from his location at the second sighting?  Round your answer to the nearest minute.

\end{enumerate}}
{\begin{enumerate}

\addtocounter{enumii}{1}

\item The fire is about 4581 feet from the base of the tower.

\item  The Sasquatch ran $200\cot(6^{\circ}) - 200\cot(6.5^{\circ}) \approx 147$ feet in those 10 seconds. This translates to $\approx 10$ miles per hour.  At the scene of the second sighting, the Sasquatch was $\approx 1755$ feet from the tower, which means, if it keeps up this pace, it will reach the tower in about $2$ minutes.

\end{enumerate}}