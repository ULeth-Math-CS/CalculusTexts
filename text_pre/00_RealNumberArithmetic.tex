\section{Real Number Arithmetic}
\label{RealNumberArithmetic}


In this section we list the properties of real number arithmetic.  This is meant to be a succinct, targeted review so we'll resist the temptation to wax poetic about these axioms and their subtleties and refer the interested reader to a more formal course in Abstract Algebra.  There are two (primary) operations one can perform with real numbers:  addition and multiplication.  

\medskip

\definition{realnumberaddition}{Properties of Real Number Addition}{

\begin{itemize}

\item  \textbf{Closure:}  For all real numbers $a$ and $b$,  $a+b$ is also a real number.

\item  \textbf{Commutativity:}  For all real numbers $a$ and $b$, $a+b = b+a$.

\item  \textbf{Associativity:}  For all real numbers $a$, $b$ and $c$, $a+(b+c) = (a+b)+c$.

\item  \textbf{Identity:}  There is a real number `$0$' so that for all real numbers $a$, $a+0 = a$.

\item  \textbf{Inverse:}  For all real numbers $a$, there is a real number $-a$ such that $a + (-a) = 0$.

\item \textbf{Definition of Subtraction:}  For all real numbers $a$ and $b$, $a - b = a + (-b)$.

\end{itemize}
}

\medskip

Next, we give real number multiplication a similar treatment.  Recall that we may denote the product of two real numbers $a$ and $b$ a variety of ways:  $ab$, $a \cdot b$, $a(b)$, $(a)(b)$ and so on.  We'll refrain from using $a \times b$ for real number multiplication in this text.

\medskip

\definition{realnumbermultiplication}{Properties of Real Number Multiplication}{

\begin{itemize}

\item  \textbf{Closure:}  For all real numbers $a$ and $b$,  $ab$ is also a real number.

\item  \textbf{Commutativity:}  For all real numbers $a$ and $b$, $ab = ba$.

\item  \textbf{Associativity:}  For all real numbers $a$, $b$ and $c$, $a(bc) = (ab)c$.

\item  \textbf{Identity:}  There is a real number `$1$' so that for all real numbers $a$, $a \cdot 1 = a$.

\item  \textbf{Inverse:}  For all real numbers $a \neq 0$, there is a real number $\dfrac{1}{a}$ such that $a \left(\dfrac{1}{a}\right) = 1$.

\item \textbf{Definition of Division:}  For all real numbers $a$ and $b \neq 0$, $a \div b = \dfrac{a}{b} = a  \left(\dfrac{1}{b}\right)$.
\end{itemize}
}

\medskip

While most students (and some faculty) tend to skip over these properties or give them a cursory glance at best, it is important to realize that the properties stated above are what drive the symbolic manipulation for all of Algebra.  When listing a tally of more than two numbers, $1 + 2 + 3$\label{howtoaddonetwothree} for example, we don't need to specify the order in which those numbers are added. Notice though, try as we might, we can add only two numbers at a time and it is the associative property of addition which assures us that we could organize this sum as $(1+2) + 3$ or $1+(2+3)$.  This brings up a note about `grouping symbols'.  Recall that parentheses and brackets are used in order to specify which operations are to be performed first.  In the absence of such grouping symbols, multiplication (and hence division) is given priority over addition (and hence subtraction). For example, $1 + 2 \cdot 3 = 1+6 = 7$, but $(1+2) \cdot 3 = 3 \cdot 3 = 9$.  As you may recall, we can `distribute' the $3$ across the addition if we really wanted to do the multiplication first:  $(1+2) \cdot 3 = 1\cdot 3 + 2 \cdot 3 = 3 + 6 = 9$. More generally, we have the following.

\medskip

\definition{distributiveproperty}{The Distributive Property and Factoring}{
%\smallskip
For all real numbers $a$, $b$ and $c$:

\begin{itemize}

\item  \textbf{Distributive Property:}   $a(b+c) = ab + ac$ and $(a+b)c = ac + bc$.

\item  \textbf{Factoring:}  $ab+ac = a(b+c)$ and $ac + bc = (a+b)c$.

\end{itemize}
}

\medskip

\noindent {\bf Warning:} A common source of errors for beginning students is the misuse (that is, lack of use) of parentheses. When in doubt, more is better than less: redundant parentheses add clutter, but do not change meaning, whereas writing $2x+1$ when you meant to write $2(x+1)$ is almost guaranteed to cause you to make a mistake. (Even if you're able to proceed correctly in spite of your lack of proper notation, this is the sort of thing that will get you on your grader's bad side, so it's probably best to avoid the problem in the first place.)

\medskip


It is worth pointing out that we didn't really need to list the Distributive Property both for $a(b+c)$ (distributing from the left) and $(a+b)c$ (distributing from the right), since the commutative property of multiplication gives us one from the other.  Also, `factoring' really is the same equation as the distributive property, just read from right to left. These are the first of many redundancies in this section, and they exist in this review section for one reason only - in our experience, many students \textit{see} these things differently so we will list them as such.   

\smallskip

It is hard to overstate the importance of the Distributive Property.  For example, in the expression $5(2+x)$, without knowing the value of $x$, we cannot perform the addition inside the parentheses first;  we must rely on the distributive property here to get  $5(2+x) = 5\cdot 2 + 5 \cdot x = 10 + 5x$.  The Distributive Property is also responsible for combining `like terms'.  Why is $3x + 2x = 5x$?  Because  $3x + 2x = (3+2)x = 5x$.  

\smallskip

We continue our review with summaries of other properties of arithmetic, each of which can be derived from the properties listed above.  First up are properties of the additive identity $0$.

\medskip

\theorem{propertiesofzero}{Properties of Zero}{

Suppose $a$ and $b$ are real numbers.

\begin{itemize}

\item  \textbf{Zero Product Property:} $ab = 0$ if and only if $a=0$ or $b=0$ (or both)

\textbf{Note:} This not only says that $0 \cdot a = 0$ for any real number $a$, it also says that the \textit{only} way to get an answer of `$0$' when multiplying two real numbers  is to have one (or both) of the numbers be `$0$' in the first place.

\item  \textbf{Zeros in Fractions:}  If $a \neq 0$, $\dfrac{0}{a} = 0 \cdot \left(\dfrac{1}{a}\right) = 0$.

\textbf{Note:}  The quantity $\dfrac{a}{0}$ is undefined.
\end{itemize}
}

\mnote{.8}{
The Zero Product Property drives most of the equation solving algorithms in Algebra because it allows us to take complicated equations and reduce them to simpler ones.  For example, you may recall that one way to solve  $x^2+x-6=0$ is by factoring the left hand side of this equation to get  $(x-2)(x+3) = 0$.  From here, we apply the Zero Product Property and set each factor equal to zero.  This yields  $x-2=0$ or $x+3=0$ so $x=2$ or $x=-3$.  This application to solving equations leads, in turn,  to some deep and profound structure theorems in Chapter \ref{Polynomials}. 
}

\mnote{.65}{The expression $\frac{0}{0}$ is technically an `indeterminate form' as opposed to being strictly `undefined' meaning that with Calculus we can make some sense of it in certain situations.  We'll talk more about this in Chapter \ref{Rationals}.}



\medskip

We now continue with a review of arithmetic with fractions.


\keyidea{fractionarithmetic}{Properties of Fractions}{
Suppose $a$, $b$, $c$ and $d$ are real numbers.  Assume them to be nonzero whenever necessary; for example,  when they appear in a denominator.

\begin{itemize}

\item  \textbf{Identity Properties:}  $a = \dfrac{a}{1}$ and $\dfrac{a}{a} = 1$.

\item  \textbf{Fraction Equality:}  $\dfrac{a}{b} = \dfrac{c}{d}$ if and only if $ad = bc$. 

\item  \textbf{Multiplication of Fractions:}  $\dfrac{a}{b} \cdot \dfrac{c}{d} = \dfrac{ac}{bd}$. In particular:  $\dfrac{a}{b} \cdot c = \dfrac{a}{b} \cdot \dfrac{c}{1} = \dfrac{ac}{b}$

\mnote{.4}{\textbf{Note:}  A common denominator is \textbf{not} required to \textbf{multiply} or \textbf{divide} fractions!}

\item  \textbf{Division of Fractions:}  $\dfrac{a}{b} \left/ \dfrac{c}{d}\right. = \dfrac{a}{b} \cdot \dfrac{d}{c} = \dfrac{ad}{bc}$. 

In particular: $1 \left/ \dfrac{a}{b}\right. = \dfrac{b}{a}$ and  $\left.\dfrac{a}{b} \right/ c  = \dfrac{a}{b} \left/ \dfrac{c}{1}\right.  = \dfrac{a}{b} \cdot \dfrac{1}{c} = \dfrac{a}{bc}$

\mnote{.48}{It's always worth remembering that division is the same as multiplication by the reciprocal. You'd be surprised how often this comes in handy.}

\item  \textbf{Addition and Subtraction of Fractions:}  $\dfrac{a}{b} \pm \dfrac{c}{b} = \dfrac{a \pm c}{b}$.  

\mnote{.36}{\textbf{Note:}  A common denominator \textbf{is} required to \textbf{add or subtract} fractions!}

\item  \textbf{Equivalent Fractions:}  $\dfrac{a}{b} = \dfrac{ad}{bd}$, since $ \dfrac{a}{b} = \dfrac{a}{b} \cdot 1 = \dfrac{a}{b} \cdot \dfrac{d}{d} = \dfrac{ad}{bd}$

\mnote{.28}{\textbf{Note:}  The \textit{only} way to change the denominator is to multiply both it and the numerator by the same nonzero value because we are, in essence, multiplying the fraction by $1$.}

\item  \textbf{`Reducing' Fractions:} $\dfrac{a\cancel{d}}{b\cancel{d}} = \dfrac{a}{b}$, since  $\dfrac{ad}{bd} = \dfrac{a}{b} \cdot \dfrac{d}{d} = \dfrac{a}{b} \cdot 1 = \dfrac{a}{b}$.

In particular, $\dfrac{ab}{b} = a$ since $\dfrac{ab}{b} = \dfrac{ab}{1 \cdot b} =  \dfrac{a \cancel{b}}{1 \cdot \cancel{b}} = \dfrac{a}{1} = a$ and $\dfrac{b-a}{a-b} = \dfrac{(-1)\cancel{(a-b)}}{\cancel{(a-b)}} = -1$.

\mnote{.2}{We reduce fractions by `cancelling' common factors - this is really just reading the previous property `from right to left'.\textbf{Caution:}  We may only cancel common \textbf{factors} from both numerator and denominator.}
\end{itemize}
}

\medskip


Next up is a review of the arithmetic of `negatives'. On page \pageref{realnumberaddition} we first introduced the dash which we all recognize as the `negative' symbol in terms of the additive inverse.  For example, the number $-3$ (read `negative $3$') is defined so that $3 + (-3) = 0$.  We then defined subtraction using the concept of the additive inverse again so that, for example, $5 - 3 = 5 + (-3)$.  



\medskip

\keyidea{propertiesofnegatives}{Properties of Negatives}{
Given real numbers $a$ and $b$ we have the following.  

\begin{itemize}

\item  \textbf{Additive Inverse Properties:}  $-a = (-1)a$ and $-(-a) = a$

\item  \textbf{Products of Negatives:} $(-a)(-b) = ab$. 

\item  \textbf{Negatives and Products:} $-ab = -(ab) = (-a)b = a(-b)$.

\item  \textbf{Negatives and Fractions:} If $b$ is nonzero, $-\dfrac{a}{b} = \dfrac{-a}{b} = \dfrac{a}{-b}$ and $\dfrac{-a}{-b} = \dfrac{a}{b}$.

\item  \textbf{`Distributing' Negatives:}  $-(a+b) = -a-b$ and $-(a-b) = -a + b = b-a$.

\item  \textbf{`Factoring' Negatives:} $-a-b = -(a+b)$ and $b-a = -(a-b)$.

\end{itemize}
}

\mnote{.5}{In this text we do not distinguish typographically between the dashes in the expressions `$5-3$' and `$-3$' even though they are mathematically quite different. In the expression `$5-3$,' the dash is a \textit{binary} operation (that is, an operation requiring \textit{two} numbers) whereas in `$-3$', the dash is a \textit{unary} operation (that is, an operation requiring only one number).  You might ask, `Who cares?'  Your calculator does - that's who!  In the text we can write $-3 - 3 = -6$ but that will not work in your calculator.  Instead you'd need to type $^{-}3 - 3$ to get $-6$ where the first dash comes from the `$+/-$' key.}

\mnote{.8}{
It might be junior high (elementary?) school material, but arithmetic with fractions is one of the most common sources of errors among university students. If you're not comfortable working with fractions, we strongly recommend seeing your instructor (or a tutor) to go over this material until you're completely confident that you understand it. Experience (and even formal educational \href{http://home.isr.umich.edu/releases/fractions-are-the-key-to-math-success-new-study-shows/}{\underline{studies}}) suggest that your success handling fractions corresponds pretty well with your overall success in passing your Mathematics courses.
}

\medskip

An important point here is that when we `distribute' negatives, we do so across addition or subtraction only.  This is because we are really distributing a factor of $-1$ across each of these terms:  $-(a+b) = (-1)(a+b) = (-1)(a) + (-1)(b) = (-a)+(-b) = -a-b$. Negatives do not `distribute' across multiplication:  $- (2 \cdot 3) \neq (-2)\cdot(-3)$. Instead, $-(2\cdot 3) = (-2)\cdot (3) = (2) \cdot (-3) = -6$.  The same sort of thing goes for fractions:  $- \frac{3}{5}$ can be written as $\frac{-3}{5}$ or $\frac{3}{-5}$, but not $\frac{-3}{-5}$.  It's about time we did a few examples to see how these properties work in practice.

\medskip

\example{fractionreview}{Arithmetic with fractions}{
Perform the indicated operations and simplify. By `simplify' here, we mean to have the final answer written in the form $\frac{a}{b}$ where $a$ and $b$ are integers which have no common factors.  Said another way, we want $\frac{a}{b}$ in `lowest terms'.

\begin{multicols}{3}
\begin{enumerate}

\item $\dfrac{1}{4} + \dfrac{6}{7}$\vphantom{$\dfrac{\dfrac{12}{5} - \dfrac{7}{24}}{1 + \left(\dfrac{12}{5}\right) \left(\dfrac{7}{24}\right)}$}
\item $\dfrac{5}{12} - \left(\dfrac{47}{30} - \dfrac{7}{3}\right)$\vphantom{$\dfrac{\dfrac{12}{5} - \dfrac{7}{24}}{1 + \left(\dfrac{12}{5}\right) \left(\dfrac{7}{24}\right)}$}
\item $\dfrac{\dfrac{12}{5} - \dfrac{7}{24}}{1 + \left(\dfrac{12}{5}\right) \left(\dfrac{7}{24}\right)}$ 

\setcounter{HW}{\value{enumi}}
\end{enumerate}
\end{multicols}


\begin{multicols}{2}
\begin{enumerate}
\setcounter{enumi}{\value{HW}}

\item $\dfrac{(2(2)+1)(-3-(-3)) - 5(4-7)}{4-2(3)}$\vphantom{$\left(\dfrac{3}{5} \right) \left(\dfrac{5}{13} \right) - \left(\dfrac{4}{5}\right) \left( - \dfrac{12}{13}\right)$}
\item $\left(\dfrac{3}{5} \right) \left(\dfrac{5}{13} \right) - \left(\dfrac{4}{5}\right) \left( - \dfrac{12}{13}\right)$

\setcounter{HW}{\value{enumi}}
\end{enumerate}
\end{multicols}
}
{
\begin{enumerate}

\item It may seem silly to start with an example this basic but experience has taught us not to take much for granted.  We start by finding the lowest common denominator and then we rewrite the fractions using that new denominator.  Since $4$ and $7$ are {\bf relatively prime},\index{relatively prime} meaning they have no factors in common, the lowest common denominator is $4 \cdot 7 = 28$.
\begin{align*}
\dfrac{1}{4} + \dfrac{6}{7} & =  \dfrac{1}{4} \cdot \dfrac{7}{7} + \dfrac{6}{7} \cdot \dfrac{4}{4}   \tag*{Equivalent Fractions} \\[5pt]
                            & =  \dfrac{7}{28}  + \dfrac{24}{28}  \tag*{Multiplication of Fractions}\\[5pt]
							& =  \dfrac{31}{28} \tag*{Addition of Fractions}
\end{align*}

The result is in lowest terms because $31$ and $28$ are relatively prime so we're done.

%%%%%%%%%%%%%%%%%%%

\mnote{.65}{We could have used $12 \cdot 30 \cdot 3 = 1080$ as our common denominator but then the numerators would become unnecessarily large.  It's best to use the \emph{lowest} common denominator.}

\item  We could begin with the subtraction in parentheses, namely $\frac{47}{30} - \frac{7}{3}$, and then subtract that result from $\frac{5}{12}$.  It's easier, however, to first distribute the negative across the quantity in parentheses and then use the Associative Property to perform all of the addition and subtraction in one step.  The lowest common denominator for all three fractions is $60$.

%\noindent\hskip-50pt
%\begin{minipage}{1.2\linewidth}

\begin{align*}
\dfrac{5}{12} - \left(\dfrac{47}{30} - \dfrac{7}{3}\right) & = \dfrac{5}{12} - \dfrac{47}{30} + \dfrac{7}{3} \quad \tag*{Distribute the Negative}\\[5pt]
& =  \dfrac{5}{12} \cdot \dfrac{5}{5} - \dfrac{47}{30} \cdot \dfrac{2}{2} + \dfrac{7}{3} \cdot \dfrac{20}{20} \quad \tag*{Equivalent Fractions}\\[5pt]
& =  \dfrac{25}{60} - \dfrac{94}{60} + \dfrac{140}{60} \quad \tag*{Multiplication of Fractions} \\[6pt]
& =  \dfrac{71}{60} \quad \tag*{Addition and Subtraction of Fractions}
\end{align*}
%\end{minipage}

\drawexampleline

The numerator and denominator are relatively prime so the fraction is in lowest terms and we have our final answer.

%%%%%%%%%%%%%%%%%%%%%%%%%%%%%%%


\item What we are asked to simplify in this problem is known as a  `complex' or `compound' fraction.  Simply put, we have fractions within a fraction.  The longest division line (also called a `vinculum') acts as a grouping symbol, quite literally dividing the compound fraction into a numerator (containing fractions) and a denominator (which in this case does not contain fractions):

\[
\dfrac{\dfrac{12}{5} - \dfrac{7}{24}}{1 + \left(\dfrac{12}{5}\right) \left(\dfrac{7}{24}\right)} =  \dfrac{\left(\dfrac{12}{5} - \dfrac{7}{24}\right)}{\left(1 + \left(\dfrac{12}{5}\right) \left(\dfrac{7}{24}\right)\right)} 
\] 

The first step to simplifying a compound fraction like this one is to see if you can simplify the little fractions inside it. There are two ways to proceed. One is to simplify the numerator and denominator separately, and then use the fact that division is the same thing as multiplication by the reciprocal, as follows:

\noindent\vskip-10pt\begin{minipage}{\textwidth}
\begin{flalign*}
 \dfrac{\left(\dfrac{12}{5} - \dfrac{7}{24}\right)}{\left(1 + \left(\dfrac{12}{5}\right) \left(\dfrac{7}{24}\right)\right)} & = \dfrac{\left(\dfrac{12}{5}\cdot \dfrac{24}{24} - \dfrac{7}{24}\cdot \dfrac{5}{5}\right)}{\left(1\cdot \dfrac{120}{120} + \left(\dfrac{12}{5}\right) \left(\dfrac{7}{24}\right)\right)} & & \tag*{Equivalent Fractions}\\[5pt]
& =  \dfrac{288/120 - 35/120}{120/120 + 84/120} & & \tag*{Multiplication of fractions} \\[5pt]
& = \dfrac{253/120}{204/120} & & \tag*{Addition and subtraction of fractions} \\[5pt]
& = \dfrac{253}{\cancel{120}}\cdot \dfrac{\cancel{120}}{204} & & \tag*{Division of fractions and cancellation} \\[5pt]
 & =  \dfrac{253}{204} & &
 \end{flalign*}
\end{minipage}
 
 \medskip
 
Since $253 = 11 \cdot 23$ and $204 = 2 \cdot 2 \cdot 3 \cdot 17$ have no common factors our result is in lowest terms which means we are done.


While there is nothing wrong with the above approach, we can also use our Equivalent Fractions property to rid ourselves of the `compound' nature of this fraction straight away.  The idea is to multiply both the numerator and denominator by the lowest common denominator of each of the `smaller' fractions - in this case, $24 \cdot 5 = 120$.

\drawexampleline

\noindent\vskip-10pt\begin{minipage}{\textwidth}
\begin{flalign*}
 \dfrac{\left(\dfrac{12}{5} - \dfrac{7}{24}\right)}{\left(1 + \left(\dfrac{12}{5}\right) \left(\dfrac{7}{24}\right)\right)} & = \dfrac{\left(\dfrac{12}{5} - \dfrac{7}{24}\right) \cdot 120}{\left(1 + \left(\dfrac{12}{5}\right) \left(\dfrac{7}{24}\right)\right) \cdot 120} & & \tag*{Equivalent Fractions}\\[5pt]
& =  \dfrac{\left(\dfrac{12}{5}\right) (120) - \left(\dfrac{7}{24}\right) (120)}{(1)(120) + \left(\dfrac{12}{5}\right) \left(\dfrac{7}{24}\right)(120)} & & \tag*{Distributive Property} \\[5pt]
& =  \dfrac{\dfrac{12 \cdot 120}{5} - \dfrac{7 \cdot 120}{24}}{120 + \dfrac{12 \cdot 7 \cdot 120}{5 \cdot 24}} & & \tag*{Multiply fractions} \\[5pt]
& =  \dfrac{\dfrac{12 \cdot 24 \cdot \cancel{5}}{\cancel{5}} - \dfrac{7 \cdot 5 \cdot \cancel{24}}{\cancel{24}}}{120 + \dfrac{12 \cdot 7 \cdot \cancel{5} \cdot \cancel{24}}{\cancel{5} \cdot \cancel{24}}} & & \tag*{Factor and cancel} \\[5pt]
 & =  \dfrac{(12 \cdot 24) - (7 \cdot 5)}{120 + (12 \cdot 7)} & &\\[5pt] 
 & =  \dfrac{288 - 35}{120 + 84} =  \dfrac{253}{204},& & 
\end{flalign*}
\end{minipage}

\medskip

which is the same as we obtained above.

%%%%%%%%%%%%%%%%%%%%%%%%%%%%%%%%%%%%%%%%

																					
\item  This fraction may look simpler than the one before it, but the negative signs and parentheses mean that we shouldn't get complacent.  Again we note that the division line here acts as a grouping symbol.  That is, 

\[ 
\dfrac{(2(2)+1)(-3-(-3)) - 5(4-7)}{4-2(3)} = \dfrac{\left((2(2)+1)(-3-(-3)) - 5(4-7) \right)}{(4-2(3))} 
\]

This means that we should simplify the numerator and denominator first, then perform the division last.  We tend to what's in parentheses first, giving multiplication priority over addition and subtraction.
\begin{align*}
\dfrac{(2(2)+1)(-3-(-3)) - 5(4-7)}{4-2(3)} & =  \dfrac{(4+1)(-3+3)-5(-3)}{4 - 6}   \\
	& =  \dfrac{(5)(0) + 15}{-2}   \\ 
	& =  \dfrac{15}{-2}  \\[5pt] 
	& =  -\dfrac{15}{2}  \tag*{Properties of Negatives}
\end{align*}
Since $15 = 3\cdot 5$ and $2$ have no common factors, we are done.
																			

%%%%%%%%%%%%%%%%%%%%%%%%%%%%%%


\item  In this problem, we have multiplication and subtraction.  Multiplication takes precedence so we perform it first.  Recall that to multiply fractions, we do \textit{not} need to obtain common denominators;  rather, we multiply the corresponding numerators together along with the corresponding denominators.  Like the previous example, we have parentheses and negative signs for added fun!
\begin{align*}
\left(\dfrac{3}{5} \right) \left(\dfrac{5}{13} \right) - \left(\dfrac{4}{5}\right) \left( - \dfrac{12}{13}\right) & =  \dfrac{3 \cdot 5}{5 \cdot 13} - \dfrac{4\cdot (-12)}{5 \cdot 13}  \tag*{Multiply fractions}\\[5pt] 
& =  \dfrac{15}{65} - \dfrac{-48}{65}  \\[5pt]
& =  \dfrac{15}{65} + \dfrac{48}{65}  \tag*{Properties of Negatives}\\[5pt]
& =  \dfrac{15+48}{65}   \tag*{Add numerators} \\[5pt] 
& =  \dfrac{63}{65} 
\end{align*}

Since $64 = 3 \cdot 3 \cdot 7$ and $65 = 5 \cdot 13$ have no common factors, our answer $\dfrac{63}{65}$ is in lowest terms and we are done.
\end{enumerate}
} 

\medskip

Of the issues discussed in the previous set of examples none causes students more trouble than simplifying compound fractions.  We presented two different methods for simplifying them:  one in which we simplified the overall numerator and denominator and then performed the division and one in which we removed the compound nature of the fraction at the very beginning.   We encourage the reader to go back and use both methods on each of the compound fractions presented.  Keep in mind that when a compound fraction is encountered in the rest of the text it will usually be simplified using only one method and we may not choose your favourite method.  Feel free to use the other one in your notes.

\smallskip

Next, we review exponents and their properties.  Recall that $2 \cdot 2 \cdot 2$  can be written as $2^3$ because exponential notation expresses repeated multiplication.  In the expression $2^3$, $2$ is called the \textbf{base}\index{base} and $3$ is called the \textbf{exponent}\index{exponent}. In order to generalize exponents from natural numbers to the integers, and eventually to rational and real numbers, it is helpful to think of the exponent as a count of the number of factors of the base we are multiplying by $1$.  For instance, \[2^3 = 1 \cdot (\text{three factors of two}) = 1 \cdot (2 \cdot 2 \cdot 2) = 8.\] From this, it makes sense that \[2^{0} = 1 \cdot (\text{zero factors of two}) = 1.\]  What about $2^{-3}$?  The `$-$' in the exponent indicates that we are `taking away' three factors of two, essentially dividing by three factors of two.  So, \[2^{-3} = 1 \div (\text{three factors of two}) = 1 \div (2 \cdot 2 \cdot 2) = \frac{1}{2 \cdot 2 \cdot 2} = \frac{1}{8}.\]  We summarize the properties of integer exponents below.

\medskip

\definition{propertiesofintegerexponents}{Properties of Integer Exponents}{
Suppose $a$ and $b$ are nonzero real numbers and $n$ and $m$ are integers.

\begin{itemize}

\item  \textbf{Product Rules:} $(ab)^{n} = a^n b^n$ and $a^n a^m = a^{n+m}$.

\item  \textbf{Quotient Rules:} $\left(\dfrac{a}{b}\right)^n = \dfrac{a^n}{b^n}$ and $\dfrac{a^n}{a^m} = a^{n-m}$. 

\item \textbf{Power Rule:}  $\left(a^{n}\right)^{m} = a^{nm}$.

\item  \textbf{Negatives in Exponents:}  $a^{-n} = \dfrac{1}{a^n}$.

 In particular, $\left(\dfrac{a}{b}\right)^{-n} = \left(\dfrac{b}{a}\right)^{n} = \dfrac{b^n}{a^n}$ and $\dfrac{1}{a^{-n}} = a^{n}$.

\item  \textbf{Zero Powers:}  $a^{0} = 1$.

\mnote{.35}{Note:  The expression $0^{0}$ is an indeterminate form. See the comment regarding `$\frac{0}{0}$' on page \pageref{propertiesofzero}.}

\item  \textbf{Powers of Zero:}  For any \textit{natural} number $n$, $0^{n} = 0$.

\textbf{Note:}  The expression $0^{n}$ for integers $n \leq 0$ is not defined.

\end{itemize}
}

While it is important the state the Properties of Exponents, it is also equally important to take a moment to discuss one of the most common errors in Algebra.  It is true that $(ab)^2 = a^2 b^2$ (which some students refer to as `distributing' the exponent to each factor) but you {\bf cannot} do this sort of thing with addition.  That is, in general,   $(a+b)^2 \neq a^2 + b^2$. (For example, take $a= 3$ and $b = 4$.)  The same goes for any other powers.

\smallskip

With exponents now in the mix, we can now state the Order of Operations Agreement.

\medskip

\definition{orderofoperations}{Order of Operations Agreement}{
When evaluating an expression involving real numbers:

\begin{enumerate}

\item  Evaluate any expressions in \textbf{p}arentheses (or other grouping symbols.)
\item  Evaluate \textbf{e}xponents.
\item  Evaluate \textbf{d}ivision and \textbf{m}ultiplication as you read from left to right.
\item  Evaluate \textbf{a}ddition and \textbf{s}ubtraction as you read from left to right.

\end{enumerate}
}

\mnote{.3}{Order of operations follows the  ``PEDMAS'' rule some of you may have encountered.}

\medskip

For example, $2 + 3\cdot 4^2 = 2 + 3\cdot 16 = 2 + 48 = 50$.  Where students get into trouble is with things like $-3^2$.  If we think of this as $0 - 3^2$, then it is clear that we evaluate the exponent first:  $-3^2 =0 -3^2 =0 -9 = -9$.  In general, we interpret $-a^n = -\left(a^n\right)$.  If we want the `negative' to also be raised to a power, we must  write $(-a)^n$ instead.  To summarize, $-3^2 = -9$ but $(-3)^2  = 9$. 

\smallskip

Of course, many of the `properties' we've stated in this section can be viewed as ways to circumvent the order of operations. We've already seen how the distributive property allows us to simplify $5(2+x)$ by performing the indicated multiplication \textbf{before} the addition that's in parentheses.  Similarly, consider trying to evaluate $2^{30172}\cdot 2^{-30169}$.  The Order of Operations Agreement demands that the exponents be dealt with first, however, trying to compute $2^{30172}$ is a challenge, even for a calculator.  One of the Product Rules of Exponents, however, allow us to rewrite this product, essentially performing the multiplication first, to get:  $2^{30172-30169} = 2^{3} = 8$.  

\medskip


\example{exponentreview}{Operations with exponents}{
Perform the indicated operations and simplify.

\begin{multicols}{2}

\begin{enumerate}

\item  $\dfrac{(4-2)(2 \cdot 4)-(4)^2}{(4-2)^2}$

\item $12(-5)(-5+3)^{-4}+6(-5)^2(-4)(-5+3)^{-5}$\vphantom{$\dfrac{(4-2)(2 \cdot 4)-(4)^2}{(4-2)^2}$}

\setcounter{HW}{\value{enumi}}

\end{enumerate}

\end{multicols}

\begin{multicols}{2}

\begin{enumerate}

\setcounter{enumi}{\value{HW}}

\item  $\dfrac{\left(\dfrac{5\cdot 3^{51}}{4^{36}}\right)}{\left(\dfrac{5 \cdot 3^{49}}{4^{34}}\right)}$

\item $\dfrac{2 \left(\dfrac{5}{12}\right)^{-1}}{1 - \left(\dfrac{5}{12}\right)^{-2}}$\vphantom{$\dfrac{\left(\dfrac{5\cdot 3^{51}}{4^{36}}\right)}{\left(\dfrac{5 \cdot 3^{49}}{4^{34}}\right)}$}

\end{enumerate}

\end{multicols}
}
{
\begin{enumerate}

\item  We begin working inside parentheses then deal with the exponents before working through the other operations.  As we saw in Example \ref{fractionreview}, the division here acts as a grouping symbol, so we save the division to the end.
\begin{align*}
\dfrac{(4-2)(2 \cdot 4)-(4)^2}{(4-2)^2} & = & \dfrac{(2)(8)-(4)^2}{(2)^2}  = \dfrac{(2)(8)-16}{4} \\
 & =  \dfrac{16-16}{4} = \dfrac{0}{4} & = & 0 
\end{align*}

\item  As before, we simplify what's in the parentheses first, then work our way through the exponents, multiplication, and finally, the addition.
\begin{align*}
12(-5)(-5+3)^{-4}+6(-5)^2&(-4)(-5+3)^{-5}\\
&  =  12(-5)(-2)^{-4} + 6(-5)^{2}(-4)(-2)^{-5} \\[5pt] 
                                         & =  12(-5)\left(\dfrac{1}{(-2)^4}\right) + 6(-5)^{2}(-4)\left(\dfrac{1}{(-2)^5}\right) \\[5pt]                                         
                                         & =  12(-5)\left(\dfrac{1}{16}\right) + 6(25)(-4)\left(\dfrac{1}{-32}\right) \\[5pt]
& =  (-60)\left(\dfrac{1}{16}\right) + (-600)\left(\dfrac{1}{-32}\right) \\[5pt]
& =  \dfrac{-60}{16} + \left(\dfrac{-600}{-32}\right)  \\[5pt]
& =  \dfrac{-15\cdot \cancel{4}}{4 \cdot \cancel{4}} + \dfrac{-75 \cdot \cancel{8}}{-4 \cdot \cancel{8}}  \\[5pt]
		& =  \dfrac{-15}{4} + \dfrac{-75}{-4}  \\[5pt]
			& =  \dfrac{-15}{4} + \dfrac{75}{4}  \\[5pt]
				& =  \dfrac{-15 + 75}{4}  \\
				& =  \dfrac{60}{4}  \\
	       & =  15  
\end{align*}

\drawexampleline

\item  The Order of Operations Agreement mandates that we work within each set of parentheses first, giving precedence to the exponents, then the multiplication, and, finally the division.  The trouble with this approach is that the exponents are so large that computation becomes a trifle unwieldy.   What we observe, however, is that the bases of the exponential expressions, $3$ and $4$, occur in both the numerator and denominator of the compound fraction, giving us hope that we can use some of the Properties of Exponents (the Quotient Rule, in particular) to help us out. Our first step here is to invert and multiply.  We see immediately that the $5$'s cancel after which we group the powers of $3$ together and the powers of $4$ together and apply the properties of exponents.
\begin{align*}
\dfrac{\left(\dfrac{5\cdot 3^{51}}{4^{36}}\right)}{\left(\dfrac{5 \cdot 3^{49}}{4^{34}}\right)} & =  \dfrac{5\cdot 3^{51}}{4^{36}} \cdot \dfrac{4^{34}}{5 \cdot 3^{49}} & =  \dfrac{\cancel{5} \cdot 3^{51} \cdot 4^{34}}{\cancel{5} \cdot 3^{49} \cdot 4^{36}} & =  \dfrac{3^{51}}{3^{49}} \cdot\dfrac{4^{34}}{4^{36}} \\
& =  3^{51-49} \cdot 4^{34-36} & =  3^{2} \cdot 4^{-2} & =  3^{2} \cdot \left( \dfrac{1}{4^2}\right) \\
& =  9 \cdot \left(\dfrac{1}{16} \right) & =  \dfrac{9}{16} &  
\end{align*}

\item We have yet another instance of a compound fraction so our first order of business is to rid ourselves of the compound nature of the fraction like we did in Example \ref{fractionreview}.  To do this, however, we need to tend to the exponents first so that we can determine what common denominator is needed to simplify the fraction.
\begin{align*}
\dfrac{2 \left(\dfrac{5}{12}\right)^{-1}}{1 - \left(\dfrac{5}{12}\right)^{-2}} & =  \dfrac{2 \left(\dfrac{12}{5}\right)}{1 - \left(\dfrac{12}{5}\right)^{2}} =  \dfrac{\left(\dfrac{24}{5}\right)}{1 - \left(\dfrac{12^2}{5^2}\right)}\\[5pt]
 & =  \dfrac{\left(\dfrac{24}{5}\right)}{1 - \left(\dfrac{144}{25}\right)}  =  \dfrac{\left(\dfrac{24}{5}\right) \cdot 25}{\left(1 - \dfrac{144}{25}\right)\cdot 25}\\[5pt]
  & =  \dfrac{\left(\dfrac{24\cdot 5 \cdot \cancel{5}}{\cancel{5}}\right)}{\left(1 \cdot 25 - \dfrac{144 \cdot \cancel{25}}{\cancel{25}}\right)}  =  \dfrac{120}{25-144} \\[5pt]
& =  \dfrac{120}{-119} = -\dfrac{120}{119}
\end{align*}
Since $120$ and $119$ have no common factors, we are done. 

\end{enumerate}
}

\medskip

We close our review of real number arithmetic with a discussion of roots and radical notation.  Just as subtraction and division were defined in terms of the inverse of addition and multiplication, respectively, we define roots by undoing natural number exponents.

\medskip


\definition{principalnthrootdefn}{The principal $n^{\text{th}}$ root}{
Let $a$ be a real number and let $n$ be a natural number.  If $n$ is odd, then the \index{$n^{\text{th}}$ root ! principal}\index{principal $n^{\text{th}}$ root}\textbf{principal \boldmath $n^{\textbf{th}}$ root} of $a$ (denoted $\sqrt[n]{a}$) is the unique real number satisfying $\left(\sqrt[n]{a}\right)^n = a$.  If $n$ is even, $\sqrt[n]{a}$ is defined similarly provided  $a \geq 0$ and $\sqrt[n]{a} \geq 0$.  The number $n$ is called the \index{root ! index}\index{index of a root}\textbf{index} of the root and the the number $a$ is called the \index{root ! radicand}\index{radicand}\textbf{radicand}.  For $n=2$, we write $\sqrt{a}$ instead of $\sqrt[2]{a}$.
}


\medskip

The reasons for the added stipulations for even-indexed roots in Definition \ref{principalnthrootdefn} can be found in the Properties of Negatives.  First, for all real numbers,  $x^{\text{even power}} \geq 0$, which means it is never negative.  Thus if $a$ is a \textit{negative} real number, there are no real numbers $x$ with $x^{\text{even power}} = a$.  This is why if $n$ is even, $\sqrt[n]{a}$ only exists if $a \geq 0$.  The second restriction for even-indexed roots is that $\sqrt[n]{a} \geq 0$.  This comes from the fact that $x^{\text{even power}} = (-x)^{\text{even power}}$, and we require $\sqrt[n]{a}$ to have just one value.  So even though $2^{4} = 16$ and $(-2)^{4} = 16$, we require $\sqrt[4]{16} = 2$ and ignore $-2$.  

\smallskip

Dealing with odd powers is much easier. For example, $x^3 = -8$ has one and only one real solution, namely $x = -2$, which means not only does $\sqrt[3]{-8}$ exist, there is only one choice, namely $\sqrt[3]{-8} = -2$. Of course, when it comes to solving $x^{5213} = -117$, it's not so clear that there is one and only one real solution, let alone that the solution is $\sqrt[5213]{-117}$. Such pills are easier to swallow once we've thought a bit about such equations graphically, (see Chapter \ref{Polynomials}) and ultimately, these things come from the completeness property of the real numbers mentioned earlier.  

\mnote{.7}{
It's important that you understand the difference between the statements $y=\sqrt{x}$ and $y^2=x$. As we'll discuss in Chapter \ref{RelationsandFunctions}, the equation $y=\sqrt{x}$ defines $y$ as a \textbf{function} of $x$, which means that for each value of $x\geq 0$ there is \text{only one} value of $y$ such that $y=\sqrt{x}$. For example, $y=\sqrt{4}$ is equivalent to $y=2$. On the other hand, there are \textbf{two} solutions to $y^2=x$; namely, $y=\sqrt{x}$ and $y=-\sqrt{x}$. For example, the equation $y^2=4$ is equivalent to the two equations $y=2$ and $y=-2$ (or, more concisely, $y=\pm 2$). Since these two equations are closely related, it's easy to mix them up. The main thing to remember is that $\sqrt{x}$ always denotes the {\em positive} square root of $x$.
}

\smallskip

We list properties of radicals below as a `theorem' since they can be justified using the properties of exponents.

\medskip

\theorem{radicalprops}{Properties of Radicals}{
Let $a$ and $b$ be real numbers and let $m$ and $n$ be natural numbers.  If $\sqrt[n]{a}$ and $\sqrt[n]{b}$ are real numbers, then\index{radical ! properties of}

\begin{itemize}

\item  \textbf{Product Rule:}  $\sqrt[n]{ab} = \sqrt[n]{a} \, \sqrt[n]{b}$ \index{product rule ! for radicals}

\item  \textbf{Quotient Rule:}  $\sqrt[n]{\dfrac{a}{b}} = \dfrac{\sqrt[n]{a}}{\sqrt[n]{b}}$, provided $b \neq 0$. \index{quotient rule ! for radicals}

\item  \textbf{Power Rule:} $\sqrt[n]{a^m} = \left(\sqrt[n]{a}\right)^m$ \index{power rule ! for radicals}

\end{itemize}
}

\medskip

The proof of Theorem \ref{radicalprops} is based on the definition of the principal $n^{\textbf{th}}$ root and the Properties of Exponents.  To establish the product rule, consider the following.  If $n$ is odd, then by definition $\sqrt[n]{ab}$ is the \underline{unique} real number such that $(\sqrt[n]{ab})^{n} = ab$.  Given that $( \sqrt[n]{a} \, \sqrt[n]{b})^n = (\sqrt[n]{a})^n (\sqrt[n]{b})^n = ab$ as well, it must be the case that $\sqrt[n]{ab} = \sqrt[n]{a} \, \sqrt[n]{b}$. If $n$ is even, then $\sqrt[n]{ab}$ is the unique non-negative real number such that $(\sqrt[n]{ab})^{n} = ab$.  Note that since $n$ is even, $\sqrt[n]{a}$ and $\sqrt[n]{b}$ are also non-negative thus $\sqrt[n]{a}\sqrt[n]{b} \geq 0$ as well.  Proceeding as above, we find that $\sqrt[n]{ab} = \sqrt[n]{a} \, \sqrt[n]{b}$.  The quotient rule is proved similarly and is left as an exercise.  The power rule results from repeated application of the product rule, so long as $\sqrt[n]{a}$ is a real number to start with. We leave that as an exercise as well.

\mnote{.3}{Things get more complicated once complex numbers are involved. Fortunately (disappointingly?), that's not a can of worms we'll be opening in this course.} 

\smallskip

We pause here to point out one of the most common errors students make when working with radicals.  Obviously $\sqrt{9} = 3$, $\sqrt{16} = 4$ and $\sqrt{9 + 16} = \sqrt{25} = 5$.  Thus we can clearly see that $5 = \sqrt{25} = \sqrt{9 + 16} \neq \sqrt{9} + \sqrt{16} = 3 + 4 = 7$ because we all know that $5 \neq 7$.  The authors urge you to \textbf{never consider `distributing' roots or exponents}.  It's wrong and no good will come of it because in general $\sqrt[n]{a+b} \neq \sqrt[n]{a} + \sqrt[n]{b}$. 

\phantomsection
\label{donotdistributeexponents}

\smallskip

Since radicals have properties inherited from exponents, they are often written as such.  We define rational exponents in terms of radicals in the box below.

\medskip

\definition{rationalexponentdefn}{Rational exponents}{
Let $a$ be a real number, let $m$ be an integer and let $n$ be a natural number. \index{rational exponent}

\begin{itemize}

\item  $a^{\frac{1}{n}} = \sqrt[n]{a}$ whenever $\sqrt[n]{a}$ is a real number. (If $n$ is even we need $a \geq 0$.)

\item  $a^{\frac{m}{n}}  = \left(\sqrt[n]{a}\right)^m = \sqrt[n]{a^m}$ whenever $\sqrt[n]{a}$ is a real number.

\end{itemize}
}

\medskip

It would make life really nice if the rational exponents defined in Definition \ref{rationalexponentdefn} had all of the same properties that integer exponents have as listed on page \pageref{propertiesofintegerexponents}  - but they don't.  Why not?  Let's look at an example to see what goes wrong.  Consider the Product Rule which says that $(ab)^{n} = a^{n}b^{n}$ and let $a = -16$, $b = -81$ and $n = \frac{1}{4}$.  Plugging the values into the Product Rule yields the equation $((-16)(-81))^{1/4} = (-16)^{1/4}(-81)^{1/4}$.  The left side of this equation is $1296^{1/4}$ which equals $6$ but the right side is undefined because neither root is a real number.  Would it help if, when it comes to even roots (as signified by even denominators in the fractional exponents), we ensure that everything they apply to is non-negative?  That works for some of the rules - we leave it as an exercise to see which ones - but does not work for the Power Rule.

\smallskip
 
Consider the expression $\left(a^{2/3}\right)^{3/2}$.  Applying the usual laws of exponents, we'd be tempted to simplify this as $\left(a^{2/3}\right)^{3/2} = a^{\frac{2}{3} \cdot \frac{3}{2}} = a^{1} = a$.  However, if we substitute $a=-1$ and apply Definition \ref{rationalexponentdefn}, we find $(-1)^{2/3} = \left(\sqrt[3]{-1}\right)^2 = (-1)^2 = 1$ so that $\left((-1)^{2/3}\right)^{3/2} = 1^{3/2} = \left(\sqrt{1}\right)^3 = 1^3 = 1$.  Thus in this case we have $\left(a^{2/3}\right)^{3/2} \neq a$ even though all of the roots were defined.  It is true, however, that $\left(a^{3/2}\right)^{2/3} = a$  and we leave this for the reader to show.  The moral of the story is that when simplifying powers of rational exponents where the base is negative or worse, unknown, it's usually best to rewrite them as radicals.

\medskip

\example{ex_combineop}{Combining operations}{
Perform the indicated operations and simplify. 

  
\begin{enumerate}

\item  $\dfrac{-(-4) - \sqrt{(-4)^2-4(2)(-3)}}{2(2)}$

\item  $\dfrac{2 \left( \dfrac{\sqrt{3}}{3}\right)}{1 - \left( \dfrac{\sqrt{3}}{3} \right)^2}$

\item  $(\sqrt[3]{-2} - \sqrt[3]{-54})^2$

\item  $2 \left(\dfrac{9}{4} - 3\right)^{1/3} + 2\left(\dfrac{9}{4}\right)\left(\dfrac{1}{3}\right)\left(\dfrac{9}{4}-3\right)^{-2/3}$

\end{enumerate}
}
{
\begin{enumerate}

\item  We begin in the numerator and note that the radical here acts a grouping symbol,  so our first order of business is to simplify the radicand. (The line extending horizontally from the square root symbol `$\sqrt{\vphantom{2}}$ is, you guessed it, another vinculum.)

\begin{align*}
\dfrac{-(-4) -\sqrt{(-4)^2-4(2)(-3)}}{2(2)}  & =  \dfrac{-(-4) - \sqrt{16-4(2)(-3)}}{2(2)}  \\[5pt]                                           
 & =  \dfrac{-(-4) - \sqrt{16-4(-6)}}{2(2)} \\[5pt]
 & =  \dfrac{-(-4) - \sqrt{16-(-24)}}{2(2)}  \\[5pt]
 & =  \dfrac{-(-4) - \sqrt{16+24}}{2(2)}  \\[5pt]
 & =  \dfrac{-(-4) - \sqrt{40}}{2(2)} 
\end{align*}
As you may recall, $40$ can be factored using a perfect square as $40 = 4 \cdot 10$ so we use the product rule of radicals to write $\sqrt{40} = \sqrt{4 \cdot 10} = \sqrt{4} \sqrt{10} = 2 \sqrt{10}$.  This lets us factor a `$2$' out of both terms in the numerator, eventually allowing us to cancel it with a factor of $2$ in the denominator.
\begin{align*}
 \dfrac{-(-4) - \sqrt{40}}{2(2)} & =   \dfrac{-(-4) - 2\sqrt{10}}{2(2)}       =   \dfrac{4  - 2\sqrt{10}}{2(2)} \\[5pt] 
   & =   \dfrac{2 \cdot 2  - 2\sqrt{10}}{2(2)}  =   \dfrac{2(2  - \sqrt{10})}{2(2)} \\[5pt] 
	& =   \dfrac{\cancel{2}(2  - \sqrt{10})}{\cancel{2}(2)}  =   \dfrac{2  - \sqrt{10}}{2}
\end{align*}
Since the numerator and denominator have no more common factors, we are done. (Do you see why we aren't `cancelling' the remaining $2$'s?)

\drawexampleline

\item  Once again we have a compound fraction, so we first simplify the exponent in the denominator to see which factor we'll need to multiply by in order to clean up the fraction.
\begin{align*}
\dfrac{2 \left( \dfrac{\sqrt{3}}{3}\right)}{1 - \left( \dfrac{\sqrt{3}}{3} \right)^2} & = \dfrac{2 \left( \dfrac{\sqrt{3}}{3}\right)}{1 - \left( \dfrac{(\sqrt{3})^2}{3^2} \right)}  = \dfrac{2 \left( \dfrac{\sqrt{3}}{3}\right)}{1 - \left( \dfrac{3}{9} \right)}\\[5pt]
& = \dfrac{2 \left( \dfrac{\sqrt{3}}{3}\right)}{1 - \left( \dfrac{1 \cdot \cancel{3}}{3 \cdot \cancel{3}} \right)}  =  \dfrac{2 \left( \dfrac{\sqrt{3}}{3}\right)}{1 - \left( \dfrac{1}{3} \right)} \\[5pt]
& =  \dfrac{2 \left( \dfrac{\sqrt{3}}{3}\right) \cdot 3}{\left(1 - \left( \dfrac{1}{3} \right)\right) \cdot 3}  =  \dfrac{\dfrac{2 \cdot \sqrt{3} \cdot \cancel{3}}{\cancel{3}}}{1\cdot 3 -  \dfrac{1\cdot \cancel{3}}{\cancel{3}}} \\[5pt]
& = \dfrac{2 \sqrt{3}}{3 - 1}  = \dfrac{\cancel{2} \sqrt{3}}{\cancel{2}} = \sqrt{3} 
\end{align*}

\item  Working inside the parentheses, we first encounter $\sqrt[3]{-2}$.  While the $-2$ isn't a perfect cube, (of an integer, that is!) we may think of $-2 = (-1)(2)$.  Since $(-1)^3 = -1$, $-1$ \textit{is} a perfect cube, and we may write $\sqrt[3]{-2} = \sqrt[3]{(-1)(2)} = \sqrt[3]{-1} \sqrt[3]{2} = - \sqrt[3]{2}$. When it comes to $\sqrt[3]{54}$, we may write it as $\sqrt[3]{(-27)(2)} = \sqrt[3]{-27} \sqrt[3]{2} = -3 \sqrt[3]{2}$.  So, \[\sqrt[3]{-2} - \sqrt[3]{-54} = -\sqrt[3]{2} - (-3\sqrt[3]{2}) = -\sqrt[3]{2} + 3 \sqrt[3]{2}.\]  At this stage, we can simplify $-\sqrt[3]{2} + 3 \sqrt[3]{2} = 2 \sqrt[3]{2}$.  You may remember this as being called `combining like radicals,' but it is in fact just another application of the distributive property:  \[-\sqrt[3]{2} + 3\sqrt[3]{2} = (-1)\sqrt[3]{2} + 3 \sqrt[3]{2} = (-1+3)\sqrt[3]{2} = 2\sqrt[3]{2}.\] 
 Putting all this together, we get:
\begin{align*}
  (\sqrt[3]{-2} - \sqrt[3]{-54})^2 & =  (-\sqrt[3]{2} + 3 \sqrt[3]{2})^2   =  (2 \sqrt[3]{2})^2  \\ 
		& =  2^2 (\sqrt[3]{2})^2 = 4 \sqrt[3]{2^2}  =  4 \sqrt[3]{4}
\end{align*}
Since there are no perfect integer cubes which are factors of $4$ (apart from $1$, of course), we are done.

\drawexampleline

\item  We start working in parentheses and get a common denominator to subtract the fractions:
\[ 
\dfrac{9}{4} - 3 = \dfrac{9}{4} - \dfrac{3 \cdot 4}{1 \cdot 4} = \dfrac{9}{4} - \dfrac{12}{4} = \dfrac{-3}{4}  
\] 
Since the denominators in the fractional exponents are odd, we can proceed using the properties of exponents:
\begin{align*}
2 \left(\dfrac{9}{4} - 3\right)^{1/3} + 2\left(\dfrac{9}{4}\right)\left(\dfrac{1}{3}\right)&\left(\dfrac{9}{4}-3\right)^{-2/3}  \\
&= 2 \left(\dfrac{-3}{4} \right)^{1/3} + 2\left(\dfrac{9}{4}\right)\left(\dfrac{1}{3}\right)\left(\dfrac{-3}{4}\right)^{-2/3}  \\ 
& =  2 \left(\dfrac{(-3)^{1/3}}{(4)^{1/3}} \right) + 2\left(\dfrac{9}{4}\right)\left(\dfrac{1}{3}\right)\left(\dfrac{4}{-3}\right)^{2/3}  \\[5pt] 
& =  2 \left(\dfrac{(-3)^{1/3}}{(4)^{1/3}} \right) + 2\left(\dfrac{9}{4}\right)\left(\dfrac{1}{3}\right)\left(\dfrac{(4)^{2/3}}{(-3)^{2/3}}\right)\\[5pt] 
& =  \dfrac{2 \cdot (-3)^{1/3}}{4^{1/3}} + \dfrac{2 \cdot 9 \cdot 1 \cdot 4^{2/3}}{4 \cdot 3 \cdot (-3)^{2/3}}  \\[5pt]
& = \dfrac{2 \cdot (-3)^{1/3}}{4^{1/3}} + \dfrac{\cancel{2} \cdot 3 \cdot \cancel{3} \cdot 4^{2/3}}{2 \cdot \cancel{2} \cdot \cancel{3} \cdot (-3)^{2/3}} \\[5pt] 
& =  \dfrac{2 \cdot (-3)^{1/3}}{4^{1/3}} + \dfrac{3 \cdot 4^{2/3}}{2 \cdot (-3)^{2/3}} 
\end{align*}
At this point, we could start looking for common denominators but it turns out that these fractions reduce even further.  Since $4 = 2^2$, $4^{1/3} = (2^2)^{1/3} = 2^{2/3}$.  Similarly, $4^{2/3} = (2^2)^{2/3} = 2^{4/3}$. The expressions $(-3)^{1/3}$ and $(-3)^{2/3}$ contain negative bases so we proceed with caution and convert them back to radical notation to get:  $(-3)^{1/3} = \sqrt[3]{-3} = -\sqrt[3]{3} = - 3^{1/3}$ and  $(-3)^{2/3} = (\sqrt[3]{-3})^2 = (-\sqrt[3]{3})^2 =(\sqrt[3]{3})^2 = 3^{2/3}$.  Hence:
\begin{align*}
\dfrac{2 \cdot (-3)^{1/3}}{4^{1/3}} + \dfrac{3 \cdot 4^{2/3}}{2 \cdot (-3)^{2/3}} & =  \dfrac{2 \cdot (-3^{1/3})}{2^{2/3}} + \dfrac{3 \cdot 2^{4/3}}{2 \cdot 3^{2/3}}  \\
& =  \dfrac{2^{1} \cdot (-3^{1/3})}{2^{2/3}} + \dfrac{3^{1} \cdot 2^{4/3}}{2^{1} \cdot 3^{2/3}}   \\[5pt]
& =  2^{1 - 2/3} \cdot (-3^{1/3}) +3^{1- 2/3} \cdot 2^{4/3 - 1}   \\
& =  2^{1/3} \cdot (-3^{1/3}) +3^{1/3} \cdot 2^{1/3}   \\
& =   - 2^{1/3} \cdot 3^{1/3} +3^{1/3} \cdot 2^{1/3}   \\
& =  0 
\end{align*}
\end{enumerate}
}

\printexercises{exercises_pre/00_02_exercises}
