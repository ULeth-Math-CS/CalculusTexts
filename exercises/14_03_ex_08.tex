{(Challenging) Take $(y-x)y' = 0$, $y(0) = 0$.  a) Find two distinct solutions.  b)
Explain why this does not violate Picard's theorem. }
{One solution is $y=0$, since this gives $y'=0$ and thus $(y-x)y'=0$. Another solution is $y=x$, since this gives $y(0)=0$ and $(y-x)y' = 0\cdot 1 = 0$. This doesn't contradict Picard's Theorem since the equation cannot be written in the form $y'=f(x,y)$.}