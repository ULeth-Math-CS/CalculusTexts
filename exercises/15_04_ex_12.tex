{Suppose a cup of coffee is at 100 degrees Celsius at time $t=0$,
it is at 70 degrees at $t=10$ minutes, and it is at 50 degrees at $t=20$
minutes.  Compute the ambient temperature.}
{Using $T(t)=A+Ce^{-kt}$, we have $100=A+C$, $70=A+Ce^{-10k}$, and $50 = A+Ce^{-20k}$. The second and third equations can be combined to give $e^{-10k}=\dfrac{50-A}{70-A}$. The first equation gives $C=100-A$. Plugging both of these into the second equation, we can solve for $A$, which gives $A=10$.}