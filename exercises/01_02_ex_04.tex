{$\displaystyle \lim_{x\to 5} (3-x) = -2$}
{Let $\epsilon >0$ be given. We wish to find $\delta >0$ such that when $|x-5|<\delta$, $|f(x)-(-2)|<\epsilon$. 

Consider $|f(x)-(-2)|<\epsilon$:
\begin{gather*}
|f(x) + 2 | < \epsilon \\
|(3-x) + 2 |<\epsilon \\
| 5-x | < \epsilon \\
-\epsilon < 5-x < \epsilon \\
-\epsilon < x-5 < \epsilon. \\
\end{gather*}
This implies we can let $\delta =\epsilon$. Then:
\begin{gather*}
|x-5|<\delta \\
-\delta < x-5 < \delta\\
-\epsilon < x-5 < \epsilon\\
-\epsilon < (x-3)-2 < \epsilon \\
-\epsilon < (-x+3)-(-2) < \epsilon \\
|3-x - (-2)| < \epsilon,
\end{gather*}
which is what we wanted to prove.
}

