\section{Slope fields}\label{sec:slopefields}

{\em Note:} you might find the software {\em DFIELD} and {\em PPLANE} useful. You can download the programs at \href{http://math.rice.edu/~dfield/dfpp.html}{http://math.rice.edu/\string~dfield/dfpp.html}. These used to be available as in-browser Java applets, but due to changes in Java security settings, you need to download the programs and run them locally. Both Java and MATLAB versions are available.

Another option is the IODE software which accompanies the lecture notes by Ji\u{r}\'{i} Lebl from which we've borrowed the text for this chapter.

\medskip

As we said, the general first order equation we are studying looks like
\begin{equation*}
y\primeskip ' = f(x,y).
\end{equation*}
In general, we cannot simply solve these kinds of equations explicitly.
It would be nice if we could at least figure out the shape and behaviour of
the solutions, or if we could find approximate solutions.

\subsection*{Slope fields}

Suppose we are able to solve a first order equation of the form $y\primeskip '=f(x,y)$, obtaining a solution $y=g(x)$. Differential calculus tells us that $y\primeskip '=g'(x)$ gives us the slope of the tangent line to the curve $y=g(x)$ at the point $(x,g(x))$.  Thus, the equation $y\primeskip ' = f(x,y)$
gives you a slope at each point 
in the
$(x,y)$-plane.  We can plot the slope at lots of points
as a short line through the point $(x,y)$ with the slope $f(x,y)$.
See Figure \ref{1.3:fig1}.

\mtable{.75}{Slope field for the equation $y\primeskip '=xy$}{1.3:fig1}{
 \myincludegraphics[width=0.95\marginparwidth]{figures/1-3-xysl}}

\mtable{.5}{Slope field of $y\primeskip ' = xy$ with a graph of solutions satisfying
 $y(0) = 0.2$, $y(0) = 0$, and $y(0) = -0.2$.}{1.3:fig2}{
 \myincludegraphics[width=0.95\marginparwidth]{figures/1-3-xysl-sol}
}

We call this picture the \sword{slope field}\index{slope field} of the equation.
If we are given a specific initial
condition $y(x_0) = y_0$, we can look at the
location $(x_0,y_0)$ and follow the slopes.  See Figure \ref{1.3:fig2}.

By looking at the slope field we can get a lot of information
about the behaviour of
solutions.  For
example, in Figure \ref{1.3:fig2} we can see what the solutions do when the initial conditions
are $y(0) > 0$, $y(0) = 0$ and $y(0) < 0$.  Note that a small change in the
initial condition causes quite different behaviour.
On the other hand, plotting a few solutions of the equation
$y\primeskip ' = -y$,
we see that no matter what $y(0)$ is, all solutions tend to zero as $x$
tends to infinity.
See Figure \ref{1.3:fig3}.

\mtable{.25}{Slope field of $y\primeskip ' = -y$ with a graph of a few solutions.}{1.3:fig3}{
\myincludegraphics[width=0.95\marginparwidth]{figures/1-3-mysl-sol}}

\subsection*{Existence and uniqueness}

We wish to ask two fundamental questions about the problem
\begin{equation*}
y\primeskip ' = f(x,y), \qquad y(x_0) = y_0.
\end{equation*}
\begin{enumerate}[label=(\roman*)]
\item Does a solution \emph{exist}?
\item Is the solution \emph{unique} (if it exists)?
\end{enumerate}

What do you think is the answer?
The answer seems to be yes to both does it not?  Well, pretty much.  But there
are cases when the answer to either question can be no.

Since generally the equations we encounter in applications
come from real life situations, it seems
logical that a solution always exists.
It also has to be unique if we believe our
universe is deterministic.  If the solution does not exist, or if it is
not unique, we have
probably not devised the correct model.  Hence, it is good to know
when things go wrong and why.\\

\example{egk1}{An initial value problem with no solution}{
Attempt to solve:
\begin{equation*}
y\primeskip ' = \frac{1}{x}, \qquad y(0) = 0 .
\end{equation*}}
{
Integrate to find the general solution $y = \ln \, \lvert x \rvert + C$.  Note that the
solution does not exist at $x=0$.  See Figure \ref{1.3:xinvfig}.
}\\


\mtable{.75}{Slope field of $y\primeskip ' = \dfrac{1}{x}$.}{1.3:xinvfig}{
 \myincludegraphics[width=0.95\marginparwidth]{figures/1-3-xinv-sol}
}

\mtable{.5}{Slope field of $y\primeskip ' = 2 \sqrt{\lvert y \rvert}$ with two
 solutions satisfying $y(0) = 0$.}{1.3:sqrtfig}{
 \myincludegraphics[width=0.95\marginparwidth]{figures/1-3-sqrt-sol}}
 
\example{egk2}{An initial value problem without a unique solution}{
Solve:
\begin{equation*}
y\primeskip ' = 2 \sqrt{\lvert y \rvert}, \qquad y(0) = 0 .
\end{equation*}}
{
See Figure \ref{1.3:sqrtfig}.
Note that $y=0$ is a solution.  But another solution is the function
\begin{equation*}
y(x) =
\begin{cases}
x^2 & \text{ if $x \geq 0$,}\\
-x^2 & \text{ if $x < 0$.}
\end{cases}
\end{equation*}
}\\

It is hard to tell by staring at the slope field that the
solution is not
unique.
Is there any hope?
Of course there is.  We have the following theorem,
known as Picard's theorem%


\theorem{slope:picardthm}{Picard's theorem on existence and uniqueness}{%
\index{existence and uniqueness}\index{Picard's theorem}
If $f(x,y)$ is continuous (as a function of two
variables) and $\frac{\partial f}{\partial y}$ exists and is
continuous near some $(x_0,y_0)$, then a solution to
\begin{equation*}
y\primeskip ' = f(x,y), \qquad y(x_0) = y_0,
\end{equation*}
exists (at least for some small interval of $x$'s) and is unique.
}

Note that the problems $y\primeskip ' = \dfrac{1}{x}$, $y(0) = 0$ and 
$y\primeskip ' = 2 \sqrt{\lvert y \rvert}$, $y(0) = 0$ do not satisfy the hypothesis of the
theorem.
Even if we can use the theorem,
we ought to be careful about this existence business.  It is quite
possible that the solution only exists for a short while.\\

\mnote{.3}{Picard's Theorem is named after the French mathematician
\href{http://en.wikipedia.org/wiki/Charles_\%C3\%89mile_Picard}{Charles {\' E}mile Picard}
(1856 -- 1941)}. 

\example{egk4}{An initial value problem with a ``finite time'' solution}{
For some constant $A$, solve:
\begin{equation*}
y\primeskip ' = y^2, \qquad y(0) = A .
\end{equation*}}
{
We know how to solve this equation.  First assume that $A \not= 0$,
so $y$ is not equal to zero at least for some $x$ near 0.  So
$x\primeskip' = \dfrac{1}{y^2}$, so
$x = \dfrac{-1}{y} + C$, so $y = \frac{1}{C-x}$.  If $y(0) = A$, then
$C = \dfrac{1}{A}$ so
\begin{equation*}
y = \frac{1}{\dfrac{1}{A} - x} .
\end{equation*}
If $A=0$, then $y=0$ is a solution.

For example, when $A=1$
the solution ``blows up'' at $x=1$.  Hence, the solution does not exist
for all $x$ even if the equation is nice everywhere.  The equation
$y\primeskip ' = y^2$ certainly
looks nice.
}\\

For most of this
course we will be interested in equations where existence and
uniqueness holds, and in fact holds ``globally'' unlike for the equation
$y\primeskip '=y^2$.
%But it is necessary to understand the examples where things fail for the
%aforementioned reasons.
\printexercises{exercises/15_03_exercises}

