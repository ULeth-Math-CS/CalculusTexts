\section{Limits and Continuity of Multivariable Functions}\label{sec:multi_limit}

We continue with the pattern we have established in this text: after defining a new kind of function, we apply calculus ideas to it. The previous section defined functions of two and three variables; this section investigates what it means for these functions to be ``continuous.''

We begin with a series of definitions. We are used to ``open intervals'' such as $(1,3)$, which represents the set of all $x$ such that $1<x<3$,  and ``closed intervals'' such as $[1,3]$, which represents the set of all $x$ such that $1\leq x\leq 3$. We need analogous definitions for open and closed sets in the $x$-$y$ plane.

\definition{def:open}{\parbox[t]{180pt}{Open Disk, Boundary and Interior Points, Open and Closed Sets, Bounded Sets}}
{An \textbf{open disk} $B$ in $\mathbb{R}^2$ centred at $(x_0,y_0)$ with radius $r$ is the set of all points $(x,y)$ such that $\ds\sqrt{(x-x_0)^2+(y-y_0)^2} < r$. \\

Let $S$ be a set of points in $\mathbb{R}^2$. A point $P$ in $\mathbb{R}^2$ is a \textbf{boundary point} of $S$  if all open disks centred at $P$ contain both points in $S$ and points not in $S$.\\

A point $P$ in $S$ is an \textbf{interior point} of $S$ if there is an open disk centred at $P$ that contains only points in $S$. \\

A set $S$ is \textbf{open} if every point in $S$ is an interior point.\\

A set $S$ is \textbf{closed} if it contains all of its boundary points.\\

A set $S$ is \textbf{bounded} if there is an $M>0$ such that the open disk, centred at the origin with radius $M$, contains $S$. A set that is not bounded is \textbf{unbounded}.
\index{open}\index{closed}\index{open disk}\index{closed disk}\index{boundary point}\index{interior point}\index{bounded set}\index{unbounded set}
}

Figure \ref{fig:multilimit_intro} shows several sets in the $x$-$y$ plane. In each set, point $P_1$ lies on the boundary of the set as all open disks centred there contain both points in, and not in, the set. In contrast, point $P_2$ is an interior point for there is an open disk centred there that lies entirely within the set.
\mtable{.5}{Illustrating open and closed sets in the $x$-$y$ plane.}{fig:multilimit_intro}{%
\begin{tabular}{c}
\myincludegraphics{figures/figmultilimit_introa}\\
(a)\\[10pt]
\myincludegraphics{figures/figmultilimit_introb}\\
(b)\\[10pt]
\myincludegraphics{figures/figmultilimit_introc}\\
(c)\\[10pt]
\end{tabular}
}

The set depicted in Figure \ref{fig:multilimit_intro}(a) is a closed set as it contains all of its boundary points. The set in (b) is open, for all of its points are interior points (or, equivalently, it does not contain any of its boundary points). The set in (c) is neither open nor closed as it contains  some of its boundary points.\\


\example{ex_multilimit1}{Determining open/closed, bounded/unbounded}{
Determine if the domain of the function $f(x,y)=\sqrt{1-x^2/9-y^2/4}$ is open, closed, or neither, and if it is bounded.}
{This domain of this function was found in Example \ref{ex_multi2} to be $D = \{(x,y)\ |\ \frac{x^2}9+\frac{y^2}4\leq 1\}$, the region \textit{bounded} by the ellipse $\frac{x^2}9+\frac{y^2}4=1$. Since the region includes the boundary (indicated by the use of ``$\leq$''), the set contains all of its boundary points and hence is closed. The region is bounded as a disk of radius 4, centred at the origin, contains $D$.
}\\
\pagebreak

\example{ex_multilimit2}{Determining open/closed, bounded/unbounded}{
Determine if the domain of $f(x,y) = \frac1{x-y}$ is open, closed, or neither.}
{As we cannot divide by 0, we find the domain to be $D = \{(x,y)\ |\ x-y\neq 0\}$. In other words, the domain is the set of all points $(x,y)$ \emph{not} on the line $y=x$. 

\mfigure{.75}{Sketching the domain of the function in Example \ref{ex_multilimit2}.}{fig:multilimit2}{figures/figmultilimit2}
The domain is sketched in Figure \ref{fig:multilimit2}. Note how we can draw an open disk around any point in the domain that lies entirely inside the domain, and also note how the only boundary points of the domain are the points on the line $y=x$. We conclude the domain is an open set. The set is unbounded.
}\\

\noindent\textbf{\large Limits}\\

Recall a pseudo--definition of the limit of a function of one variable: ``$\ds \lim_{x\to c}f(x) = L$'' means that if $x$ is ``really close'' to $c$, then $f(x)$ is ``really close'' to $L$. A similar pseudo--definition holds for functions of two variables. We'll say that 
%\enlargethispage{3\baselineskip}

\begin{center}
``$\ds \lim_{(x,y)\to (x_0,y_0)} f(x,y) = L$'' 
\end{center}
means ``if the point $(x,y)$ is really close to the point $(x_0,y_0)$, then $f(x,y)$ is really close to $L$.'' The formal definition is given below.
\mnote{.44}{\textbf{Note:} %As sets in the plane can be more complicated than simple intervals along the real line, Definition \ref{def:multilimita} contain language about the shape of a set $S$ that is different from our first limit definition. %
While our first limit definition was defined over an open interval, we now define limits over a set $S$ in the plane (where $S$ does not have to be open). As planar sets can be far more complicated than intervals, our definition adds the restriction ``$\ldots$ where every open disk centred at $P$ contains points in $S$ other than $P$.'' 
In this text, all sets we'll consider will satisfy this condition and we won't bother to check; it is included in the definition for completeness.}

\definition{def:multilimit}{Limit of a Function of Two Variables}
{Let $S$ be a set containing $P=(x_0,y_0)$ where every open disk centred at $P$ contains points in $S$ other than $P$, let $f$ be a function of two variables defined on $S$, except possibly at $P$, and let $L$ be a real number. 
%Let $f(x,y)$ be a function of two variables and let $(x_0,y_0)$ be a point in the domain of $f$. 
The \sword{limit of $f(x,y)$ as $(x,y)$ approaches $(x_0,y_0)$ is $L$}, denoted 
\[
\ds \lim_{(x,y)\to (x_0,y_0)} f(x,y) = L,
\]
means that given any $\epsilon>0$, there exists $\delta>0$ such that for all  $(x,y)$ in $S$, where $(x,y)\neq (x_0,y_0)$, if $(x,y)$ is in the open disk centred at $(x_0,y_0)$ with radius $\delta$, then $|f(x,y) - L|<\epsilon.$
\index{limit!of multivariable function}\index{multivariable function!limit}
}

\mfigurethree{width=150pt,3Dmenu,activate=onclick,deactivate=onclick,
3Droll=0.12234160136132792,
3Dortho=0.004824123345315456,
3Dc2c=0.9118747115135193 -0.1974218785762787 0.35987377166748047,
3Dcoo=21.82058334350586 66.31769561767578 47.81545639038086,
3Droo=149.99999973566392,
3Dlights=Headlamp,add3Djscript=asylabels.js}{width=150pt}{.2}{\textbf{Illustrating the definition of a limit.} The open disk in the $x$-$y$ plane has radius $\delta$. Let $(x,y)$ be any point in this disk; $f(x,y)$ is within $\epsilon$ of $L$.}{fig:multilimitdef}{figures/figmultilimit_def}
%\mfigure[scale=1.25]{.77}{\textbf{Illustrating the definition of a limit.} The open disk in the $x$-$y$ plane has radius $\delta$. Let $(x,y)$ be any point in this disk; $f(x,y)$ is within $\epsilon$ of $L$.}{fig:multilimitdef}{figures/figmultilimit_def}


The concept behind Definition \ref{def:multilimit} is sketched in Figure \ref{fig:multilimitdef}. Given $\epsilon>0$, find $\delta>0$ such that if $(x,y)$ is any point in the open disk centred at $(x_0,y_0)$ in the $x$-$y$ plane with radius $\delta$, then $f(x,y)$ should be within $\epsilon$ of $L$. 

\pagebreak

Computing limits using this definition is rather cumbersome. The following theorem allows us to evaluate limits much more easily.



%\setboxwidth{0pt}
\theorem{thm:multi_limit_algebra}{\parbox[t]{180pt}{Basic Limit Properties of Functions of Two Variables}}{%\small
Let $b$, $x_0$, $y_0$, $L$ and $K$ be real numbers,  let $n$ be a positive integer, and let $f$ and $g$ be functions with the following limits:
\[
\lim_{(x,y)\to (x_0,y_0)}f(x,y) = L \quad \text{\ and\ } \lim_{(x,y)\to (x_0,y_0)} g(x,y) = K.
\]
The following limits hold.
\index{limit!of multivariable function}\index{limit!properties}\index{multivariable function!limit}
\begin{enumerate}
\item \parbox{80pt}{Constants:} $\displaystyle \lim_{(x,y)\to (x_0,y_0)} b = b$
\item	\parbox{80pt}{Identity }	$\displaystyle \lim_{(x,y)\to (x_0,y_0)} x = x_0$;\qquad $\displaystyle \lim_{(x,y)\to (x_0,y_0)} y = y_0$
\item	\parbox{80pt}{Sums/Differences:} $\displaystyle \lim_{(x,y)\to (x_0,y_0)}\big(f(x,y)\pm g(x,y)\big) = L\pm K$
\item	\parbox{80pt}{Scalar Multiples:}	$\displaystyle \lim_{(x,y)\to (x_0,y_0)} b\cdot f(x,y) = bL$
\item	\parbox{80pt}{Products:}	$\displaystyle \lim_{(x,y)\to (x_0,y_0)} f(x,y)\cdot g(x,y) = LK$
\item	\parbox{80pt}{Quotients:} $\displaystyle \lim_{(x,y)\to (x_0,y_0)} f(x,y)/g(x,y) = L/K$, ($K\neq 0)$
\item	\parbox{80pt}{Powers:} 	$\displaystyle \lim_{(x,y)\to (x_0,y_0)} f(x,y)^n = L^n$
%\item	\parbox{80pt}{Roots:}		\parbox[t]{185pt}{$\displaystyle \lim_{(x,y)\to (x_0,y_0)} \sqrt[n]{f(x,y)} = \sqrt[n]{L}$}% \qquad \small (if $n$ is even then $L$ must be greater than 0; when $n$ is odd, it is true for all $L$.)}

\end{enumerate}
}
\restoreboxwidth
\enlargethispage{\baselineskip}

This theorem, combined with Theorems \ref{I-thm:poly_rat} and \ref{I-thm:lim_continuous} of Section \ref{I-sec:limit_analytically}, allows us to evaluate many limits.\\

\example{ex_multilimit3}{Evaluating a limit}{
Evaluate the following limits:
\[
1. \lim_{(x,y)\to (1,\pi)} \left(\frac yx + \cos(xy)\right) \qquad\qquad 2. \lim_{(x,y)\to (0,0)} \frac{3xy}{x^2+y^2}
\]
}
{\begin{enumerate}
	\item The aforementioned theorems allow us to simply evaluate $y/x+\cos(xy)$ when $x=1$ and $y=\pi$. If an indeterminate form is returned, we must do more work to evaluate the limit; otherwise, the result is the limit. Therefore
	\begin{align*}
	\lim_{(x,y)\to (1,\pi)} \left(\frac yx + \cos(xy)\right)  &= \frac\pi{1}+\cos \pi \\
		&= \pi -1.
	\end{align*}
	\item		We attempt to evaluate the limit by substituting 0 in for $x$ and $y$, but the result is the indeterminate form ``$0/0$.'' To evaluate this limit, we must ``do more work,'' but we have not yet learned what ``kind'' of work to do. Therefore we cannot yet evaluate this limit.
\end{enumerate}
\vskip -1.5\baselineskip
}\pagebreak

When dealing with functions of a single variable we also considered one--sided limits and stated
\[
\lim_{x\to c}f(x) = L \quad\text{ if, and only if,}\quad \lim_{x\to c^+}f(x) =L \quad\textbf{ and}\quad \lim_{x\to c^-}f(x) =L.
\]
That is, the limit is $L$ if and only if $f(x)$ approaches $L$ when $x$ approaches $c$ from \textbf{either} direction, the left or the right.

In the plane, there are infinitely many directions from which $(x,y)$ might approach $(x_0,y_0)$. In fact, we do not have to restrict ourselves to approaching $(x_0,y_0)$ from a particular direction, but rather we can approach that point along a path that is not a straight line. It is possible to arrive at different limiting values by approaching $(x_0,y_0)$ along different paths. If this happens, we say that $\ds \lim_{(x,y)\to(x_0,y_0) } f(x,y)$ does not exist (this is analogous to the left and right hand limits of single variable functions not being equal).

Our theorems tell us that we can evaluate most limits quite simply, without worrying about  paths. When indeterminate forms arise, the limit may or may not exist. If it does exist, it can be difficult to prove this as we need to show the same limiting value is obtained regardless of the path chosen. The case where the limit does not exist is often easier to deal with, for we can often pick two paths along which the limit is different.\\

%it can be difficult to show that the limit exists, for we need to show that the same limiting value is obtained regardless of the path taken.  we can often evaluate the limit along specific paths. If any of these limits differ, we say that \emph{the} limit does not exist.\\

\example{ex_multilimit4}{Showing limits do not exist}{
\begin{enumerate}
	\item Show $\ds \lim_{(x,y)\to (0,0)} \frac{3xy}{x^2+y^2}$ does not exist by finding the limits along the lines $y=mx$.
	\item	Show $\ds \lim_{(x,y)\to (0,0)} \frac{\sin(xy)}{x+y}$ does not exist by finding the limit along the path $y=-\sin x$. 	
\end{enumerate}
}
{\begin{enumerate}
	\item Evaluating $\ds \lim_{(x,y)\to (0,0)} \frac{3xy}{x^2+y^2}$ along the lines $y=mx$ means replace all $y$'s with $mx$ and evaluating the resulting limit:
	\begin{align*}
	\lim_{(x,mx)\to (0,0)} \frac{3x(mx)}{x^2+(mx)^2} &=\lim_{x\to 0} \frac{3mx^2}{x^2(m^2+1)}\\
				&= \lim_{x\to 0} \frac{3m}{m^2+1}\\
				&= \frac{3m}{m^2+1}.
	\end{align*}
	While the limit exists for each choice of $m$, we get a \emph{different} limit for each choice of $m$. That is, along different lines we get differing limiting values, meaning \emph{the} limit does not exist.
	
	\item		Let $f(x,y) = \frac{\sin(xy)}{x+y}$. We are to show that $\ds \lim_{(x,y)\to (0,0)} f(x,y)$ does not exist by finding the limit along the path $y=-\sin x$. First, however, consider the limits found along the lines $y=mx$ as done above.
	\begin{align*}
	\lim_{(x,mx)\to (0,0)} \frac{\sin\big(x(mx)\big)}{x+mx} &= \lim_{x\to 0} \frac{\sin (mx^2)}{x(m+1)} \\
	&= \lim_{x\to 0} \frac{\sin(mx^2)}{x}\cdot\frac1{m+1}.
	\end{align*}
	By applying L'Hospital's Rule, we can show this limit is 0 \emph{except} when $m=-1$, that is, along the line $y=-x$. This line is not in the domain of $f$, so we have found the following fact: along every line $y=mx$ in the domain of $f$, $\ds \lim_{(x,y)\to(0,0)} f(x,y)=0$. %Along this line, $f(x,y)$ is not defined, so it stands to reason that a limit along this line does not exist.
%\drawexampleline
	
	Now consider the limit along the path $y=-\sin x$:
	\begin{align*}
	\lim_{(x,-\sin x)\to (0,0)} \frac{\sin\big(-x\sin x\big)}{x-\sin x} &= \lim_{x\to0} \frac{\sin\big(-x\sin x\big)}{x-\sin x}
	\end{align*}
	Now apply L'Hospital's Rule twice:
	\small
	\begin{align*}
	 \quad &= \lim_{x\to 0}\frac{\cos\big(-x\sin x\big)(-\sin x-x\cos x)}{1-\cos x} \quad \left(\text{``}= 0/0\text{''}\right)\\
	&= \lim_{x\to 0}\frac{-\sin\big(-x\sin x\big)(-\sin x-x\cos x)^2+\cos\big(-x\sin x\big)(-2\cos x+x\sin x)}{\sin x}\\
	&= \text{``$-2/0$''} \Rightarrow \text{the limit does not exist.}
	\end{align*}
	\normalsize
Step back and consider what we have just discovered. Along any line $y=mx$ in the domain of the $f(x,y)$, the limit is 0. However, along the path $y=-\sin x$, which lies in the domain of  $f(x,y)$ for all $x\neq 0$, the limit does not exist. Since the limit is not the same along every path to $(0,0)$, we say $\ds \lim_{(x,y)\to (0,0)}\frac{\sin(xy)}{x+y}$ does not exist.
\end{enumerate}
\vskip -1.5\baselineskip
}\\

\example{ex_multilimit5}{Finding a limit}{
Let $\ds f(x,y) = \frac{5x^2y^2}{x^2+y^2}$. Find $\ds\lim_{(x,y)\to (0,0)}  f(x,y) .$
}
{It is relatively easy to show that along any line $y=mx$, the limit is 0. This is not enough to prove that the limit exists, as demonstrated in the previous example, but it tells us that if the limit does exist then it must be 0.

To prove the limit is 0, we apply Definition \ref{def:multilimit}. Let $\epsilon >0$ be given. We want to find $\delta >0$ such that if $\sqrt{(x-0)^2+(y-0)^2} <\delta$, then $\lvert f(x,y)-0\rvert <\epsilon$.

Set $\delta < \sqrt{\epsilon/5}$. Note that $\ds \left\lvert\frac{5y^2}{x^2+y^2}\right\rvert <5$ for all $(x,y)\neq (0,0)$, and that if $\sqrt{x^2+y^2} <\delta$, then $x^2<\delta^2$.

Let $\sqrt{(x-0)^2+(y-0)^2} = \sqrt{x^2+y^2}<\delta$. Consider $\lvert f(x,y)-0\rvert$:
\begin{align*}
\lvert f(x,y)-0\rvert &= \left\lvert\frac{5x^2y^2}{x^2+y^2}-0\right\rvert \\
				&= \left\lvert x^2\cdot\frac{5y^2}{x^2+y^2}\right\rvert\\
				&< \delta^2\cdot 5 \\
				&< \frac{\epsilon}{5}\cdot 5 \\
				&= \epsilon.
\end{align*}
Thus if $\sqrt{(x-0)^2+(y-0)^2}<\delta$ then $\lvert f(x,y)-0\rvert <\epsilon$, which is what we wanted to show. Thus $\ds \lim_{(x,y)\to(0,0)} \frac{5x^2y^2}{x^2+y^2} = 0$.
}\\
\pagebreak

\noindent\textbf{\large Continuity}\\

Definition \ref{I-def:continuous} defines what it means for a function of one variable to be continuous. In brief, it meant that the graph of the function did not have breaks, holes, jumps, etc. We define continuity for functions of two variables in a similar way as we did for functions of one variable.

\definition{def:multi_continuous}{Continuous}
{Let a function $f(x,y)$ be defined on a set $S$ containing the point $(x_0,y_0)$. 

\begin{enumerate}
	\item $f$ is \textbf{continuous} at $(x_0,y_0)$ if $\ds\lim_{(x,y)\to(x_0,y_0)} f(x,y) = f(x_0,y_0)$.
	\index{continuous function}\index{multivariable function!continuity}
	\item	$f$ is \textbf{continuous on $S$} if $f$ is continuous at all points in $S$. If $f$ is continuous at all points in $\mathbb{R}^2$, we say that $f$ is \textbf{continuous everywhere}.
\end{enumerate}
}

\example{ex_multicont1}{Continuity of a function of two variables}{
Let $\ds f(x,y) = \left\{ \begin{array}{rl} \frac{\cos y\sin x}{x} & x\neq 0 \\
																						\cos y & x=0
													\end{array} \right.$. Is $f$ continuous at $(0,0)$? Is $f$ continuous everywhere?
}
{To determine if $f$ is continuous at $(0,0)$, we need to compare $\ds\lim_{(x,y)\to (0,0)} f(x,y)$ to $f(0,0)$. 

Applying the definition of $f$, we see that $f(0,0) = \cos 0 = 1$. 

We now consider the limit $\ds \lim_{(x,y)\to (0,0)} f(x,y)$. Substituting $0$ for $x$ and $y$ in $(\cos y\sin x)/x$ returns the indeterminate form ``0/0'', so we need to do more work to evaluate this limit.

Consider two related limits: $\ds \lim_{(x,y)\to (0,0)} \cos y$ and $\ds \lim_{(x,y)\to(0,0)} \frac{\sin x}x$. The first limit does not contain $x$, and since $\cos y$ is continuous, 
\[
\ds \lim_{(x,y)\to (0,0)} \cos y =\lim_{y\to 0} \cos y = \cos 0 = 1.
\]
\mfigurethree{width=150pt,3Dmenu,activate=onclick,deactivate=onclick,
3Droll=1.3976649182325884,
3Dortho=0.005226649809628725,
3Dc2c=0.6559564471244812 0.554935097694397 0.5116328597068787,
3Dcoo=-1.2457355260849 0.0923926830291748 4.189182281494141,
3Droo=129.99999868169073,
3Dlights=Headlamp,add3Djscript=asylabels.js}{width=150pt}{.5}{A graph of $f(x,y)$ in Example \ref{ex_multicont1}.}{fig:multicont1}{figures/figmulticont1}
%\mfigure[scale=1.25]{.8}{A graph of $f(x,y)$ in Example \ref{ex_multicont1}.}{fig:multicont1}{figures/figmulticont1}

The second limit does not contain $y$. By Theorem \ref{I-thm:special_limits} we can say
\[
\lim_{(x,y)\to (0,0)} \frac{\sin x}{x} = \lim_{x\to 0} \frac{\sin x}{x} = 1.
\]
Finally, Theorem \ref{thm:multi_limit_algebra} of this section states that we can combine these two limits as follows:
\begin{align*}
\lim_{(x,y)\to (0,0)} \frac{\cos y\sin x}{x} &= \lim_{(x,y)\to (0,0)} (\cos y)\left(\frac{\sin x}{x}\right) \\ 
&=\left(\lim_{(x,y)\to (0,0)} \cos y\right)\left(\lim_{(x,y)\to (0,0)} \frac{\sin x}{x}\right) \\
  &= (1)(1)\\
	&=1.
\end{align*}
%\enlargethispage{2\baselineskip}

We have found that $\ds \lim_{(x,y)\to (0,0)} \frac{\cos y\sin x}{x} = f(0,0)$, so $f$ is continuous at $(0,0)$.
\pagebreak

A similar analysis shows that $f$ is continuous at all points in $\mathbb{R}^2$. As long as $x\neq0$, we can evaluate the limit directly; when $x=0$, a similar analysis shows that the limit is $\cos y$. Thus we can say that $f$ is continuous everywhere. A graph of $f$ is given in Figure \ref{fig:multicont1}. Notice how it has no breaks, jumps, etc.
}\\

The following theorem is very similar to Theorem \ref{I-thm:continuity_algebra}, giving us ways to combine continuous functions to create other continuous functions.

\theorem{thm:multi_continuous_prop}{Properties of Continuous Functions}
{Let $f$ and $g$ be continuous on a set $S$, let $c$ be a real number, and let $n$ be a positive integer. The following functions are continuous on $S$.
\index{continuous function!properties}\index{multivariable function!continuity}
		\begin{enumerate}
		\item		\parbox{80pt}{Sums/Differences:}	$f\pm g$
		\item		\parbox{80pt}{Constant Multiples:}	$c\cdot f$
		\item		\parbox{80pt}{Products:}	$f\cdot g$
		\item		\parbox{80pt}{Quotients:}	$f/g$ \qquad {\small (as longs as $g\neq 0$ on $S$)}
		\item		\parbox{80pt}{Powers:}	$f\,^n$
		\item		\parbox{80pt}{Roots:}	$\sqrt[n]{f}$ \qquad \parbox[t]{150pt}{\small (if $n$ is even then $f\geq 0$ on $S$; if $n$ is odd, then true for all values of $f$ on $S$.)}
		\item		\parbox{80pt}{Compositions:}\parbox[t]{185pt}{Adjust the definitions of $f$ and $g$ to: Let $f$ be continuous on $S$, where the range of $f$ on $S$ is $J$, and let $g$ be a single variable function that is continuous on $J$. Then $g\circ f$, i.e., $g(f(x,y))$, is continuous on $S$.}
		\end{enumerate}
}
\enlargethispage{\baselineskip}

\example{ex_multicont2}{Establishing continuity of a function}{
Let $f(x,y) = \sin (x^2\cos y)$. Show $f$ is continuous everywhere.}
{We will apply both Theorems \ref{I-thm:continuity_algebra} and \ref{thm:multi_continuous_prop}. Let $f_1(x,y) = x^2$. Since $y$ is not actually used in the function, and polynomials are continuous (by Theorem \ref{I-thm:continuity_algebra}), we conclude $f_1$ is continuous everywhere. A similar statement can be made about $f_2(x,y) = \cos y$. Part 3 of Theorem \ref{thm:multi_continuous_prop} states that $f_3=f_1\cdot f_2$ is continuous everywhere, and Part 7 of the theorem states the composition of sine with $f_3$ is continuous: that is, $\sin (f_3) = \sin(x^2\cos y)$ is continuous everywhere.
}\\
\pagebreak

\noindent\textbf{\large Functions of Three Variables}\\

The definitions and theorems given in this section can be extended in a natural way to definitions and theorems about functions of three (or more) variables. We cover the key concepts here; some terms from Definitions \ref{def:open} and \ref{def:multi_continuous} are not redefined but their analogous meanings should be clear to the reader.

\setboxwidth{20pt}
\noindent\hskip-20pt\definition{def:multi3defs}{Open Balls, Limit, Continuous}
{ 
\begin{enumerate}
\item An \textbf{open ball} in $\mathbb{R}^3$ centred at $(x_0,y_0,z_0)$ with radius $r$ is the set of all points $(x,y,z)$ such that $\sqrt{(x-x_0)^2+(y-y_0)^2+(z-z_0)^2} = r$.
\index{multivariable function!limit}\index{limit!of multivariable function}\index{multivariable function!continuity}\index{open ball}
\\

\item Let $D$ be an open set in $\mathbb{R}^3$ containing $(x_0,y_0,z_0)$ where every open ball centred at $(x_0,y_0,z_0)$ contains points of $D$ other than $(x_0,y_0,z_0)$, and let $f(x,y,z)$ be a function of three variables defined on $D$, except possibly at  $(x_0,y_0,z_0)$. The \textbf{limit} of $f(x,y,z)$ as $(x,y,z)$ approaches $(x_0,y_0,z_0)$ is $L$, denoted 
\[
\lim_{(x,y,z)\to (x_0,y_0,z_0)} f(x,y,z) = L,
\]
means that given any $\epsilon >0$, there is a $\delta >0$ such that for all  $(x,y,z)$ in $D$, $(x,y,z)\neq(x_0,y_0,z_0)$, if $(x,y,z)$ is in the open ball centred at $(x_0,y_0,z_0)$ with radius $\delta$, then $\lvert f(x,y,z) - L\rvert < \epsilon$.\\

\item Let $f(x,y,z)$ be defined on a set $D$ containing $(x_0,y_0,z_0)$. $f$ is \textbf{continuous} at $(x_0,y_0,z_0)$ if $\ds \lim_{(x,y,z)\to (x_0,y_0,z_0)} f(x,y,z) = f(x_0,y_0,z_0)$; if $f$ is continuous at all points in $D$, we say $f$ is \sword{continuous on $D$}.
\end{enumerate}
}
\restoreboxwidth

These definitions can also be extended naturally to apply to functions of four or more variables. Theorem \ref{thm:multi_continuous_prop} also applies to function of three or more variables, allowing us to say that the function 
\[
\ds f(x,y,z) = \frac{e^{x^2+y}\sqrt{y^2+z^2+3}}{\sin (xyz)+5}
\]
is continuous everywhere.

When considering single variable functions, we studied limits, then continuity, then the derivative. In our current study of multivariable functions, we have studied limits and continuity. In the next section we study derivation, which takes on a slight twist as we are in a multivarible context.

\printexercises{exercises/12_02_exercises}
