\section{Numerical methods: Euler's method} \label{numer:section}

At this point it may be good to first try the
Lab II\index{IODE software!Lab II} and/or Project II\index{IODE software!Project II} from the
IODE website: \href{http://www.math.uiuc.edu/iode/materials.html}{http://www.math.uiuc.edu/iode/materials.html}. (This is completely optional, and you're free to look for your own software solutions online, or try using Maple or similar software. But it is generally a good idea to have the computer's help when exploring Euler's method.)

\medskip

%You worked with Euler's method in the IODE lab, let us now go over this method
%to see what we've done.

As we said before, unless $f(x,y)$ is of a special form,
it is generally very hard
if not impossible to get a nice formula for the solution of the problem
\begin{equation*}
y' = f(x,y), \qquad y(x_0) = y_0 .
\end{equation*}

What if we want to find the value of the solution at some particular $x$?
Or perhaps we want to produce a graph of the solution to inspect the
behaviour.  In this section we will learn about the basics of numerical
approximation of solutions.

The simplest method for approximating a solution is
\sword{Euler's method}\index{Euler's method}%

\mnote{0.18}{Euler's Method is named after the Swiss mathematician
\href{http://en.wikipedia.org/wiki/Euler}{Leonhard Paul Euler}
(1707 -- 1783).  Do note the correct pronunciation of the name sounds more
like ``oiler.''}.  

It works as follows:
We take $x_0$ and compute the slope $k = f(x_0,y_0)$.  The slope is the
change in $y$ per unit change in $x$.  We follow the line for an interval of
length $h$ on the $x$ axis.  Hence if $y = y_0$ at $x_0$, then we will say that
$y_1$ (the approximate value of $y$ at $x_1 = x_0 + h$) will be
$y_1 = y_0 + h k$.
Rinse, repeat!  That is, compute $x_2$ and $y_2$ using $x_1$ and $y_1$.
For an example of the first two steps of the method see Figure
\ref{euler-step12:fig}.

\mtable{.7}{First two steps of Euler's method with $h=1$ for
the equation $y' = \frac{y^2}{3}$ with initial conditions $y(0)=1$.}{euler-step12:fig}{%
\begin{tabular}{c}
\myincludegraphics[scale=.45]{figures/euler-step1}\\
Step 1\\[12pt]
\myincludegraphics[scale=.45]{figures/euler-step2}\\
Step 2
\end{tabular}
}

More abstractly, for any $i=1,2,3,\ldots$, we compute
\begin{equation*}
x_{i+1} = x_i + h , \qquad y_{i+1}  = y_i + h\, f(x_i,y_i) .
\end{equation*}
The line segments we get are an
approximate graph of the solution.
Generally it is not exactly the solution.  See Figure
\ref{euler-step12-sol:fig} for the plot of the real solution
and the approximation.


\mtable{.35}{Two steps of Euler's method (step size 1) and the exact solution for
the equation $y' = \frac{y^2}{3}$ with initial conditions $y(0)=1$.}{euler-step12-sol:fig}{
\myincludegraphics[scale=.45]{figures/euler-step12-sol}
}

Let us see what happens with the equation $y' = \dfrac{y^2}{3}$, $y(0)=1$.
Let us try
to approximate $y(2)$ using Euler's method.  In
Figures \ref{euler-step12:fig}
and \ref{euler-step12-sol:fig} we have 
graphically approximated $y(2)$ with step size 1.  With step
size 1 we have $y(2) \approx 1.926$.  The real
answer is 3.  So we are approximately 1.074
off.  Let us halve the step size.  Computing $y_4$ with $h=0.5$,
we find that $y(2) \approx 2.209$, so an error of about 0.791.
Table \ref{euler-table:table} gives the values computed
for various parameters.

\medskip

\noindent{\bf Exercise:} Solve this equation exactly and show that $y(2) = 3$.

\medskip

The difference between the actual solution and the approximate solution we
will call the error.  We will usually talk about just the size of the error
and we do not care much about its sign.  The main point is, that we usually
do not know the real solution, so we only have a vague understanding of the
error.  If we knew the error exactly \ldots what is the point of doing the
approximation?

\begin{table}[h!t]
\begin{center}
\begin{tabular}{@{}rrrr@{}}
\hline
$h$ & Approximate $y(2)$ & Error & $\frac{\text{Error}}{\text{Previous
error}}$ \\
\hline
1        & 1.92593 & 1.07407 & \\
0.5      & 2.20861 & 0.79139 & 0.73681 \\
0.25     & 2.47250 & 0.52751 & 0.66656 \\
0.125    & 2.68034 & 0.31966 & 0.60599 \\
0.0625   & 2.82040 & 0.17960 & 0.56184 \\
0.03125  & 2.90412 & 0.09588 & 0.53385 \\
0.015625 & 2.95035 & 0.04965 & 0.51779 \\
0.0078125& 2.97472 & 0.02528 & 0.50913 \\
\hline
\end{tabular}
\end{center}
\caption{Euler's method approximation of $y(2)$ where
of $y'=\dfrac{y^2}{3}$, $y(0)=1$.\label{euler-table:table}}
\end{table}


We notice that except for the first few times, every time we halved 
the interval the error approximately halved.
This halving of the error is a general feature of Euler's method as it is a
\sword{first order method}\index{first order method}.  In the IODE Project II you are
asked to implement a \sword{second order method}\index{second order method}.
A second order method reduces the error to approximately one
quarter every time we halve the interval (second order as $\dfrac{1}{4} =
\dfrac{1}{2} \times \dfrac{1}{2}$).

To get the error to be within 0.1 of the answer we had to already
do 64 steps.  To get it to within 0.01 we would have to halve another three or
four times, meaning doing 512 to 1024 steps.  That is quite a bit to do by
hand.  The improved Euler method from IODE  Project II should quarter the
error every time we
halve the interval, so we would have to approximately do half as many
``halvings'' to get the same error.  This reduction can be a big deal.  With 10
halvings (starting at $h=1$) we have 1024 steps, whereas with 5 halvings
we only have to do 32 steps, assuming that the error was comparable to start
with.  A computer may not care about this
difference for a problem this simple, but suppose each step would take a
second to compute (the function may be substantially more difficult to compute
than $\dfrac{y^2}{3}$).  Then the difference is 32 seconds versus about 17 minutes.
Note: We are not being altogether fair, a second order method would probably
double the time to do each step.  Even so, it is 1 minute versus 17 minutes.
Next, suppose that we have to repeat such a calculation for different
parameters a thousand times.  You get the idea.

Note that in practice we do not know how large the error is!
How do we know what is
the right step size?  Well, essentially we keep halving the interval, and if we
are lucky, we can estimate the error from a few of these calculations and the
assumption that the error goes down by a factor of one half each time (if
we are using standard Euler).

\medskip

\noindent{\bf Exercise:} In the table above, suppose you do not know the error.  Take
the approximate values of the function in the last two lines,
assume that the error goes down by a factor of 2.  Can you estimate the
error in the last time from this?  Does it (approximately) agree with the
table?  Now do it for the first two rows.  Does this agree with the table?

\medskip

Let us talk a little bit more about the example
$y' = \dfrac{y^2}{3}$, $y(0) =
1$.  Suppose that instead of the value $y(2)$ we wish to find $y(3)$.
The results of this effort are listed in Table
\ref{euler-table2:table} for successive halvings of $h$.  What is
going on here?  Well, you should solve the equation exactly and you will
notice that the solution does not exist at $x=3$.  In fact, the solution goes
to infinity when you approach $x=3$.

\mtable{.6}{Attempts to use Euler's to approximate $y(3)$ where
of $y'=\dfrac{y^2}{3}$, $y(0)=1$.}{euler-table2:table}{
\begin{tabular}{@{}rr@{}}
\hline
$h$ & Approximate $y(3)$ \\
\hline
1        & 3.16232 \\
0.5      & 4.54329 \\
0.25     & 6.86079 \\
0.125    & 10.80321 \\
0.0625   & 17.59893 \\
0.03125  & 29.46004 \\
0.015625 & 50.40121 \\
0.0078125& 87.75769 \\
\hline
\end{tabular}
}

Another case where things go bad is if the solution oscillates wildly
near some point.
Such an example is given in IODE Project II.  The solution
may exist at all points, but even a much better numerical method than
Euler would need an insanely small step size to approximate the solution
with reasonable precision.
And computers might not be able to easily handle such a small step size.

\medskip

In real applications we would not use a simple method such as Euler's.  The
simplest method that would probably be used in a real application is the
standard Runge-Kutta method.  That is a
fourth order method,
meaning that if we halve the interval, the error generally
goes down by a factor of 16 (it is fourth order as $\dfrac{1}{16} =
\dfrac{1}{2} \times \dfrac{1}{2}
\times \dfrac{1}{2} \times \dfrac{1}{2}$).

Choosing the right method to use and the right step size can be very tricky.
There are several competing factors to consider.
\begin{itemize}
\item Computational time:  Each step takes computer time.  Even if the
function $f$ is simple to compute, we do it many times over.
Large step size means faster computation, but perhaps not
the right precision.
\item Roundoff errors: Computers only compute with a certain number of
significant digits.  Errors introduced by rounding numbers off during our
computations become noticeable when the step size becomes
too small relative to the quantities we are working with.
So reducing
step size may in fact make errors worse.
There is a certain optimum step size
such that the precision increases as we approach it, but then starts getting
worse as we make our step size
smaller still.  Trouble is: this optimum may be hard to find.
\item Stability: Certain equations may be numerically unstable.  What may
happen is that the numbers never seem to stabilize no matter how many times
we halve the interval.  We may need a ridiculously small interval size,
which may not be practical due to roundoff errors or computational time
considerations.  Such problems are sometimes called
\emph{stiff}\index{stiff problem}.
In the worst case, the numerical computations might be
giving us bogus numbers that look like a correct answer.  Just because the
numbers seem to have stabilized after successive halving, does not mean that we must
have the right answer.
\end{itemize}

We have seen just the beginnings of the challenges that appear in real
applications.  Numerical approximation of solutions to differential
equations is an active research area for engineers and mathematicians.  For
example, the general purpose method used for the ODE solver in Matlab and
Octave (as of this writing) is a method that appeared in the literature only
in the 1980s.

\printexercises{exercises/14_06_exercises}

