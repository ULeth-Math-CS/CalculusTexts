\section{Finding Limits Analytically}\label{sec:limit_analytically}

%\noindent\hskip-65pt\hskip-45pt\parbox{45pt}{\textbf{\itshape Review:}}
%\begin{minipage}[t]{\textwidth+65pt}
%In Section \ref{sec:limit_intro} we explored the concept of the limit without a strict definition. Without a definition, we could only approximate values of the limit. In the previous section we gave the definition of the limit and demonstrated how to use that definition to verify our approximations were correct. Thus far, our method of finding a limit is 1) make a really good approximation either graphically or numerically, and 2) verify our approximation is correct using a $\epsilon$-$\delta$ proof.
%
%This process has its shortcomings, not the least of which is the fact that $\epsilon$--$\delta$ proofs are cumbersome. This section gives a series of theorems which allow us to find limits much more quickly and intuitively. 
%\end{minipage}

In Section \ref{sec:limit_intro} we explored the concept of the limit without a strict definition, meaning we could only make approximations. In the previous section we gave the definition of the limit and demonstrated how to use it to verify our approximations were correct. Thus far, our method of finding a limit is 1) make a really good approximation either graphically or numerically, and 2) verify our approximation is correct using a $\epsilon$-$\delta$ proof.

Recognizing that $\epsilon$-$\delta$ proofs are cumbersome, this section gives a series of theorems which allow us to find limits much more quickly and intuitively. \\
%\vskip \baselineskip

Suppose that $\lim_{x\to 2} f(x)=2$ and $\lim_{x\to 2} g(x) = 3$. What is $\lim_{x\to 2}(f(x)+g(x))$? Intuition tells us that the limit should be 5, as we expect limits to behave in a nice way. The following theorem states that already established limits do behave nicely.

%\enlargethispage{4\baselineskip}

\theorem{thm:limit_algebra}{Basic Limit Properties}{\small
Let $b$, $c$, $L$ and $K$ be real numbers, let $n$ be a positive integer, and let $f$ and $g$ be functions with the following limits: \index{limit!properties}
$$\lim_{x\to c}f(x) = L \text{\ and\ } \lim_{x\to c} g(x) = K.$$
The following limits hold.
\begin{enumerate}
\item \parbox{80pt}{Constants:} $\displaystyle \lim_{x\to c} b = b$
\item	\parbox{80pt}{Identity }						$\displaystyle \lim_{x\to c} x = c$
\item	\parbox{80pt}{Sums/Differences:} $\displaystyle \lim_{x\to c}(f(x)\pm g(x)) = L\pm K$
\item	\parbox{80pt}{Scalar Multiples:}	$\displaystyle \lim_{x\to c} b\cdot f(x) = bL$
\item	\parbox{80pt}{Products:}	$\displaystyle \lim_{x\to c} f(x)\cdot g(x) = LK$
\item	\parbox{80pt}{Quotients:} $\displaystyle \lim_{x\to c} f(x)/g(x) = L/K$, ($K\neq 0)$
\item	\parbox{80pt}{Powers:} 	$\displaystyle \lim_{x\to c} f(x)^n = L^n$
\item	\parbox{80pt}{Roots:}		\parbox[t]{185pt}{$\displaystyle \lim_{x\to c} \sqrt[n]{f(x)} = \sqrt[n]{L}$}% \qquad \small (if $n$ is even then $L$ must be greater than 0; when $n$ is odd, it is true for all $L$.)}
\item	\parbox{80pt}{Compositions:} \parbox[t]{200pt}{Adjust our previously given limit situation to: $$\lim_{x\to c}f(x) = L \text{\ and\ } \lim_{x\to L} g(x) = K.$$ Then $\ds \lim_{x\to c}g(f(x)) = K$.}
\end{enumerate}
}

We make a note about Property \#8: when $n$ is even, $L$ must be greater than 0. If $n$ is odd, then the statement is true for all $L$.

We apply the theorem to an example.\\

\example{ex_basic_limit_1}{Using basic limit properties}{
Let $$\lim_{x\to 2} f(x)=2,\quad\lim_{x\to 2} g(x) = 3\quad \text{\ and \ }\quad p(x) = 3x^2-5x+7.$$ Find the following limits:

\noindent\begin{minipage}[t]{.5\textwidth}
\begin{enumerate}
\item		$\ds \lim_{x\to 2} \big(f(x) + g(x)\big)$
\item		$\ds \lim_{x\to 2} \big(5f(x) + g(x)^2\big)$
\end{enumerate}
\end{minipage}
\begin{minipage}[t]{.5\textwidth}
\begin{enumerate}\addtocounter{enumi}{2}
\item		$\ds \lim_{x\to 2} p(x)$
\end{enumerate}
\end{minipage}}
{\begin{enumerate}
\item		Using the Sum/Difference rule, we know that $\ds \lim_{x\to 2} \big(f(x) + g(x)\big) = 2+3 =5$.
\item		Using the Scalar Multiple and Sum/Difference rules, we find that $\ds \lim_{x\to 2} \big(5f(x) + g(x)^2\big) = 5\cdot 2 + 3^2 = 19.$
\item		Here we combine the Power, Scalar Multiple, Sum/Difference and Constant Rules. We show quite a few steps, but in general these can be omitted:
				\begin{align*}
				\lim_{x\to 2} p(x) &= \lim_{x\to 2} (3x^2-5x+7) \\
				&= \lim_{x\to 2} 3x^2-\lim_{x\to 2} 5x+\lim_{x\to 2}7 \\
				 &= 3\cdot 2^2 - 5\cdot 2+7 \\
				 &= 9
				\end{align*}
\end{enumerate}
\vskip -2\baselineskip
}\\

Part 3 of the previous example demonstrates how the limit of a quadratic polynomial can be determined using the properties of Theorem \ref{thm:limit_algebra}. Not only that, recognize that $$\lim_{x\to 2} p(x) = 9 = p(2);$$ i.e., the limit at 2 was found just by plugging 2 into the function. This holds true for all polynomials, and also for rational functions (which are quotients of polynomials), as stated in the following theorem.

\theorem{thm:poly_rat}{Limits of Polynomial and Rational Functions}{Let $p(x)$ and $q(x)$ be polynomials and $c$ a real number. Then:
\begin{enumerate}
\item	$\ds \lim_{x\to c} p(x) = p(c)$
\item	$\ds \lim_{x\to c} \frac{p(x)}{q(x)} = \frac{p(c)}{q(c)}$, where $q(c) \neq 0$.
\end{enumerate}
}
\enlargethispage{1\baselineskip}

\example{ex_limit_rat}{Finding a limit of a rational function}{
Using Theorem \ref{thm:poly_rat}, find $$\lim_{x\to -1} \frac{3x^2-5x+1}{x^4-x^2+3}.$$}
{Using Theorem \ref{thm:poly_rat}, we can quickly state that 
	\begin{align*} \lim_{x\to -1}\frac{3x^2-5x+1}{x^4-x^2+3} &= \frac{3(-1)^2-5(-1)+1}{(-1)^4-(-1)^2+3} \\
												&= \frac{9}{3} =3.
	\end{align*}
\vskip -2\baselineskip
}\\

It was likely frustrating in Section \ref{sec:limit_def} to do a lot of work to prove that $$\lim_{x\to 2} x^2 = 4$$ as it seemed fairly obvious. The previous theorems state that many functions behave in such an ``obvious'' fashion, as demonstrated by the rational function in Example \ref{ex_limit_rat}. 

Polynomial and rational functions are not the only functions to behave in such a predictable way. The following theorem gives a list of functions whose behaviour is particularly ``nice'' in terms of limits. In the next section, we will give a formal name to these functions that behave ``nicely.''

\enlargethispage{2\baselineskip}
\ifthenelse{\boolean{longpage}}{}
{\setboxwidth{100pt}
}
\noindent\ifthenelse{\isodd{\thepage}}{}{\hskip -100pt}%
\noindent\begin{minipage}{\specialboxlength}
\theorem{thm:lim_continuous}{Special Limits}{%
Let $c$ be a real number in the domain of the given function and let $n$ be a positive integer. The following limits hold: 

\noindent\begin{minipage}[t]{.33\specialboxlength}
\begin{enumerate}
\item		$\ds \lim_{x\to c} \sin x = \sin c$
\item		$\ds \lim_{x\to c} \cos x = \cos c$
\item		$\ds \lim_{x\to c} \tan x = \tan c$
\end{enumerate}
\end{minipage}
\begin{minipage}[t]{.33\specialboxlength}
\begin{enumerate}\addtocounter{enumi}{3}
\item		$\ds \lim_{x\to c} \csc x = \csc c$
\item		$\ds \lim_{x\to c} \sec x = \sec c$
\item		$\ds \lim_{x\to c} \cot x = \cot c$
\end{enumerate}
\end{minipage}
\begin{minipage}[t]{.33\specialboxlength}
\begin{enumerate}\addtocounter{enumi}{6}
\item		$\ds \lim_{x\to c} a^x = a^c$ ($a>0$)
\item		$\ds \lim_{x\to c} \ln x = \ln c$
\item		$\ds \lim_{x\to c} \sqrt[n]{x} = \sqrt[n]{c}$\end{enumerate}
\end{minipage}
}
\end{minipage}
%\end{minipage}
\normalsize
\restoreboxwidth

\example{ex_limit_1}{Evaluating limits analytically}{
Evaluate the following limits. 

\noindent\begin{minipage}[t]{.5\textwidth}
\begin{enumerate}
\item		$\ds \lim_{x\to \pi} \cos x$
\item		$\ds \lim_{x\to 3} (\sec^2x - \tan^2 x)$
\item		$\ds \lim_{x\to \pi/2} \cos x\sin x$
\end{enumerate}
\end{minipage}
\begin{minipage}[t]{.5\textwidth}
\begin{enumerate}\addtocounter{enumi}{3}
\item		$\ds \lim_{x\to 1} e^{\ln x}$
\item		$\ds \lim_{x\to 0} \frac{\sin x}{x}$
\end{enumerate}
\end{minipage}
}
{
\begin{enumerate}
\item		This is a straightforward application of Theorem \ref{thm:lim_continuous}. $\ds \lim_{x\to \pi} \cos x = \cos \pi = -1$.
\item		We can approach this in at least two ways. First, by directly applying Theorem \ref{thm:lim_continuous}, we have:
				$$\lim_{x\to 3} (\sec^2x - \tan^2 x) = \sec^23-\tan^23.$$ Using the Pythagorean Theorem, this last expression is 1; therefore $$\lim_{x\to 3} (\sec^2x - \tan^2 x) = 1.$$
				
				We can also use the Pythagorean Theorem from the start. $$\lim_{x\to 3} (\sec^2x - \tan^2 x) = \lim_{x\to 3} 1 = 1,$$ using the Constant limit rule. Either way, we find the limit is 1.
				
\item		Applying the Product limit rule of Theorem \ref{thm:limit_algebra} and Theorem \ref{thm:lim_continuous} gives $$\ds \lim_{x\to \pi/2} \cos x\sin x = \cos (\pi/2)\sin(\pi/2) = 0\cdot 1 = 0.$$

\item		Again, we can approach this in two ways. First, we can use the exponential/logarithmic identity that $e^{\ln x} = x$ and evaluate $\ds \lim_{x\to 1} e^{\ln x} = \lim_{x\to 1} x = 1.$ 

We can also use the Composition limit rule of Theorem \ref{thm:limit_algebra}. Using Theorem \ref{thm:lim_continuous}, we have $\ds \lim_{x\to 1}\ln x = \ln 1 = 0$. Applying the Composition rule, $$\ds \lim_{x\to 1} e^{\ln x} = \lim_{x\to 0} e^x = e^0 = 1.$$ Both approaches are valid, giving the same result.

\item		We encountered this limit in Section \ref{sec:limit_intro}. Applying our theorems, we attempt to find the limit as $$\lim_{x\to 0}\frac{\sin x}{x}\rightarrow \frac{\sin 0}{0} \rightarrow \raisebox{8pt}{\text{``\ }}\frac{0}{0}\raisebox{8pt}{\text{\ ''}}.$$ This, of course, violates a condition of Theorem \ref{thm:limit_algebra}, as the limit of the denominator is not allowed to be 0. Therefore, we are still unable to evaluate this limit with tools we currently have at hand.
\end{enumerate}
\vskip -1.5\baselineskip
}\\

The section could have been titled ``Using Known Limits to Find Unknown Limits.'' By knowing certain limits of functions, we can find limits involving sums, products, powers, etc., of these functions. We further the development of such comparative tools with the Squeeze Theorem, a clever and intuitive way to find the value of some limits. 

Before stating this theorem formally, suppose we have functions $f$, $g$ and $h$ where $g$ always takes on values between $f$ and $h$; that is, for all $x$ in an interval, $$f(x) \leq g(x) \leq h(x).$$ If $f$ and $h$ have the same limit at $c$, and $g$  is always ``squeezed'' between them, then $g$ must have the same limit as well. That is what the Squeeze Theorem states.

\theorem{thm:sqz}{Squeeze Theorem}
{Let $f$, $g$ and $h$ be functions on an open interval $I$ containing $c$ such that for all $x$ in $I$, $$f(x)\leq g(x) \leq h(x).$$ If $$\lim_{x\to c} f(x) = L = \lim_{x\to c} h(x),$$ then $$\lim_{x\to c} g(x) = L.$$ \index{limit!Squeeze Theorem}\index{Squeeze Theorem}
}

It can take some work to figure out appropriate functions by which to ``squeeze'' the given function of which you are trying to evaluate a limit. However, that is generally the only place work is necessary; the theorem makes the ``evaluating the limit part'' very simple. 

We use the Squeeze Theorem in the following example to finally prove that $\ds \lim_{x\to 0} \frac{\sin x}{x} = 1$.\\

\example{ex_limit_sinx_prove}{Using the Squeeze Theorem}{
Use the Squeeze Theorem to show that $$\ds \lim_{x\to 0} \frac{\sin x}{x} = 1.$$}
{We begin by considering the unit circle. Each point on the unit circle has coordinates $(\cos \theta,\sin \theta)$ for some angle $\theta$ as shown in Figure \ref{fig:squeeze_sinx}. Using similar triangles, we can extend the line from the origin through the point to the point $(1,\tan \theta)$, as shown. (Here we are assuming that $0\leq \theta \leq \pi/2$. Later we will show that we can also consider $\theta \leq 0$.)

\mfigure{.7}{The unit circle and related triangles.}{fig:squeeze_sinx}{figures/figSqueeze1}

Figure \ref{fig:squeeze_sinx} shows three regions have been constructed in the first quadrant, two triangles and a sector of a circle, which are also drawn below. The area of the large triangle is $\frac12\tan\theta$; the area of the sector is $\theta/2$; the area of the triangle contained inside the sector is $\frac12\sin\theta$. It is then clear from the diagram that 

\begin{center}
\begin{tabular}{ccccc}
\myincludegraphics{figures/figSqueeze1a} & & \myincludegraphics{figures/figSqueeze1b} & & \myincludegraphics{figures/figSqueeze1c}\\
$\ds \frac{\tan \theta}{2}$\rule{0pt}{25pt} & $\geq$ & $\ds \frac{\theta}{2}$ & $\geq$ & $\ds \frac{\sin \theta}{2}$
\end{tabular}
\end{center}

%$$\frac{\tan\theta}{2} \geq \frac{\theta}{2} \geq \frac{\sin \theta}{2}.$$

Multiply all terms by $\ds\frac{2}{\sin \theta}$, giving $$\frac{1}{\cos\theta} \geq \frac{\theta}{\sin \theta} \geq 1.$$

Taking reciprocals reverses the inequalities, giving $$ \cos \theta \leq \frac{\sin \theta}{\theta} \leq 1.$$ (These inequalities hold for all values of $\theta$ near 0, even negative values, since $\cos (-\theta) = \cos \theta$ and $\sin (-\theta) = -\sin \theta$.)

Now take limits.

$$\lim_{\theta\to 0} \cos \theta \leq \lim_{\theta\to 0} \frac{\sin\theta}{\theta} \leq \lim_{\theta\to 0}  1 $$
$$\cos 0 \leq \lim_{\theta\to 0} \frac{\sin\theta}{\theta} \leq  1 $$
$$1 \leq \lim_{\theta\to 0} \frac{\sin\theta}{\theta} \leq  1 $$

Clearly this means that $\ds \lim_{\theta\to 0} \frac{\sin\theta}{\theta}=1$.\\
}\\

Two notes about the previous example are worth mentioning. First, one might be discouraged by this application, thinking ``I would \textit{never} have come up with that on my own. This is too hard!'' Don't be discouraged; within this text we will guide you in your use of the Squeeze Theorem. As one gains mathematical maturity, clever proofs like this are easier and easier to create.

Second, this limit tells us more than just that as $x$ approaches 0, $\sin(x)/x$ approaches 1. Both $x$ and $\sin x$ are approaching 0, but the \textit{ratio} of $x$ and $\sin x$ approaches 1, meaning that they are approaching 0 in essentially the same way. Another way of viewing this is: for small $x$, the functions $y=x$ and $y=\sin x$ are essentially indistinguishable.\\

We include this special limit, along with three others, in the following theorem.

\theorem{thm:special_limits}{Special Limits}{%
\noindent\begin{minipage}[t]{.5\specialboxlength}
\begin{enumerate}
	\item		$\ds \lim_{x\to 0} \frac{\sin x}{x} = 1$
	\item		$\ds \lim_{x\to 0} \frac{\cos x-1}{x} = 0$
\end{enumerate}
\end{minipage}
\begin{minipage}[t]{.5\specialboxlength}
\begin{enumerate}\addtocounter{enumi}{2}
	\item		$\ds \lim_{x\to 0} (1+x)^\frac1x = e$
	\item		$\ds \lim_{x\to 0} \frac{e^x-1}{x} = 1$
\end{enumerate}
\end{minipage}
}

A short word on how to interpret the latter three limits. We know that as $x$ goes to 0, $\cos x$ goes to 1. So, in the second limit, both the numerator and denominator are approaching 0. However, since the limit is 0, we can interpret this as saying that ``$\cos x$ is approaching 1 faster than $x$ is approaching 0.''

In the third limit, inside the parentheses we have an expression that is approaching 1 (though never equaling 1), and we know that 1 raised to any power is still 1. At the same time, the power is growing toward infinity. What happens to a number near 1 raised to a very large power? In this particular case, the result approaches Euler's number, $e$, approximately $2.718.$

In the fourth limit, we see that as $x\to 0$, $e^x$ approaches 1 ``just as fast'' as $x\to 0$, resulting in a limit of 1.\\

Our final theorem for this section will be motivated by the following example.\\

\example{ex_limit_onept}{Using algebra to evaluate a limit}{
Evaluate the following limit: $$\lim_{x\to 1}\frac{x^2-1}{x-1}.$$}
{We begin by attempting to apply Theorem \ref{thm:lim_continuous} and substituting 1 for $x$ in the quotient. This gives:
		$$\lim_{x\to 1}\frac{x^2-1}{x-1} = \frac{1^2-1}{1-1} = \raisebox{8pt}{\text{``\ }}\frac{0}{0}\raisebox{8pt}{\text{\ ''}},$$ and indeterminate form. We cannot apply the theorem.

\mfigure{.6}{Graphing $f$ in Example \ref{ex_limit_onept} to understand a limit.}{fig:limitxplus1}{figures/fig_LimitXplus1}
		
		By graphing the function, as in Figure \ref{fig:limitxplus1}, we see that the function seems to be linear, implying that the limit should be easy to evaluate. Recognize that the numerator of our quotient can be factored:
		$$\frac{x^2-1}{x-1} = \frac{(x-1)(x+1)}{x-1}.$$
		The function is not defined when $x=1$, but for all other $x$, $$\frac{x^2-1}{x-1} = \frac{(x-1)(x+1)}{x-1} = \frac{\hbox{\sout{$(x-1)$}}(x+1)}{\hbox{\sout{$x-1$}}}= x+1.$$
		Clearly $\ds \lim_{x\to 1}x+1 = 2$. Recall that when considering limits, we are not concerned with the value of the function at 1, only the value the function approaches as $x$ approaches 1. Since $(x^2-1)/(x-1)$ and $x+1$ are the same at all points except $x=1$, they both approach the same value as $x$ approaches 1. Therefore we can conclude that $$\lim_{x\to 1}\frac{x^2-1}{x-1}=2.$$
\vskip -\baselineskip
}\\

The key to the above example is that the functions $y=(x^2-1)/(x-1)$ and $y=x+1$ are identical except at $x=1$. Since limits describe a value the function is approaching, not the value the function actually attains, the limits of the two functions are always equal.

\theorem{thm:limit_allbut1}{Limits of Functions Equal At All But One Point}{Let $g(x) = f(x)$ for all $x$ in an open interval, except possibly at $c$, and let $\ds \lim_{x\to c} g(x) = L$ for some real number $L$. Then $$\lim_{x\to c}f(x) = L.$$}

The Fundamental Theorem of Algebra tells us that when dealing with a rational function of the form $g(x)/f(x)$ and directly evaluating the limit $\ds \lim_{x\to c} \frac{g(x)}{f(x)}$ returns ``0/0'', % $\ds\raisebox{8pt}{\text{``\ }}\frac{0}{0}\raisebox{8pt}{\text{\ ''}}$, 
then $(x-c)$ is a factor of both $g(x)$ and $f(x)$. One can then use algebra to factor this term out, cancel, then apply Theorem \ref{thm:limit_allbut1}. We demonstrate this once more.\\

\example{ex_limit_allbut1}{Evaluating a limit using Theorem \ref{thm:limit_allbut1}}
{Evaluate $\ds \lim_{x\to 3} \frac{x^3-2 x^2-5 x+6}{2 x^3+3 x^2-32 x+15}$.}
{We begin by applying Theorem \ref{thm:lim_continuous} and substituting 3 for $x$. This returns the familiar indeterminate form of ``0/0''. %\zerooverzero. 
Since the numerator and denominator are each polynomials, we know that $(x-3)$ is factor of each. Using whatever method is most comfortable to you, factor out $(x-3)$ from each (using polynomial division, synthetic division, a computer algebra system, etc.). We find that $$\frac{x^3-2 x^2-5 x+6}{2 x^3+3 x^2-32 x+15} = \frac{(x-3)(x^2+x-2)}{(x-3)(2 x^2+9 x-5)}.$$ We can cancel the $(x-3)$ terms as long as $x\neq 3$. Using Theorem \ref{thm:limit_allbut1} we conclude:
		\begin{align*}
		\lim_{x\to 3} \frac{x^3-2 x^2-5 x+6}{2 x^3+3 x^2-32 x+15} &= \lim_{x\to 3}\frac{(x-3)(x^2+x-2)}{(x-3)(2 x^2+9 x-5)} \\
																															&=	\lim_{x\to 3} \frac{(x^2+x-2)}{(2 x^2+9 x-5)}\\
																															&= \frac{10}{40} = \frac14.
		\end{align*}
\vskip -\baselineskip
}\\
																															


We end this section by revisiting a limit first seen in Section \ref{sec:limit_intro}, a limit of a difference quotient. Let $f(x) = -1.5x^2+11.5x$; we approximated the limit $\ds \lim_{h\to 0}\frac{f(1+h)-f(1)}{h}\approx 8.5.$ We formally evaluate this limit in the following example.\\

\example{ex_limit_diffquot}{Evaluating the limit of a difference quotient}{
Let $f(x) = -1.5x^2+11.5x$; find $\ds \lim_{h\to 0}\frac{f(1+h)-f(1)}{h}.$}
{Since $f$ is a polynomial, our first attempt should be to employ Theorem \ref{thm:lim_continuous} and substitute 0 for $h$. However, we see that this gives us ``$0/0$.'' %\zerooverzero.
 Knowing that we have a rational function hints that some algebra will help. Consider the following steps:
		\begin{align*}
		\lim_{h\to 0}\frac{f(1+h)-f(1)}{h} 	&= 	\lim_{h\to 0}\frac{-1.5(1+h)^2 + 11.5(1+h) - \left(-1.5(1)^2+11.5(1)\right)}{h} \\
																				&=	\lim_{h\to 0}\frac{-1.5(1+2h+h^2) + 11.5+11.5h - 10}{h}\\
																				&=	\lim_{h\to 0}\frac{-1.5h^2 +8.5h}{h}\\
																				&= 	\lim_{h\to 0}\frac{h(-1.5h+8.5)}h\\
																				&=	\lim_{h\to 0}(-1.5h+8.5) \quad (\text{\small using Theorem \ref{thm:limit_allbut1}, as $h\neq 0$}) \\
																				&= 	8.5 \quad (\text{\small using Theorem \ref{thm:lim_continuous}})
		\end{align*}																		

This matches our previous approximation.
}\\

%\noindent\parbox{60pt}{\textbf{\itshape Preview:}}
%\begin{minipage}[t]{\textwidth-63pt}
This section contains several valuable tools for evaluating limits. One of the main results of this section is Theorem \ref{thm:lim_continuous}; it states that many functions that we use regularly behave in a very nice, predictable way. In the next section we give a name to this nice behaviour; we label such functions as \textit{continuous.} Defining that term will require us to look again at what a limit is and what causes limits to not exist.
%\end{minipage}\\

%\clearpage
\printexercises{exercises/01_03_exercises}