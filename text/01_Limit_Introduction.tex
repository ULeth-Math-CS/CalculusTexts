\textit{Calculus} means ``a method of calculation or reasoning.'' When one computes the sales tax on a purchase, one employs a simple calculus. When one finds the area of a polygonal shape by breaking it up into a set of triangles, one is using another calculus. Proving a theorem in geometry employs yet another calculus.

Despite the wonderful advances in mathematics that had taken place into the first half of the $17^\text{th}$ century, mathematicians and scientists were keenly aware of what they \textit{could not do.} (This is true even today.) In particular, two important concepts eluded mastery by the great thinkers of that time: area and rates of change. 

Area seems innocuous enough; areas of circles, rectangles, parallelograms, etc., are standard topics of study for students today just as they were then. However, the areas of \textit{arbitrary} shapes could not be computed, even if the boundary of the shape could be described exactly. 

Rates of change were also important. When an object moves at a constant rate of change, then ``distance = rate $\times $ time.'' But what if the rate is not constant -- can distance still be computed? Or, if distance is known, can we discover the rate of change?

It turns out that these two concepts were related. Two mathematicians, Sir Isaac Newton and Gottfried Leibniz, are credited with independently formulating a system of computing that solved the above problems and showed how they were connected. Their system of reasoning was ``a'' calculus. However, as the power and importance of their discovery took hold, it became known to many as ``the'' calculus. Today, we generally shorten this to discuss ``calculus.''

The foundation of ``the calculus'' is the \textit{limit.} It is a tool to describe a particular behaviour of a function. This chapter begins our study of the limit by approximating its value graphically and numerically. After a formal definition of the limit, properties are established that make ``finding limits'' tractable. Once the limit is understood, then the problems of area and rates of change can be approached.



\section{An Introduction To Limits}\label{sec:limit_intro}

We begin our study of \textit{limits} by considering examples that demonstrate key concepts that will be explained as we progress.\\

Consider the function $y = \dfrac{\sin x}{x}$. When $x$ is near the value 1, what value (if any) is $y$ near?%

While our question is not precisely formed (what constitutes ``near the value 1''?), the answer does not seem difficult to find. One might think first to look at a graph of this function to approximate the appropriate $y$ values. Consider Figure \ref{fig:zoom_sinx_over_x}, where $y = \frac{\sin x}{x}$ is graphed. For values of $x$ near 1, it seems that $y$ takes on values near $0.85$. In fact, when $x=1$, then $y=\frac{\sin 1}{1} \approx 0.84$, so it makes sense that when $x$ is ``near'' 1, $y$ will be ``near'' $0.84$.

\mfigure{.5}{$\sin(x)/x$ near $x=1$.}{fig:zoom_sinx_over_x}{figures/figZoomSinXOverX}
\mfigure{.2}{$\sin(x)/x$ near $x=0$.}{fig:sinx_over_x}{figures/figSinXOverX}
Consider this again at a different value for $x$. When $x$ is near 0, what value (if any) is $y$ near? By considering Figure \ref{fig:sinx_over_x}, one can see that it seems that $y$ takes on values near $1$. But what happens when $x=0$? We have 
\[
 y \rightarrow \frac{\sin 0}{0} \rightarrow \raisebox{8pt}{\text{``\ }}\frac{0}{0}\raisebox{8pt}{\text{\ ''}}.
\] 
The expression ``$0/0$'' has no value; it is \emph{indeterminate.} \index{limit!indeterminate form}\index{indeterminate form} Such an expression gives no information about what is going on with the function nearby. We cannot find out how $y$ behaves near $x=0$ for this function simply by letting $x=0$. 

\emph{Finding a limit} entails understanding how a function behaves near a particular value of $x$. Before continuing, it will be useful to establish some notation. Let $y=f(x)$; that is, let $y$ be a function of $x$ for some function $f$. The expression ``the limit of $y$ as $x$ approaches 1'' describes a number, often referred to as $L$, that $y$ nears as $x$ nears 1. We write all this as 
\[
\lim_{x\to 1} y = \lim_{x\to 1} f(x) = L.
\]
This is not a complete definition; this is a pseudo-definition that will allow us to explore the idea of a limit. \index{limit!pseudo-definition} A more detailed, but still informal, definition of the limit is given in Definition \ref{def:limit_informal} at the end of this section. The precise definition is given in the next section.



Above, where $f(x) = \sin(x)/x$, we approximated 
\[
\lim_{x\to 1} \frac{\sin x}{x} \approx 0.84 \quad \text{ and } \quad \lim_{x\to 0}\frac{\sin x}{x} \approx 1.
\]
(We \textit{approximated} these limits, hence used the ``$\approx$'' symbol, since we are working with the pseudo-definition of a limit, not the actual definition.)

Once we have the true definition of a limit, we will find limits \textit{analytically}; that is, exactly using a variety of mathematical tools. For now, we will \textit{approximate} limits both graphically and numerically. Graphing a function can provide a good approximation, though often not very precise. Numerical methods can provide a more accurate approximation. We have already approximated limits graphically, so we now turn our attention to numerical approximations.


Consider again $\lim_{x\to 1}\sin (x)/x$. To approximate this limit numerically, we can create a table of $x$ and $f(x)$ values where $x$ is ``near'' 1. This is done in Figure \ref{table:sinx_1}.\par

Notice that for values of $x$ near $1$, we have $\sin (x)/x$ near $0.841$. The $x=1$ row is in bold to highlight the fact that when considering limits, we are \textit{not} concerned with the value of the function at that particular $x$ value; we are only concerned with the values of the function when $x$ is \textit{near} 1. 

\mtable{.7}{Approximate values of $\sin(x)/x$ with $x$ near 1.}{table:sinx_1}{\begin{tabular}{cc}
$x$ & $\sin(x)/x$ \\ \hline 
0.9 & 0.870363 \\
 0.99 & 0.844471 \\
 0.999 & 0.841772 \\
 \textbf{1} & \textbf{0.841471} \\
 1.001 & 0.84117 \\
 1.01 & 0.838447 \\
 1.1 & 0.810189
\end{tabular}
%\caption{Values of $\frac{\sin x}{x}$ for $x$ near 1.}\label{fig:sinx_1_table}
}

%\vskip \baselineskip

Now approximate $\lim_{x\to 0} \sin(x)/x$ numerically. We already approximated the value of this limit as 1 graphically in Figure \ref{fig:sinx_over_x}. The table in Figure \ref{table:sinx_2} shows the value of $\sin(x)/x$ for values of $x$ near 0. Ten places after the decimal point are shown to highlight how close to 1 the value of $\sin(x)/x$ gets as $x$ takes on values very near 0. We include the $x=0$ row in bold again to stress that we are not concerned with the value of our function at $x=0$, only on the behaviour of the function \textit{near} 0. 

\mtable{.45}{Approximate values of $\sin(x)/x$ with $x$ near 0.}{table:sinx_2}{\begin{tabular}{cc}
$x$ & $\sin(x)/x$ \\ \hline
 -0.1 & 0.9983341665 \\
 -0.01 & 0.9999833334 \\
 -0.001 & 0.9999998333 \\
 \textbf{0} & \textbf{not defined} \\
 0.001 & 0.9999998333 \\
 0.01 & 0.9999833334 \\
 0.1 & 0.9983341665
 \end{tabular}
% \caption{Values of $\frac{\sin x}{x}$ for $x$ near 0.}\label{fig:sinx_0_table}}
 
This numerical method gives confidence to say that 1 is a good approximation of $\lim_{x\to 0} \sin(x)/x$; that is, 
\[
\lim_{x\to 0} \sin(x)/x \approx 1.
\]
Later we will be able to prove that the limit is \textit{exactly} 1.

We now consider several examples that allow us explore different aspects of the limit concept.\\

\mfigure{.2}{Graphically approximating a limit in Example \ref{ex_limit1}.}{fig:limit1}{figures/figlimit1}


\example{ex_limit1}{Approximating the value of a limit}{
Use graphical and numerical methods to approximate 
\[
\lim_{x\to 3} \frac{x^2-x-6}{6x^2-19x+3}.
\]
}%
{To graphically approximate the limit, graph 
\[
y = (x^2-x-6)/(6x^2-19x+3)
\]
on a small interval that contains 3. To numerically approximate the limit, create a table of values where the $x$ values are near 3. This is done in Figures \ref{fig:limit1} and \ref{table:limit1}, respectively.

\enlargethispage{2\baselineskip}

The graph shows that when $x$ is near 3, the value of $y$ is very near $0.3$. By considering values of $x$ near 3, we see that $y=0.294$ is a better approximation. The graph and the table imply that 
\[
\lim_{x\to 3} \frac{x^2-x-6}{6x^2-19x+3} \approx 0.294.
\] 
\vskip -\baselineskip
}\\
\mtable{.8}{Numerically approximating a limit in Example \ref{ex_limit1}.}{table:limit1}{\begin{tabular}{cc}
$x$ & $\frac{x^2-x-6}{6x^2-19x+3}$ \\ \hline
2.9 & 0.29878 \\
 2.99 & 0.294569 \\
 2.999 & 0.294163 \\
 \textbf{3} & \textbf{not defined}\\
 3.001 & 0.294073 \\
 3.01 & 0.293669 \\
 3.1 & 0.289773
 \end{tabular}
}

This example may bring up a few questions about approximating limits (and the nature of limits themselves). 
\begin{enumerate}
\item		If a graph does not produce as good an approximation as a table, why bother with it?
\item		How many values of $x$ in a table are ``enough?'' In the previous example, could we have just used $x=3.001$ and found a fine approximation?
\end{enumerate}

Graphs are useful since they give a visual understanding concerning the behaviour of a function. Sometimes a function may act ``erratically'' near certain $x$ values which is hard to discern numerically but very plain graphically. Since graphing utilities are very accessible, it makes sense to make proper use of them.


Since tables and graphs are used only to \textit{approximate} the value of a limit, there is not a firm answer to how many data points are ``enough.'' Include enough so that a trend is clear, and use values (when possible) both less than and greater than the value in question. In Example \ref{ex_limit1}, we used both values less than and greater than 3. Had we used just $x=3.001$, we might have been tempted to conclude that the limit had a value of $0.3$. While this is not far off, we could do better. Using values ``on both sides of 3'' helps us identify trends.\\

\example{ex_limit2}{Approximating the value of a limit}{
Graphically and numerically approximate the limit of $f(x)$ as $x$ approaches 0, where 
\[
f(x) = \left\{\begin{array}{rl} x+1 & x< 0 \\ -x^2+1 & x > 0 \end{array}\right..
\]
}{Again we graph $f(x)$ and create a table of its values near $x=0$ to approximate the limit. Note that this is a piecewise defined function, so it behaves differently on either side of 0. Figure \ref{fig:limit2} shows a graph of $f(x)$, and on either side of 0 it seems the $y$ values approach 1. Note that $f(0)$ is not actually defined, as indicated in the graph with the open circle.

\mfigure{.5}{Graphically approximating a limit in Example \ref{ex_limit2}.}{fig:limit2}{figures/figlimit2}
\mtable{.25}{Numerically approximating a limit in Example \ref{ex_limit2}.}{table:limit2}{\begin{tabular}{cc}
$x$ & $f(x)$ \\ \hline
-0.1 & 0.9 \\
 -0.01 & 0.99 \\
 -0.001 & 0.999 \\
 0.001 & 0.999999 \\
 0.01 & 0.9999 \\
 0.1 & 0.99
 \end{tabular}
}

The table shown in Figure \ref{table:limit2} shows values of $f(x)$ for values of $x$ near 0. It is clear that as $x$ takes on values very near 0, $f(x)$ takes on values very near 1. It turns out that if we let $x=0$ for either ``piece'' of $f(x)$, 1 is returned; this is significant and we'll return to this idea later.

The graph and table allow us to say that $\lim_{x\to 0}f(x) \approx 1$; in fact, we are probably very sure it \textit{equals} 1.
}\\

\vskip \baselineskip

\pagebreak

\noindent\textbf{\large Identifying When Limits Do Not Exist}\\

A function may not have a limit for all values of $x$. That is, we cannot say $\lim_{x\to c}f(x)=L$ for some numbers $L$ for all values of $c$, for there may not be a number that $f(x)$ is approaching. There are three ways in which a limit may fail to exist. \index{limit!does not exist}
\begin{enumerate}
\item		The function $f(x)$ may approach different values on either side of $c$.
\item		The function may grow without upper or lower bound as $x$ approaches $c$.
\item		The function may oscillate as $x$ approaches $c$.
\end{enumerate}

We'll explore each of these in turn.\\

\vskip \baselineskip

%\noindent
%

\example{ex_no_limit1}{Different Values Approached From Left and Right}{
Explore why $\ds\lim_{x\to 1} f(x)$ does not exist, where 
\[
f(x) = \left\{\begin{array}{cl} x^2-2x+3 & x\leq 1 \\ x & x>1 \end{array}\right.
\]}%
{
A graph of $f(x)$ around $x=1$ and a table are given Figures \ref{fig:nolimit1} and \ref{table:nolimit1}, respectively. It is clear that as $x$ approaches 1, $f(x)$ does not seem to approach a single number. Instead, it seems as though $f(x)$ approaches two different numbers. When considering values of $x$ less than 1 (approaching 1 from the left), it seems that $f(x)$ is approaching 2; when considering values of $x$ greater than 1 (approaching 1 from the right), it seems that $f(x)$ is approaching 1. Recognizing this behaviour is important; we'll study this in greater depth later. Right now, it suffices to say that the limit does not exist since $f(x)$ is not approaching one value as $x$ approaches 1.
\mfigure{.8}{Observing no limit as $x\to 1$ in Example \ref{ex_no_limit1}.}{fig:nolimit1}{figures/fignolimit1}
\mtable{.6}{Values of $f(x)$ near $x=1$ in Example \ref{ex_no_limit1}.}{table:nolimit1}{\begin{tabular}{cc}
$x$ & $f(x)$ \\ \hline
 0.9 & 2.01 \\
 0.99 & 2.0001 \\
 0.999 & 2.000001 \\
 1.001 & 1.001 \\
 1.01 & 1.01 \\
 1.1 & 1.1
\end{tabular}
}
}\\

%
%\vskip \baselineskip
%\noindent\textbf{The Function Grows Without Bound}\\

%\begin{tikzpicture}
\begin{axis}[width=\marginparwidth+25pt,tick label style={font=\scriptsize},minor x tick num=1,axis y line=middle,axis x line=middle,ymin=-1,ymax=110,xmin=-.1,xmax=2.1,name=myplot]

\addplot [{\colorone},smooth,thick] coordinates {(0.,1.) (0.05,1.10803) (0.1,1.23457) (0.15,1.38408) (0.2,1.5625)(0.25,1.77778) (0.3,2.04082) (0.35,2.36686) (0.4,2.77778)(0.45,3.30579) (0.5,4.) (0.55,4.93827) (0.6,6.25) (0.65,8.16327)(0.7,11.1111) (0.75,16.) (0.8,25.) (0.85,44.4444) (0.9,100.) };
\addplot [{\colorone},smooth,thick] coordinates {(1.1,100.) (1.15,44.4444) (1.2,25.) (1.25,16.) (1.3,11.1111) (1.35,8.16327) (1.4,6.25) (1.45,4.93827) (1.5,4.) (1.55,3.30579) (1.6,2.77778) (1.65,2.36686) (1.7,2.04082) (1.75,1.77778) (1.8,1.5625) (1.85,1.38408) (1.9,1.23457) (1.95,1.10803) (2.,1.)};
%\addplot [{\colorone},smooth] coordinates {(1,1) (1.1,1.1) (1.2,1.2) (1.3,1.3) (1.4,1.4) (1.5,1.5) (1.6,1.6) (1.7,1.7) (1.8,1.8) (1.9,1.9) (2.,2.)
\draw [dashed,thick] (axis cs: 1,1) -- (axis cs: 1,100);
\end{axis}
\node [right] at (myplot.right of origin) {\scriptsize $x$};
\node [above] at (myplot.above origin) {\scriptsize $y$};
\end{tikzpicture}
%\caption{of $f(x)$ in Example \ref{ex_no_limit2}.}\label{fig:nolimit2}
\example{ex_no_limit2}{The Function Grows Without Bound}{
Explore why $\ds\lim_{x\to 1} 1/(x-1)^2$ does not exist.}%
{A graph and table of $f(x) = 1/(x-1)^2$ are given in Figures \ref{fig:nolimit2} and \ref{table:nolimit2}, respectively. Both show that as $x$ approaches 1, $f(x)$ grows larger and larger. 
\mfigure{.4}{Observing no limit as $x\to 1$ in Example \ref{ex_no_limit2}.}{fig:nolimit2}{figures/fignolimit2}
\mtable{.2}{Values of $f(x)$ near $x=1$ in Example \ref{ex_no_limit2}.}{table:nolimit2}{\begin{tabular}{cc}
$x$ & $f(x)$ \\ \hline
 0.9 & 100. \\
 0.99 & 10000. \\
 0.999 & $1.\times 10^6$ \\
 1.001 & $1.\times 10^6$ \\
 1.01 & 10000. \\
 1.1 & 100.
\end{tabular}}

We can deduce this on our own, without the aid of the graph and table. If $x$ is near 1, then $(x-1)^2$ is very small, and: 
\[
\frac{1}{\text{very small number}} = \text{very large number}.
\]
Since $f(x)$ is not approaching a single number, we conclude that 
\[
\lim_{x\to 1}\frac{1}{(x-1)^2}
\]
does not exist.
}\\

%\vskip \baselineskip
%\noindent\textbf{The Function Oscillates}\\

\example{ex_no_limit3}{The Function Oscillates}{
Explore why $\ds\lim_{x\to 0}\sin(1/x)$ does not exist.}%
{%\mfigure{.4}{Observing no limit as $x\to 0$ in Example \ref{ex_no_limit3}.}{fig:nolimit3a}{figures/figNoLimit3a}
%\mfigure{.2}{Zooming in to observing no limit as $x\to 0$ in Example \ref{ex_no_limit3}.}{fig:nolimit3b}{figures/figNoLimit3b}
Two graphs of $f(x) = \sin(1/x)$ are given in Figures \ref{fig:nolimit3}. Figure \ref{fig:nolimit3}(a) shows $f(x)$ on the interval $[-1,1]$; notice how $f(x)$ seems to oscillate near $x=0$. One might think that despite the oscillation, as $x$ approaches 0, $f(x)$ approaches 0. However, Figure \ref{fig:nolimit3}(b) zooms in on $\sin(1/x)$, on the interval $[-0.1,0.1]$. Here the oscillation is even more pronounced. Finally, in the table in Figure \ref{fig:nolimit3}(c), we see $\sin(x)/x$ evaluated for values of $x$ near 0. As $x$ approaches 0, $f(x)$ does not appear to approach any value. 

It can be shown that in reality, as $x$ approaches 0, $\sin(1/x)$ takes on all values between $-1$ and 1 infinitely many times! Because of this oscillation,

 $\ds\lim_{x\to 0}\sin(1/x)$ does not exist.}\\

\ifthenelse{\boolean{longpage}}%%% if longpage, squeeze it in
{\vskip\baselineskip
\noindent\begin{minipage}{\textwidth}\centering
\begin{tabular}{cc}
(a) \myincludegraphics[scale=.9]{figures/figNoLimit3a} & (b) \myincludegraphics[scale=.9]{figures/figNoLimit3b}\end{tabular}
\vskip \baselineskip
\begin{tabular}{c}
(c)\begin{tabular}[b]{cc}
 $x$ & $\sin(1/x)$ \\ \hline 0.1 & $-0.544021$ \\ 0.01 & $-0.506366$ \\ 0.001 & 0.82688 \\ 0.0001 & $-0.305614$ \\ $1.\times 10^{-5}$ & 0.0357488 \\
 $1.\times 10^{-6}$ & $-0.349994$ \\ $1.\times 10^{-7}$ & 0.420548 \\ \\\end{tabular}\end{tabular}%
\captionsetup{type=figure}%
\caption{Observing that $f(x) = \sin(1/x)$ has no limit as $x\to 0$ in Example \ref{ex_no_limit3}.}\label{fig:nolimit3}
\end{minipage}} %% else, not longpage

%\vskip 1\baselineskip
\ifthenelse{\isodd{\thepage}}{}{\noindent\hskip -\marginparwidth }
\noindent\begin{minipage}{\textwidth+\marginparwidth+\marginparsep}%\centering
\begin{tabular}{ccc}
\myincludegraphics{figures/figNoLimit3a} &  \myincludegraphics{figures/figNoLimit3b} &  \begin{tabular}[b]{cc}
 $x$ & $\sin(1/x)$ \\ \hline 0.1 & $-0.544021$ \\ 0.01 & $-0.506366$ \\ 0.001 & 0.82688 \\ 0.0001 & $-0.305614$ \\ $1.\times 10^{-5}$ & 0.0357488 \\
 $1.\times 10^{-6}$ & $-0.349994$ \\ $1.\times 10^{-7}$ & 0.420548 \\ \\\end{tabular}\\
(a) & (b) & (c)\end{tabular}%
\captionsetup{type=figure}%
\caption{Observing that $f(x) = \sin(1/x)$ has no limit as $x\to 0$ in Example \ref{ex_no_limit3}.}\label{fig:nolimit3}
\end{minipage}
}
\vskip 2\baselineskip

%\vskip \baselineskip
\noindent\textbf{\large Limits of Difference Quotients}\\

We have approximated limits of functions as $x$ approached a particular number. We will consider another important kind of limit after explaining a few key ideas.\index{limit!difference quotient}

\mfigure{.35}{Interpreting a difference quotient as the slope of a secant line.}{fig:diffquot1}{figures/figDiffQuot1}

Let $f(x)$ represent the position function, in feet, of some particle that is moving in a straight line, where $x$ is measured in seconds. Let's say that when $x=1$, the particle is at position 10 ft., and when $x=5$, the particle is at 20 ft. Another way of expressing this is to say 
\[
f(1)=10 \quad \text{ and } \quad f(5) = 20.
\]
Since the particle travelled 10 feet in 4 seconds, we can say the particle's \textit{average velocity} was 2.5 ft/s. We write this calculation using a ``quotient of differences,'' or, a \textit{difference quotient}: 
\[
\frac{f(5) - f(1)}{5-1} = \frac{10}4 = 2.5 \text{ft/s}.
\]

This difference quotient can be thought of as the familiar ``rise over run'' used to compute the slopes of lines. In fact, that is essentially what we are doing: given two points on the graph of $f$, we are finding the slope of the \textit{secant line} through those two points. See Figure \ref{fig:diffquot1}.

Now consider finding the average speed on another time interval. We again start at $x=1$, but consider the position of the particle $h$ seconds later. That is, consider the positions of the particle when $x=1$ and when $x=1+h$. The difference quotient is now 
\[
\frac{f(1+h)-f(1)}{(1+h)-1} = \frac{f(1+h)-f(1)}h.
\]

Let $f(x) = -1.5x^2+11.5x$; note that $f(1)=10$ and $f(5) = 20$, as in our discussion. We can compute this difference quotient for all values of $h$ (even negative values!) except $h=0$, for then we get ``0/0,'' the indeterminate form introduced earlier. For all values $h\neq 0$, the difference quotient computes the average velocity of the particle over an interval of time of length $h$ starting at $x=1$. 

For small values of $h$, i.e., values of $h$ close to 0, we get average velocities over very short time periods and compute secant lines over small intervals. See Figure \ref{fig:diff_quot_small_h}. This leads us to wonder what the limit of the difference quotient is as $h$ approaches 0. That is, 
\[
\lim_{h\to 0} \frac{f(1+h)-f(1)}{h} = \text{ ? }
\]

\vskip \baselineskip
\ifthenelse{\boolean{longpage}}% in longpage form
			{% is longpage
			\noindent\begin{minipage}{\textwidth}\centering
			\begin{tabular}{cc}
			(a) \myincludegraphics{figures/figDiffQuotSmallha} & (b) \myincludegraphics{figures/figDiffQuotSmallhb}%
			\end{tabular}
			\begin{tabular}{c} (c)\myincludegraphics{figures/figDiffQuotSmallhc}\end{tabular}
			\captionsetup{type=figure}%
			\caption{Secant lines of $f(x)$ at $x=1$ and $x=1+h$, for shrinking values of $h$ (i.e., $h\rightarrow 0$).}\label{fig:diff_quot_small_h}
			\end{minipage}
			\vskip 2\baselineskip
			}% end longpage
			{% isn't longpage
%			\ifthenelse{\isodd{\thepage}}{}{\noindent\hskip -\marginparwidth \hskip -\marginparsep}
%			\noindent\begin{minipage}{\textwidth+\marginparwidth+\marginparsep}%\centering
			\mtable{.61}{Secant lines of $f(x)$ at $x=1$ and $x=1+h$, for shrinking values of $h$ (i.e., $h\rightarrow 0$).}{fig:diff_quot_small_h}{\begin{tabular}{c}
			\myincludegraphics{figures/figDiffQuotSmallha}\\ (a)\\ \myincludegraphics{figures/figDiffQuotSmallhb} 
			\\ (b)\\ \myincludegraphics{figures/figDiffQuotSmallhc}\\(c)\end{tabular}%
			}
%			\captionsetup{type=figure}%
%			\caption{Secant lines of $f(x)$ at $x=1$ and $x=1+h$, for shrinking values of $h$ (i.e., $h\rightarrow 0$).}\label{fig:diff_quot_small_h}
%			\end{minipage}
%			\vskip 2\baselineskip
			}% ends isn't a longpage

As we do not yet have a true definition of a limit nor an exact method for computing it, we settle for approximating the value. While we could graph the difference quotient (where the $x$-axis would represent $h$ values and the $y$-axis would represent values of the difference quotient) we settle for making a table. See Figure \ref{table:diff_quot_smallh}. The table gives us reason to assume the value of the limit is about 8.5. \\

%\ifthenelse{\boolean{longpage}}{\mtable{.3}{The difference quotient evaluated at values of $h$ near 0.}{table:diff_quot_smallh}{\begin{tabular}{cc}$h$ & $\frac{f(1+h)-f(1)}{h}$\vspace{1pt} \\ \hline $-0.5$ & 9.25 \\ $-0.1$ & 8.65 \\ $-0.01$ & 8.515 \\ 0.01 & 8.485 \\ 0.1 & 8.35 \\ 0.5 & 7.75 \end{tabular}} }
%{}

%\vskip \baselineskip
\enlargethispage{\baselineskip}

Proper understanding of limits is key to understanding calculus. With limits, we can accomplish seemingly impossible mathematical things, like adding up an infinite number of numbers (and not get infinity) and finding the slope of a line between two points, where the ``two points'' are actually the same point. These are not just mathematical curiosities; they allow us to link position, velocity and acceleration together, connect cross-sectional areas to volume, find the work done by a variable force, and much more.

\ifthenelse{\boolean{longpage}}{}{\mtable{.2}{The difference quotient evaluated at values of $h$ near 0.}{table:diff_quot_smallh}{\begin{tabular}{cc}$h$ & $\frac{f(1+h)-f(1)}{h}$\vspace{1pt} \\ \hline $-0.5$ & 9.25 \\ $-0.1$ & 8.65 \\ $-0.01$ & 8.515 \\ 0.01 & 8.485 \\ 0.1 & 8.35 \\ 0.5 & 7.75 \end{tabular}} }

Despite the importance of limits to calculus, we often settle for an imprecise, intuitive understanding of what the limit of a function means. The precise definition of the limit omitted from a course like Math 1560, and left for later courses, such as Math 3500. For this course, we will use the following informal definition.

\smallskip

\definition{def:limit_informal}{Informal Definition of the Limit}
{\index{limit!definition}\index{limit!informal definition}

\indent Let $I$ be an open interval containing $c$, and let $f$ be a function defined on $I$, except possibly at $c$. 
We say that the \sword{limit of $f(x)$, as $x$ approaches $c$, is $L$}, and write 
\[
\lim_{x\rightarrow c} f(x) = L,
\]
if we can make the value of $f(x)$ arbitrarily close to $L$ by choosing $x\neq c$ sufficiently close to $c$.}


The formal definition of the limit makes precise the meaning of the phrases ``arbitrarily close'' and ``sufficiently close''. The problem with the definition we have given is that, while it gives an intuitive understanding of the meaning of the limit, it's of no use for \textit{proving} theorems about limits. In Section \ref{sec:limit_analytically} we will state (but not prove) several theorems about limits which will allow use to compute their values analytically, without recourse to tables of values.

In the next section we give the formal definition of the limit and begin our study of finding limits analytically. Your Math 1560 instructor will likely choose to skip Section \ref{sec:limit_def}, in which case it can be considered optional reading for the interested reader.  In the following exercises, we continue our introduction and approximate the value of limits.\\

\printexercises{exercises/01_01_exercises}

%\clearpage
