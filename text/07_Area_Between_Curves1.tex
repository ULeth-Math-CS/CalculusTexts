%\noindent\begin{minipage}{\specialboxlength}
%%We begin this chapter with a reminder of a few key concepts from Chapter \ref{chapter:integration}. Let $f$ be a continuous function on $[a,b]$ which is partitioned into $n$ subintervals as 
%%$$a<x_1 < x_2 < \cdots < x_n<x_{n+1}=b.$$ Let $\dx_i$ denote the length of the $i^\text{ th}$ subinterval, and let $c_i$ be any $x$-value in that subinterval. Definition \ref{def:rie_sum} states that the sum $$\sum_{i=1}^n f(c_i)\dx_i$$ is a \textit{Riemann Sum.} Riemann Sums are often used to approximate some quantity (area, volume, work, pressure, etc.). The \textit{approximation} becomes \textit{exact} by taking the limit 
%%$$\lim_{||\dx_i||\to0} \sum_{i=1}^n f(c_i)\dx_i,$$ where $||\dx_i||$ the length of the largest subinterval in the partition. Theorem \ref{thm:riemann_sum} connects limits of Riemann Sums to definite integrals:
%%$$\lim_{||\dx_i||\to0} \sum_{i=1}^n f(c_i)\dx_i = \int_a^b f(x)\ dx.$$ Finally, the Fundamental Theorem of Calculus states how definite integrals can be evaluated using antiderivatives. 
%%
%%This chapter employs the following technique to a variety of applications. Suppose the value $Q$ of a quantity is to be calculated. We first approximate the value of $Q$ using a Riemann Sum, then find the exact value via a definite integral. We spell out this technique in the following Key Idea.
%%\end{minipage}
%%\enlargethispage{20\baselineskip}
%%
%%\setboxwidth{100pt}
%%\keyidea{idea:app_of_defint}{Application of Definite Integrals Strategy}
%%{Let a quantity be given whose value $Q$ is to be computed.\index{integration!general application technique}
%%\begin{enumerate}
%%\item		Divide the quantity into $n$ smaller ``subquantities'' of value $Q_i$.
%%\item		Identify a variable $x$ and function $f(x)$ such that each subquantity can be approximated with the product $f(c_i)\dx_i$, where $\dx_i$ represents a small change in $x$. Thus $Q_i \approx f(c_i)\dx_i$. A sample approximation $f(c_i)\dx_i$ of $Q_i$ is called a \textit{differential element}.
%%\item		Recognize that $\ds Q= \sum_{i=1}^n Q_i \approx \sum_{i=1}^n f(c_i)\dx_i$, which is a Riemann Sum.
%%\item		Taking the appropriate limit gives $\ds Q = \int_a^b f(x)\ dx$
%%\end{enumerate}
%%}
%%\restoreboxwidth
%%
%%This Key Idea will make more sense after we have had a chance to use it several times. We begin with Area Between Curves, which we addressed briefly in Section \ref{chapter:integration}.\ref{sec:FTC}.
%%\clearpage

%We begin this section with a reminder of a few key concepts from Chapter \ref{chapter:integration}. Let $f$ be a continuous function on $[a,b]$ which is partitioned into $n$ equally spaced subintervals as 
%$$a<x_1 < x_2 < \cdots < x_n<x_{n+1}=b.$$ Let $\dx=(b-a)/n$ denote the length of the  subintervals, and let $c_i$ be any $x$-value in the $i^\text{ th}$ subinterval. Definition \ref{def:rie_sum} states that the sum $$\sum_{i=1}^n f(c_i)\dx$$ is a \textit{Riemann Sum.} Riemann Sums are often used to approximate some quantity (area, volume, work, pressure, etc.). The \textit{approximation} becomes \textit{exact} by taking the limit 
%$$\lim_{n\to\infty} \sum_{i=1}^n f(c_i)\dx.$$ Theorem \ref{thm:riemann_sum} connects limits of Riemann Sums to definite integrals:
%$$\lim_{n\to\infty} \sum_{i=1}^n f(c_i)\dx = \int_a^b f(x)\ dx.$$ Finally, the Fundamental Theorem of Calculus states how definite integrals can be evaluated using antiderivatives. 

%This chapter employs the following technique to a variety of applications. Suppose the value $Q$ of a quantity is to be calculated. We first approximate the value of $Q$ using a Riemann Sum, then find the exact value via a definite integral. We spell out this technique in the following Key Idea.
%\end{minipage}
%\enlargethispage{20\baselineskip}

%\setboxwidth{0pt}
%\keyidea{idea:app_of_defint}{Application of Definite Integrals Strategy}
%{Let a quantity be given whose value $Q$ is to be computed.\index{integration!general application technique}
%\begin{enumerate}
%\item		Divide the quantity into $n$ smaller ``subquantities'' of value $Q_i$.
%\item		Identify a variable $x$ and function $f(x)$ such that each subquantity can be approximated with the product $f(c_i)\dx$, where $\dx$ represents a small change in $x$. Thus $Q_i \approx f(c_i)\dx$. A sample approximation $f(c_i)\dx$ of $Q_i$ is called a \textit{differential element}.
%\item		Recognize that $\ds Q= \sum_{i=1}^n Q_i \approx \sum_{i=1}^n f(c_i)\dx$, which is a Riemann Sum.
%\item		Taking the appropriate limit gives $\ds Q = \int_a^b f(x)\ dx$
%\end{enumerate}
%}
%\restoreboxwidth

%\noindent\begin{minipage}{\specialboxlength}
%This Key Idea will make more sense after we have had a chance to use it several times. We begin with Area Between Curves, which we addressed briefly in Section \ref{chapter:integration}.\ref{sec:FTC}.
%\end{minipage}
%\clearpage

\section{Area Between Curves}\label{sec:ABC1}

We are often interested in knowing the area of a region. Forget momentarily that we addressed this already in Section  \ref{chapter:integration}.\ref{sec:FTC} and approach it instead using the partitioning technique used when we first introduced the concept of the definite integral.

Let $Q$ be the area of a region bounded by continuous functions $f$ and $g$. If we break the region into many subregions, we have an obvious equation:

\hfill Total Area = sum of the areas of the subregions. \hfill \null

The issue to address next is how to systematically break a region into subregions. A graph will help. Consider Figure \ref{fig:abcintro1} (a) where a region between two curves is shaded. While there are many ways to break this into subregions, one particularly efficient way is to ``slice'' it vertically, as shown in Figure \ref{fig:abcintro1} (b), into $n$ equally spaced slices. 

We now approximate the area of a slice. Again, we have many options, but using a rectangle seems simplest. Picking any $x$-value $c_i$ in the $i^\text{ th}$ slice, we set the height of the rectangle to be $f(c_i)-g(c_i)$, the difference of the corresponding $y$-values. The width of the rectangle is a small difference in $x$-values, which we represent with $\dx$. Figure \ref{fig:abcintro1} (c) shows sample points $c_i$ chosen in each subinterval and appropriate rectangles drawn. (Each of these rectangles represents a differential element.) Each slice has an area approximately equal to $\big(f(c_i)-g(c_i)\big)\dx$; hence, the total area is approximately the Riemann Sum
$$Q = \sum_{i=1}^n \big(f(c_i)-g(c_i)\big)\dx.$$
Taking the limit as $n\to \infty$ gives the exact area as $\int_a^b \big(f(x)-g(x)\big)\ dx.$

\mtable{.6}{Subdividing a region into vertical slices and approximating the areas with rectangles.}{fig:abcintro1}{\small%
\begin{tabular}{c}%
\myincludegraphics{figures/figabcintroa} \\ (a)\\
\myincludegraphics{figures/figabcintrob} \\ (b)\\
\myincludegraphics{figures/figabcintroc} \\ (c)
\end{tabular}%
}

\theorem{thm:areabetweencurves1}{Area Between Curves\quad (restatement of Theorem \ref{thm:areabtwncurves})}
{Let $f(x)$ and $g(x)$ be continuous functions defined on $[a,b]$ where $f(x)\geq g(x)$ for all $x$ in $[a,b]$. The area of the region bounded by the curves $y=f(x)$, $y=g(x)$ and the lines $x=a$ and $x=b$ is \index{integration!area between curves}
$$\int_a^b \big(f(x)-g(x)\big)\ dx.$$
}

\example{ex_abc1_1}{Finding area enclosed by curves}{
Find the area of the region bounded by $f(x) = \sin x+2$, $g(x) = \frac12\cos (2x)-1$, $x=0$ and $x=4\pi$, as shown in Figure \ref{fig:abc1_1}.}
{The graph verifies that the upper boundary of the region is given by $f$ and the lower bound is given by $g$. Therefore the area of the region is the value of the integral
\begin{align*} 
\int_0^{4\pi} \big(f(x)- g(x)\big)\ dx & = \int_0^{4\pi} \Big(\sin x+2 - \big(\frac12\cos (2x)-1\big)\Big)\ dx \\
		&= -\cos x -\frac14\sin(2x)+3x\Big|_0^{4\pi}\\
		&=	12\pi \approx 37.7\ \text{units}^2.
\end{align*}
\vskip-\baselineskip
\mfigure{.2}{Graphing an enclosed region in Example \ref{ex_abc1_1}.}{fig:abc1_1}{figures/figabc1}
}\\

\example{ex_abc21}{Finding total area enclosed by curves}{
Find the total area of the region enclosed by the functions $f(x) = -2x+5$ and $g(x) = x^3-7x^2+12x-3$ as shown in Figure \ref{fig:abc21}.}
{\mfigure{.55}{Graphing a region enclosed by two functions in Example \ref{ex_abc21}.}{fig:abc21}{figures/figabc2}
A quick calculation shows that $f=g$ at $x=1, 2$ and 4. One can proceed thoughtlessly by computing $\ds \int_1^4\big(f(x)-g(x)\big)\ dx$, but this ignores the fact that on $[1,2]$, $g(x)>f(x)$. (In fact, the thoughtless integration returns $-9/4$, hardly the expected value of an \textit{area}.) Thus we compute the total area by breaking the interval $[1,4]$ into two subintervals, $[1,2]$ and $[2,4]$ and using the proper integrand in each.
\begin{align*}
\text{Total Area} &= \int_1^2 \big(g(x)-f(x)\big)\ dx + \int_2^4\big(f(x)-g(x)\big)\ dx\\
			&= \int_1^2 \big(x^3-7x^2+14x-8\big) \ dx + \int_2^4\big(-x^3+7x^2-14x+8\big)\ dx\\
			&= 5/12 + 8/3 \\
			&= 37/12 = 3.083\ \text{units}^2.
\end{align*}		
\vskip-\baselineskip
}\\

The previous example makes note that we are expecting area to be \textit{positive}. When first learning about the definite integral, we interpreted it as ``signed area under the curve,'' allowing for ``negative area.'' That doesn't apply here; area is to be positive.

The previous example also demonstrates that we often have to break a given region into subregions before applying Theorem \ref{thm:areabetweencurves1}. The following example shows another situation where this is applicable, along with an alternate view of applying the Theorem.\\

\example{ex_abc31}{Finding area: integrating with respect to $y$}{
Find the area of the region enclosed by the functions $y=\sqrt{x}+2$, $y=-(x-1)^2+3$ and $y=2$, as shown in Figure \ref{fig:abc31}.}
{We give two approaches to this problem. In the first approach, we notice that the region's ``top'' is defined by two different curves. On $[0,1]$, the top function is $y=\sqrt{x}+2$; on $[1,2]$, the top function is $y=-(x-1)^2+3$. 
\mfigure{.3}{Graphing a region for Example \ref{ex_abc31}.}{fig:abc31}{figures/figabc3}
Thus we compute the area as the sum of two integrals:
\begin{align*}
\text{Total Area} &= \int_0^1 \Big(\big(\sqrt{x}+2\big)-2\Big)\ dx + \int_1^2 \Big(\big(-(x-1)^2+3\big)-2\Big)\ dx \\
									&= 2/3 + 2/3\\
									&=4/3.
\end{align*}

The second approach is clever and very useful in certain situations. We are used to viewing curves as functions of $x$; we input an $x$-value and a $y$-value is returned. Some curves can also be described as functions of $y$: input a $y$-value and an $x$-value is returned. We can rewrite the equations describing the boundary by solving for $x$:
	$$y=\sqrt{x}+2 \quad \Rightarrow\quad x=(y-2)^2$$
	$$y=-(x-1)^2+3 \quad \Rightarrow \quad x=\sqrt{3-y}+1.$$

	
Figure \ref{fig:abc3b1} shows the region with the boundaries relabelled. A differential element, a horizontal rectangle, is also pictured.	The width of the rectangle is a small change in $y$: $\Delta y$. The height of the rectangle is a difference in $x$-values. The ``top'' $x$-value is the largest value, i.e., the rightmost. The ``bottom'' $x$-value is the smaller, i.e., the leftmost. Therefore the height of the rectangle is $$\big(\sqrt{3-y}+1\big) - (y-2)^2.$$

The area is found by integrating the above function with respect to $y$ with the appropriate bounds. We determine these by considering the $y$-values the region occupies. It is bounded below by $y=2$, and bounded above by $y=3$. That is, both the ``top'' and ``bottom'' functions exist on the $y$ interval $[2,3]$. Thus
\begin{align*}
\text{Total Area} &= \int_2^3 \big(\sqrt{3-y}+1 - (y-2)^2\big)\ dy \\
			&= \Big(-\frac23(3-y)^{3/2}+y-\frac13(y-2)^3\Big)\Big|_2^3 \\
			&= 4/3.
\end{align*}
\vskip-\baselineskip
}\\

This calculus--based technique of finding area can be useful even with shapes that we normally think of as ``easy.'' Example \ref{ex_abc41} computes the area of a triangle. While the formula ``$\frac12\times\text{base}\times\text{height}$'' is well known, in arbitrary triangles it can be nontrivial to compute the height. Calculus makes the problem simple.\\

\mfigure{.8}{The region used in Example \ref{ex_abc31} with boundaries relabelled as functions of $y$.}{fig:abc3b1}{figures/figabc3b}

\example{ex_abc41}{Finding the area of a triangle}{
Compute the area of the regions bounded by the lines 

\noindent $y=x+1$, $y=-2x+7$ and $y=-\frac12x+\frac52$, as shown in Figure \ref{fig:abc41}.}
{Recognize that there are two ``top'' functions to this region, causing us to use two definite integrals.
\begin{align*}
\text{Total Area} &= \int_1^2\big((x+1)-(-\frac12x+\frac52)\big)\ dx + \int_2^3\big((-2x+7)-(-\frac12x+\frac52)\big)\ dx \\
						&= 3/4+3/4\\
						&=3/2.
\end{align*}
\mfigure{.3}{Graphing a triangular region in Example \ref{ex_abc41}.}{fig:abc41}{figures/figabc4}
We can also approach this by converting each function into a function of $y$. This also requires 2 integrals, so there isn't really any advantage to doing so. We do it here for demonstration purposes.

The ``top'' function is always $x=\frac{7-y}2$ while there are two ``bottom'' functions. Being mindful of the proper integration bounds, we have
\begin{align*}
\text{Total Area} &= \int_1^2\big(\frac{7-y}2 - (5-2y)\big)\ dy + \int_2^3\big(\frac{7-y}2-(y-1)\big)\ dy \\
			&= 3/4 + 3/4\\
			&= 3/2.
\end{align*}
Of course, the final answer is the same. (It is interesting to note that the area of all 4 subregions used is 3/4. This is coincidental.)
}\\

While we have focused on producing exact answers, we are also able to make approximations using the principle of Theorem \ref{thm:areabetweencurves1}. The integrand in the theorem is a distance (``top minus bottom''); integrating this distance function gives an area. By taking discrete measurements of distance, we can approximate an area using numerical integration techniques developed in Section \ref{sec:numerical_integration}. The following example demonstrates this.\\

\example{ex_abc51}{Numerically approximating area}{
To approximate the area of a lake, shown in Figure \ref{fig:abc51} (a),  the ``length'' of the lake is measured at 200-foot increments as shown in Figure \ref{fig:abc51} (b), where the lengths are given in hundreds of feet. Approximate the area of the lake.}
{The measurements of length can be viewed as measuring ``top minus bottom'' of two functions. The exact answer is found by integrating $\ds \int_0^{12} \big(f(x)-g(x)\big)\ dx$, but of course we don't know the functions $f$ and $g$. Our discrete measurements instead allow us to approximate.

\mtable{.55}{(a) A sketch of a lake, and (b) the lake with length measurements.}{fig:abc51}{\noindent\begin{tabular}{c}\myincludegraphics{figures/figabc5b}\\(a)\\ \myincludegraphics{figures/figabc5} \\ (b)\end{tabular}}
We have the following data points:
$$(0,0),\ (2,2.25),\ (4,5.08),\ (6,6.35),\ (8,5.21),\ (10,2.76),\ (12,0).$$
We also have that $\dx=\frac{b-a}{n} = 2$, so Simpson's Rule gives
\begin{align*}
\text{Area}&\approx \frac{2}{3}\Big(1\cdot0+4\cdot2.25+2\cdot5.08+4\cdot6.35+2\cdot5.21+4\cdot2.76+1\cdot0\Big)\\
			&= 44.01\overline{3} \ \text{units}^2.
\end{align*}

Since the measurements are in hundreds of feet, units$^2 = (100\ \text{ft})^2 = 10,000\ \text{ft}^2$, giving a total area of $440,133\ \text{ft}^2$. (Since we are approximating, we'd likely say the area was about $440,000\ \text{ft}^2$, which is a little more than 10 acres.)
}
\\

In the next section we apply our applications--of--integration techniques to finding the volumes of certain solids.
\printexercises{exercises/07_01_exercises1}