\section{Integrals as solutions}\label{sec:ODE_int_sol}


A first order ODE is an equation of the form
\begin{equation*}
\frac{dy}{dx} = f(x,y) ,
\end{equation*}
or just
\begin{equation*}
y\primeskip ' = f(x,y) .
\end{equation*}
In general, there is no simple formula or procedure one can follow to find
solutions.
In the next few lectures we will look at special cases where solutions are not
difficult to obtain.
In this section, let us assume that $f$ is a function of $x$ alone,
that is, the equation is
\begin{equation} \label{ias:inteq}
y\primeskip ' = f(x) .
\end{equation}
We could just integrate (antidifferentiate) both sides with respect to $x$.
\begin{equation*}
\int y\primeskip '(x) ~dx = \int f(x) ~dx + C ,
\end{equation*}
that is
\begin{equation*}
y(x) = \int f(x) ~dx + C .
\end{equation*}
This $y(x)$ is actually the general solution.
So to solve \eqref{ias:inteq},
we find some antiderivative of $f(x)$
and then we add an arbitrary constant to get the general solution.

\medskip

Now is a good time to discuss a point about
calculus notation and terminology.  Calculus
textbooks muddy the waters by talking about the integral as primarily the
so-called indefinite integral.  The \index{indefinite integral}
is really the \sword{antiderivative}\index{antiderivative}
(in fact the whole one-parameter family
of antiderivatives).  There really exists only one integral and that
is the definite integral.
The only reason for the indefinite integral notation is that we can always
write an antiderivative as a (definite) integral.  That is, by the fundamental
theorem of calculus we can always write
$\int f(x) ~dx + C$ as
\begin{equation*}
\int_{x_0}^x f(t) ~dt + C .
\end{equation*}
Hence the terminology \emph{to integrate} when we may really mean
\emph{to antidifferentiate}.
Integration is just one way to compute the
antiderivative (and it is a way that always works, see the following
examples).  Integration is defined as the area under the graph, it
only happens to also compute antiderivatives.
For sake of consistency, we will keep using the
indefinite integral notation when we want an antiderivative,
and you should \emph{always} think of the definite integral.\\

\example{egg1}{Finding a general solution}{
Find the general solution of $y\primeskip ' = 3 x^2$.}
{Elementary calculus tells us
that the general solution must be $y = x^3 + C$.  Let us check by
differentiating:
$y\primeskip ' = 3x^2$.  We have gotten \emph{precisely} our equation back.
}\\

Normally, we also have an initial condition such as $y(x_0) = y_0$
for some two numbers $x_0$ and $y_0$ ($x_0$ is usually 0, but not always).
We can then write the solution as a definite integral in a nice way.
Suppose our problem is $y\primeskip ' = f(x)$, $y(x_0) = y_0$.  Then the solution is
\begin{equation} \label{int:eqdef}
y(x) = \int_{x_0}^x f(s) ~ds + y_0 .
\end{equation}
Let us check!
We compute
$y\primeskip ' = f(x)$, via the fundamental theorem of calculus, and by Jupiter, $y$ is a
solution.  Is it the one satisfying the initial condition?  Well,
$y(x_0) = \int_{x_0}^{x_0} f(x)~dx + y_0 = y_0$.  It is!

Do note that the definite integral and the indefinite integral
(antidifferentiation) are completely different beasts.  The definite integral
always evaluates to a number.  Therefore, \eqref{int:eqdef} is a formula we
can plug into the calculator or a computer, and it will be happy to calculate
specific values for us.  We will easily be able to plot the
solution and work with it just like with any other function.
It is not so crucial to always find a
closed form for the antiderivative.\\

\example{egg2}{An ODE with no closed-form solution}{
Solve
\begin{equation*}
y\primeskip ' = e^{-x^2}, \qquad y(0) = 1 .
\end{equation*}
}
{By the preceding discussion, the solution must be
\begin{equation*}
y(x) = \int_0^x e^{-s^2} ~ds + 1 .
\end{equation*}
Here is a good way to make fun of your friends taking second semester
calculus.  Tell them to
find the closed form solution.  Ha ha ha (bad math joke).  It is
not possible (in closed form).
There is absolutely nothing wrong with writing the solution as a
definite integral.
This particular integral
is in fact very important
in statistics.
}\\

Using this method, we can also solve equations of the form
\begin{equation*}
y\primeskip ' = f(y) .
\end{equation*}
Let us write the equation in \index{Leibniz notation}.
\begin{equation*}
\frac{dy}{dx} = f(y) .
\end{equation*}
Now we use the inverse function theorem from calculus
to switch the roles of $x$ and $y$
to obtain
\begin{equation*}
\frac{dx}{dy} = \frac{1}{f(y)} .
\end{equation*}
What
we are doing seems like algebra with $dx$ and $dy$.
It is tempting to just do algebra with $dx$
and $dy$ as if they were numbers.  And in this case it does work.  Be
careful,
however, as this sort of hand-waving calculation can lead to trouble,
especially when
more than one independent variable is involved.
At this point we can simply integrate,
\begin{equation*}
x(y) = \int \frac{1}{f(y)} ~dy + C .
\end{equation*}
Finally, we try to solve for $y$.\\

\example{egg3}{Solving the exponential growth equation}{
Previously, we guessed $y\primeskip ' = ky$ (for some $k > 0$) has the solution
$y=Ce^{kx}$.  We can now find the solution without guessing.}
{First we note that $y=0$ is a solution.
Henceforth, we assume $y\not= 0$.  We write
\begin{equation*}
\frac{dx}{dy} = \frac{1}{ky} .
\end{equation*}
We integrate to obtain
\begin{equation*}
x(y) = x = \frac{1}{k} \ln \, \lvert y \rvert + D,
\end{equation*}
where $D$ is an arbitrary constant.
Now we solve for $y$ (actually for $\lvert y \rvert$).
\begin{equation*}
\lvert y \rvert =
e^{kx-kD} = 
e^{-kD} e^{k x} .
\end{equation*}
If we replace $e^{-kD}$ with an arbitrary constant $C$ we can
get rid of the absolute value bars (which we can do as $D$ was arbitrary).  In
this way, we
also incorporate the solution $y=0$.  We get the same general solution as
we guessed before, $y = Ce^{kx}$.
}\\

\example{egg4}{Solving an ODE by integration}{
Find the general solution of
$y\primeskip ' = y^2$.}
{
First we note that $y=0$ is a solution.  We can now assume that $y \not= 0$.
Write
\begin{equation*}
\frac{dx}{dy} = \frac{1}{y^2} .
\end{equation*}
We integrate to get
\begin{equation*}
x = \frac{-1}{y} + C .
\end{equation*}
We solve for $y = \frac{1}{C-x}$.
So the general solution is
\begin{equation*}
y = \frac{1}{C-x} \qquad \text{or} \qquad y = 0.
\end{equation*}
Note the singularities of the solution.  If for example $C=1$, then the
solution ``blows up'' as we approach $x=1$.  Generally,
it is hard to tell
from just looking at the equation itself how the solution is going to behave.
The equation $y\primeskip ' = y^2$ is very nice and defined everywhere, but
the solution is only defined on some interval $(-\infty, C)$ or
$(C, \infty)$.
}\\

Classical problems leading to differential equations solvable by integration
are problems 
dealing with \index{velocity},
\index{acceleration} and \index{distance}.  You have surely seen these
problems before in your calculus class.\\

\example{egg5}{Finding the distance travelled}{
Suppose a car drives at a speed $e^{t/2}$ metres per second,
where $t$ is time in seconds.
How far did the car get in 2 seconds (starting at $t=0$)?  How far in 10 seconds?}
{
Let $x$ denote the distance the car travelled.
The equation is
\begin{equation*}
x\primeskip' = e^{t/2} .
\end{equation*}
We can just integrate this equation to get that
\begin{equation*}
x(t) = 2 e^{t/2} + C . 
\end{equation*}
We still need to figure out $C$.  We know that when $t=0$, then
$x=0$.  That is, $x(0) = 0$.  So
\begin{equation*}
0 = x(0) = 2e^{0/2} + C = 2 + C .
\end{equation*}
Thus $C = -2$ and 
\begin{equation*}
x(t) = 2 e^{t/2} - 2 .
\end{equation*}
Now we just plug in to get where the car is at 2 and at 10 seconds.
We obtain
\begin{equation*}
x(2) = 2e^{2/2} - 2 \approx 3.44 \text{ metres} ,
\qquad
x(10) = 2e^{10/2} - 2 \approx 294 \text{ metres} .
\end{equation*}
}\\

\example{egg6}{Another car problem}{
Suppose that the car accelerates at a rate of $\unitfrac[t^2]{m}{s^2}$.
At time $t=0$ the car is at the 1 metre mark and is travelling at
\unitfrac[10]{m}{s}.  Where is the car at time $t=10$?}
{
Well this is actually a second order problem.  If $x$ is the distance
travelled, then $x\primeskip'$ is the velocity, and $x\primeskip''$ is the acceleration.
The equation with initial conditions is
\begin{equation*}
x\primeskip'' = t^2 , \qquad x(0) = 1 , \qquad x\primeskip'(0) = 10 .
\end{equation*}
What if we say $x\primeskip' = v$.  Then we have the problem
\begin{equation*}
v' = t^2, \qquad v(0) = 10 .
\end{equation*}
Once we solve for $v$, we can integrate and find $x$.
}\\

\medskip

\noindent{\bf Exercise:} Solve for $v$, and then solve for $x$.  Find $x(10)$ to answer the
question.

\medskip

\printexercises{exercises/15_02_exercises}

