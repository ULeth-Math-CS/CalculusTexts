\section{Linear equations and the integrating factor}\label{sec:lineareq}



One of the most important types of equations we will learn how to solve are
the so-called
\sword{linear equations}\index{linear differential equation}\index{differential equation!linear}
In this lecture we focus on the
\sword{first order linear equation}\index{first order linear differential equation}\index{linear differential equation!first order}.
A first order equation is linear if we can put it
into the form:
\begin{equation} \label{lineq:eq1}
y' + p(x) y = f(x) .
\end{equation}
Here the word
``linear'' means linear in $y$ and $y'$;
no higher powers nor functions of $y$ or $y'$ appear.
The dependence on $x$ can be more
complicated.

Solutions of linear equations have nice properties.  For example, the
solution exists wherever $p(x)$ and $f(x)$ are defined, and has the same
regularity (read: it is just as nice).  But most importantly for us right now,
there is a method for solving linear first order equations.

The trick is to rewrite the left hand side
of \eqref{lineq:eq1} as a derivative of a product of $y$ with another
function.
To this end
we find a function $r(x)$ such that
\begin{equation*}
r(x) y' + r(x) p(x) y = \frac{d}{dx}\Bigl[ r(x) y \Bigr] .
\end{equation*}
This is the left hand side of
\eqref{lineq:eq1} multiplied by $r(x)$.  So if we multiply \eqref{lineq:eq1} by
$r(x)$, we obtain
\begin{equation*}
\frac{d}{dx}\Bigl[ r(x) y \Bigr] = r(x)f(x) .
\end{equation*}
Now we integrate both sides.
The right hand side does not depend on $y$ and the left hand side
is written as a derivative of a function.  Afterwards, we solve for $y$.
The function $r(x)$ is called the \sword{integrating factor}\index{integrating factor} and the
method is called the \sword{integrating factor method}.

We are looking for a function $r(x)$, such that if
we differentiate it, we get the same function back multiplied by $p(x)$.
That seems like a job for the exponential function!  Let
\begin{equation*}
r(x) = e^{\int p(x) \,dx} .
\end{equation*}
We compute:
\begin{align*}
y' + p(x) y &= f(x) , \\
e^{\int p(x) \,dx} y' + e^{\int p(x) \,dx} p(x) y & = e^{\int p(x) \,dx} f(x) , \\
\frac{d}{dx}\left[ e^{\int p(x) \,dx} y \right] & = e^{\int p(x) \,dx} f(x) , \\
e^{\int p(x) \,dx} y & = \int e^{\int p(x) \,dx} f(x) ~dx + C , \\
y & = e^{-\int p(x) \,dx} \left( \int e^{\int p(x) \,dx} f(x) ~dx + C \right) .
\end{align*}

Of course, to get a closed form formula for $y$,
we need to be able to find a
closed form formula for the integrals appearing above.\\

\example{egl1}{A linear equation with a closed form solution}{
Solve
\begin{equation*}
y' + 2xy = e^{x-x^2}, \qquad y(0) = -1 .
\end{equation*}}
{
First note that $p(x) = 2x$ and $f(x) = e^{x-x^2}$.
The integrating factor is $r(x) = e^{\int p(x)\, dx} = e^{x^2}$.
We multiply both sides of the equation by $r(x)$ to get
\begin{align*}
e^{x^2} y' + 2xe^{x^2}y & = e^{x-x^2} e^{x^2} , \\
\frac{d}{dx} \left[ e^{x^2} y \right] &= e^x .
\end{align*}
We integrate
\begin{align*}
e^{x^2} y &= e^x +C , \\
y &= e^{x-x^2} + C e^{-x^2} .
\end{align*}
Next, we solve for the initial condition $-1 = y(0) = 1 + C$, so $C=-2$.
The solution is
\begin{equation*}
y = e^{x-x^2} - 2 e^{-x^2} .
\end{equation*}
}\\

Note that we do not care which antiderivative we take when computing
$e^{\int p(x) dx}$.  You can always add a constant of integration,
but those constants
will not matter in the end.

\medskip

\noindent{\bf Exercise:} Try it!  Add a constant of integration to the integral in
the integrating factor and show that the solution you get in the end is the
same as what we got above.

\medskip

A piece of advice: Do not try to remember the formula itself, that is way too
hard.  It is easier to remember the process and repeat it.

Since we cannot always evaluate the integrals in closed form, it is useful to
know how to write the solution in definite integral form.  A definite
integral is something that
you can plug into a computer or a calculator.  Suppose we are given
\begin{equation*}
y' + p(x) y = f(x) , \qquad y(x_0) = y_0 .
\end{equation*}
Look at the solution and write the integrals
as definite integrals.
\begin{equation} \label{lei:defsol}
\boxed{
~~
y(x) = e^{-\int_{x_0}^x p(s)\, ds} \left( \int_{x_0}^x e^{\int_{x_0}^t p(s)\, ds}
f(t) ~dt + y_0 \right).
~~
}
\end{equation}
You should
be careful to properly use dummy variables here.  If you now plug such a
formula into a
computer or a calculator, it will be happy to give you numerical answers.

\medskip

\noindent{\bf Exercise:} Check that $y(x_0) = y_0$ in formula \eqref{lei:defsol}.

\medskip

\noindent {\bf Exercise:} Write the solution of the following problem
as a definite integral, but try to simplify as far as you can.  You will not
be able to find the solution in closed form.
\begin{equation*}
y' + y = e^{x^2-x}, \qquad y(0) = 10 .
\end{equation*}

\medskip

\noindent {\bf Remark:} Before we move on, we should note some interesting properties of linear
equations.  First, for the linear initial value problem
$y' + p(x) y = f(x)$, $y(x_0) = y_0$,
there is always an explicit formula \eqref{lei:defsol} for the
solution.  Second, it follows
from the formula \eqref{lei:defsol} that if $p(x)$
and $f(x)$ are continuous on some interval $(a,b)$, then the 
solution $y(x)$ exists and is differentiable on $(a,b)$.  Compare
with the simple nonlinear example we have seen previously, $y'=y^2$,
and compare to Theorem \ref{slope:picardthm}.

\medskip

Let us discuss a common
simple application of linear equations.
This type of 
problem is used often in real life.
For example, linear equations are used in
figuring out the concentration of
chemicals in bodies of water (rivers and lakes).\\


\example{egl3}{An application of linear ODEs}{
A 100 litre tank contains 10 kilograms of salt dissolved in 60 litres of
water.  Solution of water and salt (brine) with concentration of 0.1
kilograms per
litre is flowing in at the rate of 5 litres a minute.  The solution
in the tank is well stirred and flows out at a rate of 3 litres a minute.
How much salt is in the tank when the tank is full?
}
{Let us come up with the equation.  Let $x$ denote the kilograms of salt in the tank,
let $t$ denote the time in minutes.  For a small change $\Delta t$ in
time, the change in $x$ (denoted $\Delta x$) is approximately
\begin{equation*}
\Delta x \approx
(\text{rate in} \times \text{concentration in}) \Delta t - 
(\text{rate out} \times \text{concentration out}) \Delta t .
\end{equation*}
Dividing through by $\Delta t$ and
taking the limit $\Delta t \to 0$ we see that
\begin{equation*}
\frac{dx}{dt} =
(\text{rate in} \times \text{concentration in})  - 
(\text{rate out} \times \text{concentration out}) .
\end{equation*}
In our example, we have
\begin{align*}
\text{rate in} &= 5 , \\
\text{concentration in} &= 0.1 , \\
\text{rate out} &= 3 , \\
\text{concentration out} &= \frac{x}{\text{volume}} = \frac{x}{60+(5-3)t} .
\end{align*}
Our equation is, therefore,
\begin{equation*}
\frac{dx}{dt} =
(5 \times 0.1)  - 
\left(3 \frac{x}{60+2t}\right) .
\end{equation*}
Or in the form \eqref{lineq:eq1}
\begin{equation*}
\frac{dx}{dt} +
\frac{3}{60+2t} x
=
0.5 .
\end{equation*}

Let us solve.  The integrating factor is
\begin{equation*}
r(t) = \exp \left( \int \frac{3}{60+2t} dt  \right)
=
\exp \left( \frac{3}{2} \ln (60+2t) \right)
=
{(60+2t)}^{3/2} .
\end{equation*}

\mtable{.5}{The tank in Example \ref{egl3}}{lintank}{
\myincludegraphics[scale=.5]{figures/lin-tank}
}

We multiply both sides of the equation to get
\[
{(60+2t)}^{3/2} \frac{dx}{dt} +
{(60+2t)}^{3/2} \frac{3}{60+2t} x
 =
0.5{(60+2t)}^{3/2} ,
\]
and reversing the product rule gives us
\[
\frac{d}{dt}\left[
{(60+2t)}^{3/2} x \right]
 =
0.5{(60+2t)}^{3/2} ,
\]
so
\[
{(60+2t)}^{3/2} x
 =
\int 
0.5{(60+2t)}^{3/2}
dt
+C.
\]
Thus,
\begin{align*}
 x
& =
{(60+2t)}^{-3/2} \int 
\frac{
{(60+2t)}^{3/2}
}{2}
dt
+C{(60+2t)}^{-3/2} ,\\
& =
{(60+2t)}^{-3/2}
\frac{1}{10}{(60+2t)}^{5/2}
+C{(60+2t)}^{-3/2} ,\\
& =
\frac{60+2t}{10}
+C{(60+2t)}^{-3/2} .
\end{align*}

We need to find $C$.  We know that at $t=0$, $x=10$.  So
\begin{equation*}
10 = x(0)
=
\frac{60}{10}
+C{(60)}^{-3/2}
=
6
+C{(60)}^{-3/2} ,
\end{equation*}
or
\begin{equation*}
C=4 ({60}^{3/2}) \approx 1859.03 .
\end{equation*}

We are interested in $x$ when the tank is full.  So we note that the tank is
full when $60+2t = 100$, or when $t=20$.  So
\begin{equation*}
x(20)= 
\frac{60+40}{10}
+C{(60+40)}^{-3/2}
\approx
10
+1859.03 {(100)}^{-3/2}
\approx
11.86 .
\end{equation*}

The concentration at the end is approximately $\unitfrac[0.1186]{kg}{litre}$ and we started
with $\dfrac{1}{6}$ or $\unitfrac[0.167]{kg}{litre}$.
}

\printexercises{exercises/14_05_exercises}


